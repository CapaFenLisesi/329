\chapter{Wave Equation}
\section{Introduction}
The wave equation, which in one dimension takes the form
\begin{equation}\label{wave1d}
\frac{\partial^2\xi}{\partial t^2} = c^2\,\frac{\partial^2 \xi}{\partial x^2},
\end{equation}
occurs so frequently in physics that it is not necessary to enumerate examples.
Here, $\xi$ is usually some sort of displacement or perturbation, whereas $c$ is the
(constant) wave speed. The wave equation possesses the formal solution
\begin{equation}
\xi(x,t) = F(x-c\,t) + G(x+c\,t),
\end{equation}
where $F$ and $G$ are arbitrary functions. The above solution  represents arbitrarily
shaped wave pulses propagating  with speed $c$ in the $+x$ and $-x$ directions, respectively, without
changing shape.

The wave equation, which is second-order in space and time,
 can be written as two coupled first-order  equations
by defining the new variables $v=\partial\xi/\partial t$ and $\theta=-c\,\partial\xi/\partial x$.
Expressing Eq.~(\ref{wave1d}) in terms of these new variables, we obtain
\begin{eqnarray}
\frac{\partial v}{\partial t} + c\,\frac{\partial \theta}{\partial x} &=& 0,\\[0.5ex]
\frac{\partial \theta}{\partial t} + c\,\frac{\partial v}{\partial x} &=& 0.
\end{eqnarray}
Note that when solving the wave equation numerically it is generally preferable to write it
as a set of coupled first-order equations, as shown above.

\section{1-D Advection Equation}
The wave equation is closely related to the so-called {\em advection equation}, which
in one dimension takes the form
\begin{equation}\label{advection}
\frac{\partial u}{\partial t} = -v\,\frac{\partial u}{\partial x}.
\end{equation}
This equation describes the passive advection of some scalar field $u(x,t)$
carried along by a flow of constant speed $v$. Since the advection equation
is somewhat simpler than the wave equation, we shall discuss it first.
The advection equation
possesses the formal solution
\begin{equation}
u(x,t) = F(x-v\,t),
\end{equation}
where $F$ is an arbitrary function. This solution describes an  arbitrarily
shaped pulse which is swept along by the flow, at constant speed $v$, without
changing shape. 

We seek the solution of Eq.~(\ref{advection}) in the region $x_l\leq x\leq x_h$,
subject to the simple Dirichlet boundary conditions $u(x_l) = u(x_h)=0$.
As usual, we discretize in time on the uniform grid $t_n=t_0+n\,\delta t$, for $n=0,1,2,\cdots$.
Furthermore, in the  $x$-direction, we discretize on the uniform grid $x_i = x_l + i\,\delta x$, for
$i=0,N+1$, where $\delta x = (x_h-x_l)/(N+1)$. Adopting an explicit temporal differencing
scheme, and  a centered spatial differencing scheme,  Eq.~(\ref{advection}) yields
\begin{equation}\label{ftcs1}
\frac{u_i^{n+1} - u_i^n}{\delta t} = - v\,\frac{u_{i+1}^n-u_{i-1}^n}{2\,\delta x},
\end{equation}
where $u_i^n\equiv u(x_i,t_n)$. The above equation can be rewritten
\begin{equation}\label{ftcs}
u_i^{n+1} = u_i^n -\frac{C}{2}\,(u_{i+1}^n-u_{i-1}^n),
\end{equation}
where $C= v\,\delta t/\delta x$.

Let us perform a von Neumann stability analysis of the above differencing scheme.
Writing $u(x,t)= \hat{u}(t)\,\exp ({\rm i}\,k\,x)$, we obtain
$\hat{u}_i^{n+1}=A\,\hat{u}_i^n$, where
\begin{equation}
A = 1 - {\rm i} \,C\,\sin (k \,\delta x).
\end{equation}
Note that 
\begin{equation}
|A|^2 = 1 + C^2\,\sin^2(k\,\delta x) > 1.
\end{equation}
Thus, the magnitude of the amplification factor is greater than unity for
all $k$. This implies, unfortunately, that the  simple differencing scheme (\ref{ftcs})
is {\em unconditionally unstable}.

\section{Lax Scheme}
The instability in the differencing scheme (\ref{ftcs}) can be fixed by replacing
$u_i^n$ on the right-hand side by the {\em spatial average} of $u$ taken over the neighbouring grid
points. Thus, we obtain
\begin{equation}\label{lax}
u_i^{n+1} = \frac{1}{2}\,(u_{i+1}^n+u_{i-1}^n) -\frac{C}{2}\,(u_{i+1}^n-u_{i-1}^n),
\end{equation}
which is known as the {\em Lax} scheme. A von Neumann stability analysis
of the Lax scheme yields the following expression for the amplification
factor:
\begin{equation}
A = \cos (k\,\delta x) - {\rm i} \,C\,\sin (k \,\delta x).
\end{equation}
Now
\begin{equation}\label{laxa}
|A|^2 = 1 - (1-C^2)\,\sin^2(k\,\delta x).
\end{equation}
It follows that the Lax scheme is unconditionally stable ({\em i.e.}, $|A| < 1$
for all $k$), provided that $C<1$. From the definition of $C$, the
inequality $C<1$ can also be written
\begin{equation}
\delta t < \frac{\delta x}{v}.
\end{equation}
This is the famous Courant-Friedrichs-Lewy (or CFL) stability criterion.
In fact, {\em all} stable {\em explicit} differencing schemes for solving the advection equation
are subject to the CFL constraint, which determines  the maximum allowable time-step.

Listed below is a routine which solves the 1-d advection equation
via the Lax method.
{\small\begin{verbatim}
// Lax1D.cpp

// Function to evolve advection equation in 1-d:

//  du / dt + v du / dx = 0  for  xl <= x <= xh

//  u = 0  at x=xl and x=xh

// Array u assumed to be of extent N+2.

// Now, ith element of array corresponds to

//  x_i = xl + i * dx    i=0,N+1

// Here, dx = (xh - xl) / (N+1) is grid spacing.

// Function evolves u by single time-step.

// C = v dt / dx, where dt is time-step.

// Uses Lax scheme.

#include <blitz/array.h>

using namespace blitz;

void Lax1D (Array<double,1>& u, double C)
{
  // Set N. Declare local array.
  int N = u.extent(0) - 2;
  Array<double,1> u0(N+2);

  // Evolve u
  u0 = u;
  for (int i = 1; i <= N; i++)
    u(i) = 0.5 * (u0(i+1) + u0(i-1)) - 0.5 * C * (u0(i+1) - u0(i-1));

  // Set boundary conditions
  u(0) = 0.;
  u(N+1) = 0.;
}
\end{verbatim}}

Figure~\ref{decay} shows an example calculation which uses the above routine to
advect a Gaussian pulse. The initial condition is
\begin{equation}\label{initcn}
u(x,0) = \exp[-100\,(x-0.5)^2],
\end{equation}
and the calculation is performed with $v=1$, $\delta t = 2.49\times 10^{-3}$, and $N=200$.
Furthermore, $x_l = v\,t$ and $x_h = 1 + v\,t$.
Note that $C=0.5$ for these parameters. It can be
seen that the pulse is advected at the correct speed: {\em i.e.}, the
pulse  appears approximately stationary when  plotted versus $x-v\,t$. 
Unfortunately, the pulse does not remain the same shape (as it should). Instead, the
pulse becomes gradually lower and wider as it propagates, and eventually
diffuses away entirely.

\begin{figure}
\epsfysize=3in
\centerline{\epsffile{Chapter07/decay.eps}}
\caption{\em Advection of a 1-d Gaussian pulse.
Numerical  calculation performed using
$v=1$, $\delta t = 2.49\times 10^{-3}$, and $N=200$.  The
solid curve shows the initial condition at $t=0$, the short-dashed curve  the numerical solution
at $t=1$, the long-dashed curve  the numerical solution
at $t=2$, and the dot-dashed  curve  the numerical solution at 
$t=3$. }\label{decay}
\end{figure}

It is clear, from the above calculation, that the Lax scheme introduces a spurious
dispersion effect into the advection problem. We can understand the origin
of this effect by attempting a Fourier solution, $u(x,t)=\hat{u}_k(t)\,\exp(\,{\rm i}\,k\,x)$,
of Eq.~(\ref{advection}). We easily obtain
\begin{equation}
\hat{u}_k(t) = \hat{u}_k(0)\,\exp(-{\rm i}\,k\,v\,t).
\end{equation}
Note that $|\hat{u}_k|$ is {\em constant} in time for all values of $k$. In other words, the
{\em amplitudes} of the Fourier harmonics of a true solution of the advection equation
remain constant in time---it is the {\em phases} of the harmonics which
evolve. Let us now examine Eq.~(\ref{laxa}). It can be seen that, provided  the CFL condition
$C<1$ is satisfied,  the magnitude of the amplification factor, $|A|$,
is {\em less than unity}  for all Fourier harmonics. In other words,
the Lax differencing scheme causes the Fourier harmonics to {\em decay} in time. It is this unphysical
attenuation of the Fourier harmonics which gives rise to the strong dispersion effect
illustrated in Fig.~\ref{decay}.

\begin{figure}
\epsfysize=3in
\centerline{\epsffile{Chapter07/inst.eps}}
\caption{\em Advection of a 1-d Gaussian pulse.
Numerical  calculation performed using
$v=1$, $\delta t = 9.95\times 10^{-3}$, and $N=200$.  The
dotted curve shows the initial condition at $t=0.00$, the short-dashed curve  the numerical solution
at $t=0.15$, the long-dashed curve  the numerical solution
at $t=0.30$, and the solid  curve  the numerical solution at 
$t=0.37$. }\label{inst}
\end{figure}

Figure~\ref{inst} shows a calculation made using the Lax scheme in
which the CFL condition is violated. This
calculation is identical to the one discussed previously, except that the time-step
has been increased to $\delta t=9.95\times 10^{-3}$,
yielding a CFL parameter, $C=2.0$, which exceeds unity. It can be seen that
the pulse grows in amplitude, and eventually starts to break up due to a short-wavelength
instability. 

\section{Crank-Nicholson Scheme}\label{cnwave}
The Crank-Nicholson implicit scheme for solving the diffusion equation (see Sect.~\ref{cn})
 can be adapted to solve the advection equation.
Thus, taking the average of the right-hand side of Eq.~(\ref{advection}) between the
beginning and  end of  the time-step, we obtain the  differencing
scheme written below:
\begin{equation}
u_i^{n+1} + \frac{C}{4}\,(u_{i+1}^{n+1}-u_{i-1}^{n+1}) = u_i^n - \frac{C}{4}\,(u_{i+1}^{n}-u_{i-1}^{n}).
\end{equation}
A von Neumann stability analysis of the above scheme yields the following
expression for the amplification factor:
\begin{equation}
A = \frac{1 - {\rm i} \,(C/2)\,\sin (k \,\delta x)}{1 + {\rm i} \,(C/2)\,\sin (k \,\delta x)}.
\end{equation}
Note that $|A| = 1$ for all values of $k$, irrespective of the value of $C$. This implies that
the Crank-Nicholson implicit scheme is {\em not} subject to the CFL constraint, $C<1$
(since there is no value of $k$ for which $|A|>1$).  Moreover,
there is no spurious decay in the Fourier harmonics
of the solution (since $|A|=1$). 
Hence, unlike the Lax scheme, we would not expect the  Crank-Nicholson scheme to introduce
strong numerical dispersion into the advection problem.

Listed below is a routine which solves the 1-d advection equation
via the Crank-Nicholson method.
{\small\begin{verbatim}
// Advect1D.cpp

// Function to evolve advection equation in 1-d:

//  du / dt + v du / dx = 0  for  xl <= x <= xh

//  u = 0  at x=xl and x=xh

// Array u assumed to be of extent N+2.

// Now, ith element of array corresponds to

//  x_i = xl + i * dx    i=0,N+1

// Here, dx = (xh - xl) / (N+1) is grid spacing.

// Function evolves u by single time-step.

// C = v dt / dx, where dt is time-step.

// Uses Crank-Nicholson scheme.

#include <blitz/array.h>

using namespace blitz;

void Tridiagonal (Array<double,1> a, Array<double,1> b, Array<double,1> c, 
                  Array<double,1> w, Array<double,1>& u);

void Advect1D (Array<double,1>& u, double C)
{  
  // Find N. Declare local arrays.
  int N = u.extent(0) - 2;
  Array<double,1> a(N+2), b(N+2), c(N+2), w(N+2);

  // Initialize tridiagonal matrix
  for (int i = 2; i <= N;   i++) a(i) = - 0.25 * C;
  for (int i = 1; i <= N; i++) b(i) = 1.;
  for (int i = 1; i <= N-1; i++) c(i) = + 0.25 * C;

  // Initialize right-hand side vector
  for (int i = 1; i <= N; i++)
    w(i) = u(i) - 0.25 * C * (u(i+1) - u(i-1));

  // Invert tridiagonal matrix equation
  Tridiagonal (a, b, c, w, u);

  // Calculate i=0 and i=N+1 values
  u(0) = 0.;
  u(N+1) = 0.;
}
\end{verbatim}}

Figure~\ref{decay} shows an example calculation which uses the above routine to
advect a Gaussian pulse. The initial condition is as specified in Eq.~(\ref{initcn}),
and the calculation is performed with $v=1$, $\delta t = 9.95\times 10^{-3}$, and $N=200$.
Note that $C=2.0$ for these parameters. It can be seen that the pulse propagates at (almost) the
correct speed, and maintains approximately the same shape. Clearly, the
performance of the Crank-Nicholson scheme is vastly superior to that of the Lax scheme.

\begin{figure}
\epsfysize=3in
\centerline{\epsffile{Chapter07/nodisperse.eps}}
\caption{\em Advection of a 1-d Gaussian pulse.
Numerical  calculation performed using
$v=1$, $\delta t = 9.95\times 10^{-3}$, and $N=200$.  The
solid curve shows the initial condition at $t=0$, the short-dashed curve  the numerical solution
at $t=1$, the long-dashed curve  the numerical solution
at $t=2$, and the dot-dashed  curve  the numerical solution at 
$t=3$. }\label{nodisperse}
\end{figure}

\section{Upwind Differencing}\label{supwind}
We might be forgiven for concluding that the Crank-Nicholson scheme represents an
efficient and accurate general purpose numerical method for solving the advection equation.
This is indeed the case, provided we restrict ourselves to fairly {\em smooth} wave-forms.
Unfortunately, the Crank-Nicholson scheme does a very poor
job at advecting wave-forms with sharp leading or trailing edges. This is illustrated in
Fig.~\ref{upwind1}, which shows a calculation in which the  Crank-Nicholson scheme 
is used to advect a square wave-pulse. It can be seen that spurious oscillations are generated
at both the leading and trailing edges of the wave-form. It turns out that all {\em central
difference}  schemes for solving the advection equation suffer from a
similar problem.
\begin{figure}
\epsfysize=3in
\centerline{\epsffile{Chapter07/upwind1.eps}}
\caption{\em Advection of a 1-d square wave-pulse.
Numerical  calculation performed using
$v=1$, $\delta t = 2.49\times 10^{-3}$, and $N=200$.  The
dotted curve shows the initial condition at $t=0.0$, whereas the solid curve shows the numerical solution
at $t=0.1$.}\label{upwind1}
\end{figure}

The only known way to suppress spurious oscillations 
at  the leading and trailing edges of a sharp wave-form is to adopt a so-called {\em upwind}
differencing scheme. In such a scheme, the spatial differences are skewed in the ``upwind''
direction: {\em i.e.}, the direction from which the advecting flow emanates. Thus, the
upwind version of the simple explicit differencing scheme (\ref{ftcs1})
is written
\begin{equation}
\frac{u_i^{n+1} - u_i^n}{\delta t} = - v\,\frac{u_{i}^n-u_{i-1}^n}{\delta x},
\end{equation}
or
\begin{equation}
u_i^{n+1} = u_i^n -C\,(u_{i}^n-u_{i-1}^n),
\end{equation}
Note that this scheme is only {\em first-order} in space, whereas every other scheme
we have discussed has been {\em second-order}. A von Neumann stability analysis
of the above scheme yields
\begin{equation}
A = 1 - C\,[1-\cos (k \,\delta x)]-{\rm i} \,C\,\sin (k \,\delta x).
\end{equation}
Note that 
\begin{equation}
|A|^2 =1-2\,C\,(1-C)\,[1-\cos (k \,\delta x)].
\end{equation}
It follows that $|A|<1$ for all $k$ provided that $C<1$. Thus, the upwind differencing
scheme is stable provided that the CFL condition is satisfied.
Fig.~\ref{upwind2}  shows a calculation in which the  above scheme 
is used to advect a square wave-pulse. There are now no spurious oscillations generated
at the sharp edges of the wave-form. On the other hand, the wave-form shows evidence
of dispersion. Indeed, the upwind differencing scheme suffers from the
same type of spurious dispersion problem as the Lax scheme. 
Unfortunately, there is no known  differencing scheme which is both
non-dispersive and
capable of dealing well with sharp wave-fronts.
In fact, sophisticated
codes which solve the  advection (or wave) equation generally employ an upwind scheme in regions close to
sharp wave-fronts, or shocks, and a more accurate non-dispersive scheme elsewhere. 

\begin{figure}
\epsfysize=3in
\centerline{\epsffile{Chapter07/upwind2.eps}}
\caption{\em Advection of a 1-d square wave-pulse.
Numerical  calculation performed using
$v=1$, $\delta t = 2.49\times 10^{-3}$, and $N=200$.  The
dotted curve shows the initial condition at $t=0.0$, whereas the solid curve shows the numerical solution
at $t=0.1$.}\label{upwind2}
\end{figure}

Incidentally, it is easily demonstrated that the downwind differencing scheme,
\begin{equation}
\frac{u_i^{n+1} - u_i^n}{\delta t} = - v\,\frac{u_{i+1}^n-u_{i}^n}{\delta x},
\end{equation}
is unconditionally unstable.

\section{1-D Wave Equation}\label{swave1d}
Consider a plane polarized electromagnetic wave propagating {\em in vacuo} along the $z$-axis.
Suppose that the electric and magnetic fields take the form ${\bf E} = [E_x(z,t),0,0]$, and
${\bf B} = [0,B_y(z,t),0]$. Now, according to Maxwell's equations,
\begin{eqnarray}
\frac{\partial E_x}{\partial t} + c\,\frac{\partial H_y}{\partial z} &=& 0,\label{wave1da}\\[0.5ex]
\frac{\partial H_y}{\partial t} + c\,\frac{\partial E_x}{\partial z} &=& 0,\label{wave1db}
\end{eqnarray}
where $H_y = c\,B_y$, and $c$ is the velocity of light. Note that the above equations
take the form of two coupled advection equations.
Let us find the numerical 
solution of
these equations in some region $0\leq z \leq L$ which is bounded by 
{\em perfectly
conducting walls} at $z=0$ and $z=L$. Now, both the {\em tangential} electric field and
the {\em normal} magnetic field must be zero at a perfect conductor. It follows that
\begin{equation}\label{wavebca}
E_x(0,t) = E_x(L,t) = 0.
\end{equation}
Moreover, Eq.~(\ref{wave1da}) yields
\begin{equation}\label{wavebcb}
\frac{\partial H_y(0,t)}{\partial z} = \frac{\partial H_y(L,t)}{\partial z} = 0.
\end{equation}
We expect Eqs.~(\ref{wave1da}) and~(\ref{wave1db})  to support wave-like solutions which
bounce backwards and forwards between the conducting walls. 

As before, we discretize in time on the uniform grid $t_n=t_0+n\,\delta t$, for $n=0,1,2,\cdots$.
Furthermore, in the  $z$-direction, we discretize on the uniform grid $z_i = i\,\delta z$, for
$i=0,I$, where $\delta z = L/I$. Adopting a Crank-Nicholson  temporal differencing scheme
similar to that discussed in Sect.~\ref{cnwave}, Eqs.~(\ref{wave1da})--(\ref{wave1db}) yield
\begin{eqnarray}\label{cnwave1da}
\frac{(E_x)_i^{n+1}-(E_x)_i^n}{\delta t} +\frac{c}{2}\left(\frac{\partial H_y}{\partial z}\right)_i^{n+1}
+ \frac{c}{2}\left(\frac{\partial H_y}{\partial z}\right)_i^n&=& 0,\\[0.5ex]
\frac{(H_y)_i^{n+1}-(H_y)_i^n}{\delta t} +\frac{c}{2}\left(\frac{\partial E_x}{\partial z}\right)_i^{n+1}
+ \frac{c}{2}\left(\frac{\partial E_x}{\partial z}\right)_i^n&=& 0,\label{cnwave1db}
\end{eqnarray}
where $(E_x)_i^n\equiv E_x(z_i,t_n)$, {\em etc.}

Adopting a Fourier approach, we write
\begin{eqnarray}
(E_x)_i^n &=& \sum_{j=0,I} \hat{E}_j^n\,\sin(i\,j\,\pi/I),\\[0.5ex]
(H_y)_i^n &=& \sum_{j=0,I} \hat{H}_j^n\,\cos(i\,j\,\pi/I),
\end{eqnarray}
which automatically satisfies the boundary conditions (\ref{wavebca}) and (\ref{wavebcb}). 
Equations (\ref{cnwave1da}) and (\ref{cnwave1db}) yield
\begin{eqnarray}
\hat{E}_i^{n+1}-\hat{E}_i^n - i\,D\,(\hat{H}_i^{n+1}+\hat{H}_i^n) &=& 0,\\[0.5ex]
\hat{H}_i^{n+1}-\hat{H}_i^n +  i\,D\,(\hat{E}_i^{n+1}+\hat{E}_i^n) &=& 0,
\end{eqnarray}
where $D=\pi\,c\,\delta t/(2\,L)$. It follows that
\begin{eqnarray}\label{stepa}
\hat{E}_i^{n+1} &=& +\frac{2\,i\,D}{1+i^2\,D^2}\,\hat{H}_i^n + \frac{1-i^2\,D^2}{1+i^2\,D^2}\,\hat{E}_i^n,\\[0.5ex]
\hat{H}_i^{n+1} &=& -\frac{2\,i\,D}{1+i^2\,D^2}\,\hat{E}_i^n + \frac{1-i^2\,D^2}{1+i^2\,D^2}\,\hat{H}_i^n,
\label{stepb}
\end{eqnarray}
for $i=0,I$. 

Let us determine the eigenvalues $\lambda$ of the above linear system,
assuming that $(\hat{E}_i^{n+1}, \hat{H}_i^{n+1}) = \lambda\,(\hat{E}_i^n, \hat{H}_i^n)$. After a little analysis,
we obtain
\begin{equation}
\lambda^2 - 2\,\frac{1-i^2\,D^2}{1+i^2\,D^2}\,\lambda + 1 = 0.
\end{equation}
For all values of $i$, the two roots of the above quadratic are complex, but have modulus unity: {\em i.e.},
$|\lambda|=1$. This implies that our differencing scheme is both
numerically stable (if $|\lambda|>1$ then the scheme would be unstable, since the associated
eigenfunction would be amplified at each time-step)
and non-dispersive (if $|\lambda|<1$ then all Fourier harmonics would eventually
decay away, and, hence, so would the solution). Note that the above calculation is
equivalent to von Neumann stability analysis.

The routine listed below solves the 1-d wave equation using the Crank-Nicholson scheme
discussed above. The routine first Fourier transforms $E_x$ and $H_y$, takes
a time-step using Eqs.~(\ref{stepa}) and (\ref{stepb}), and then reconstructs
$E_x$ and $H_y$ via an inverse Fourier transform.
{\small\begin{verbatim}
// Wave1D.cpp

// Function to evolve 1-d wave equation:

//  d E_x / dt + c d H_y / dz = 0

//  d H_y / dt + c d E_x / dz = 0

// in region 0 < z < L. 

// Boundary conditions: 

//  E_x(0,t) = E_x(L,t) = 0

//  d H_y(0,t) / dz =  d H_y(L,t) / dz = 0

// Arrays Ex, Hy assumed to be of extent I+1.

// Now, ith elements of arrays correspond to

//  z_i = i * L / J      i=0,I

// Also, D = pi c dt / (2 L), where dt is time-step.

// Uses Crank-Nicholson scheme.

#include <blitz/array.h>

using namespace blitz;

void fft_forward_cos (Array<double,1> f, Array<double,1>& F);
void fft_backward_cos (Array<double,1> F, Array<double,1>& f);
void fft_forward_sin (Array<double,1> f, Array<double,1>& F);
void fft_backward_sin (Array<double,1> F, Array<double,1>& f);

void Wave1D (Array<double,1>& Ex, Array<double,1>& Hy, double D)
{
  // Find I. Declare local arrays
  int I = Ex.extent(0) - 1;
  Array<double,1> EE(I+1), HH(I+1), EE0(I+1), HH0(I+1);

  // Fourier transform Ex and Hy
  fft_forward_sin (Ex, EE0);
  fft_forward_cos (Hy, HH0);

  // Evolve EE and HH
  for (int i = 0; i <= I; i++)
    {
      double x = double (i) * D;
      double fp = 1. + x*x;
      double fm = 1. - x*x;

      EE(i) = 2. * x * HH0(i) + fm * EE0(i);
      EE(i) /= fp;
      HH(i) = - 2. * x * EE0(i) + fm * HH0(i);
      HH(i) /= fp;
    }

  // Reconstruct Ex and Hy via inverse Fourier transform
  fft_backward_sin (EE, Ex);
  fft_backward_cos (HH, Hy);
}
\end{verbatim}}

Figure~\ref{wave1d} shows an example calculation which uses the above routine to
propagate a Gaussian wave-pulse. The initial condition is
\begin{equation}
E_x(z,0) =  H_y(z,0) = \exp[-100\,(z-0.5)^2],
\end{equation}
and the calculation is performed with $L=1$, $c=1$, $\delta t = 5\times 10^{-3}$, and $N=100$.
It can be seen that the pulse moves to the right, reflects off the conducting wall
at $z=1$, and then moves to the left. Note, that $E_x=+H_y$ when the wave is far
from the conducting walls and
moving to the right, whereas $E_x=-H_y$ when the wave is far
from the conducting walls and
moving to the left.

\begin{figure}
\epsfysize=5in
\centerline{\epsffile{Chapter07/wave1d.eps}}
\caption{\em Propagation of a 1-d Gaussian wave-pulse.
Numerical  calculation performed using
$c=1$, $\delta t = 5\times 10^{-3}$, and $N=100$.  The
solid curves show $E_x$, whereas the dashed curves show $H_y$. The
top-left, top-right, middle-left, middle-right, bottom-left, and
bottom-right panels show the solution at $t=0.0$, $0.2$, $0.4$, $0.6$, $0.8$,
and $1.0$, respectively. }\label{fwave1d}
\end{figure}

Figure~\ref{wave1da} shows a second example calculation performed with the
same parameters as the first. In this calculation, the pulse is allowed to reflect
off the conducting walls {\em ten} times before returning to its initial position. Note
that the pulse amplitude and shape remain constant to a very good approximation during
this process. The fact that the pulse returns almost exactly to its initial position
after ten time periods have elapsed ({\em i.e.}, at $t=10$) 
demonstrates that it is propagating at the
correct speed ({\em i.e.}, $c=1$).

\begin{figure}
\epsfysize=3in
\centerline{\epsffile{Chapter07/wave1da.eps}}
\caption{\em Propagation of a 1-d Gaussian wave-pulse.
Numerical  calculation performed using
$c=1$, $\delta t = 5\times 10^{-3}$, and $N=100$.
The solid curve shows $E_x$ at $t=0.$, the dashed curve shows $E_x$ at $t=10.$,
and the dotted curve (obscured by the dashed curve) shows $B_y$ at $t=10.$}\label{fwave1da}
\end{figure}

Figure~\ref{fwave1db} shows a third example calculation which uses the above listed routine to
propagate a square wave-pulse. Note that the routine does a very poor job, since
spurious oscillations are generated at the sharp leading and trailing edges of
the wave-form. As discussed in Sect.~\ref{supwind}, such oscillations can
only be suppressed by adopting an {\em upwind} differencing scheme, which in this
case means that the spatial differences must be skewed in the direction from
which the wave is propagating. Unfortunately, simple  explicit upwind schemes are
subject to the CFL constraint,
\begin{equation}
\delta t < \frac{\delta z}{c},
\end{equation}
and also tend to be highly dispersive.

\begin{figure}
\epsfysize=3in
\centerline{\epsffile{Chapter07/waveb.eps}}
\caption{\em Propagation of a 1-d square wave-pulse.
Numerical  calculation performed using
$c=1$, $\delta t = 5\times 10^{-3}$, and $N=100$.
The dotted curve shows $E_x$ at $t=0.0$, and  the solid curve shows $E_x$ at $t=0.1$.}\label{fwave1db}
\end{figure}

\section{2-D Resonant Cavity}
Figure~\ref{frescav} shows a 2-d resonant cavity consisting of a hollow, rectangular, perfectly
conducting channel of dimensions $L_x\times L_y$. Suppose that the walls
of the channel are aligned along the $x$- and $y$-axes.
We shall excite this cavity in a rather artificial manner
by imposing a $z$-directed alternating current pattern of frequency $f$,
which has the same spatial structure as the mode in which we are interested. Let us calculate the electric
and magnetic field patterns excited within the cavity by such a current pattern.

\begin{figure}
\epsfysize=3in
\centerline{\epsffile{Chapter07/rescav.eps}}
\caption{\em A 2-d resonant cavity.}\label{frescav}
\end{figure}

The electric and magnetic fields within the cavity can be written
${\bf E} = [0,0,$\\ $E_z(x,y,t)]$, and ${\bf B} = [B_x(x,y,t), B_y(x,y,t), 0]$, respectively.
It follows from Maxwell's equations that
\begin{eqnarray}
\frac{\partial H_x}{\partial t} + c\,\frac{\partial E_z}{\partial y} &=& 0,\label{rc1}\\[0.5ex]
\frac{\partial H_y}{\partial t} + c\,\frac{\partial E_z}{\partial x} &=& 0,\\[0.5ex]
\frac{\partial E_z}{\partial t} + c\,\frac{\partial H_y}{\partial x}
+ c\,\frac{\partial H_x}{\partial y}&=& J_z,\label{rc3}
\end{eqnarray}
where $c$ is the velocity of light, $H_x=c\,B_x$, $H_y=-c\,B_y$, and $J_z =-\mu_0\,c^2\,j_z$. Note that the
above system of equations takes the form of three coupled advection equations with a source term.
The boundary conditions are that the {\em tangential} electric field and the {\em normal}
magnetic field must be zero at the conducting walls. It follows that
\begin{equation}
E_z=0\label{rcbc1}
\end{equation}
at all the walls (which are located at $x=0,L_x$ and $y=0,L_y$), 
\begin{equation}
H_x = \frac{\partial H_y}{\partial x} = 0
\end{equation}
at $x=0,L_x$, and
\begin{equation}
H_y = \frac{\partial H_x}{\partial y} = 0\label{rcbc2}
\end{equation}
at $y=0,L_y$. Finally, the normalized current pattern associated
with the ($m$, $n$) mode takes the form
\begin{equation}
J_z(x,y,t) = J_0\,\sin(m\,\pi\,x/L_x)\,\sin(n\,\pi\,y/L_y)\,\sin(2\,\pi\,f\,t).
\end{equation}

As usual, we discretize in time on the uniform grid $t_n=t_0+n\,\delta t$, for $n=0,1,2,\cdots$.
Furthermore, in the  $x$-direction, we discretize on the uniform grid $x_i = i\,\delta x$, for
$i=0,I$, where $\delta x = L_x/I$.
 Finally, in the  $y$-direction, we discretize on the uniform grid $y_j = j\,\delta y$, for
$j=0,J$, where $\delta y = L_y/J$. Adopting a Crank-Nicholson  temporal differencing scheme
similar to that discussed in Sects.~\ref{cnwave} and \ref{swave1d}, Eqs.~(\ref{rc1})--(\ref{rc3}) yield
\begin{eqnarray}\label{xrc1}
\frac{(H_x)_{i,j}^{n+1} -(H_x)_{i,j}^{n}}{\delta t} 
+\frac{c}{2}\left(\frac{\partial E_z}{\partial y}\right)^{n+1}_{i,j}
+\frac{c}{2}\left(\frac{\partial E_z}{\partial y}\right)^{n}_{i,j} &=& 0,\\[0.5ex]
\frac{(H_y)_{i,j}^{n+1} -(H_y)_{i,j}^{n}}{\delta t} 
+\frac{c}{2}\left(\frac{\partial E_z}{\partial x}\right)^{n+1}_{i,j}
+\frac{c}{2}\left(\frac{\partial E_z}{\partial x}\right)^{n}_{i,j} &=& 0,\\[0.5ex]
\frac{(E_z)_{i,j}^{n+1} -(E_z)_{i,j}^{n}}{\delta t} 
+\frac{c}{2}\left(\frac{\partial H_y}{\partial x}\right)^{n+1}_{i,j}
+\frac{c}{2}\left(\frac{\partial H_y}{\partial x}\right)^{n}_{i,j}&&\nonumber\\[0.5ex]
+\frac{c}{2}\left(\frac{\partial H_x}{\partial y}\right)^{n+1}_{i,j}
+\frac{c}{2}\left(\frac{\partial H_x}{\partial y}\right)^{n}_{i,j} &=& (J_z)_{i,j}^n,\label{xrc3}
\end{eqnarray}
where $(H_x)_{i,j}^n\equiv H_x(x_i, y_j, t_n)$, {\em etc.}

Adopting a Fourier approach, we write
\begin{eqnarray}
(E_z)_{i,j}^n &=& \sum_{i'=0,I}^{j'=0,J}\hat{E}_{i',j'}^n\,\sin(i\,i'\,\pi/I)\,\sin(j\,j'\,\pi/J),\\[0.5ex]
(H_x)_{i,j}^n &=& \sum_{i'=0,I}^{j'=0,J} \hat{X}_{i',j'}^n\,\sin(i\,i'\,\pi/I)\,\cos(j\,j'\,\pi/J),\\[0.5ex]
(H_y)_{i,j}^n &=& \sum_{i'=0,I}^{j'=0,J} \hat{Y}_{i',j'}^n\,\cos(i\,i'\,\pi/I)\,\sin(j\,j'\,\pi/J),\\[0.5ex]
(J_z)_{i,j}^n &=& \sum_{i'=0,I}^{j'=0,J} \hat{J}_{i',j'}^n\,\sin(i\,i'\,\pi/I)\,\sin(j\,j'\,\pi/J)
\end{eqnarray}
which automatically satisfies the boundary conditions (\ref{rcbc1})--(\ref{rcbc2}). 
Equations~(\ref{xrc1})--(\ref{xrc3}) yield
\begin{eqnarray}
\hat{X}_{i,j}^{n+1}-\hat{X}_{i,j}^n + j\,D_y\,(\hat{E}_{i,j}^{n+1}+\hat{E}_{i,j}^{n})&=&0,\\[0.5ex]
\hat{Y}_{i,j}^{n+1}-\hat{Y}_{i,j}^n + i\,D_x\,(\hat{E}_{i,j}^{n+1}+\hat{E}_{i,j}^{n})&=&0,\\[0.5ex]
\hat{E}_{i,j}^{n+1}-\hat{E}_{i,j}^n - i\,D_x\,(\hat{Y}_{i,j}^{n+1}+\hat{Y}_{i,j}^{n})
- j\,D_j\,(\hat{X}_{i,j}^{n+1}+\hat{X}_{i,j}^{n})&=&\delta t\,\hat{J}^{n}_{i,j},
\end{eqnarray}
for $i=0,I$ and $j=0,J$, where $D_x=\pi\,c\,\delta t/(2\,L_x)$ and  $D_y=\pi\,c\,\delta t/(2\,L_y)$.
It follows that
\begin{eqnarray}
\hat{E}^{n+1}_{i,j} &=&  \frac{(1-i^2\,D_x^2-j^2\,D_y^2)\,\hat{E}_{i,j}^n + 2\,j\,D_y\,
\hat{X}_{i,j}^n+ 2\,i\,D_x\,\hat{Y}_{i,j}^n
+\delta t \,\hat{J}^n_{i,j}}
{1 + i^2\,D_x^2+j^2\,D_y^2}\nonumber\\[0.5ex]
&&\label{xstepa}\\[0.5ex]
\hat{X}_{i,j}^{n+1}& =& \hat{X}_{i,j}^n - j\,D_y\,(\hat{E}_{i,j}^{n+1}+\hat{E}_{i,j}^{n}),\\[0.5ex]
\hat{Y}_{i,j}^{n+1}&=& \hat{Y}_{i,j}^n - i\,D_x\,(\hat{E}_{i,j}^{n+1}+\hat{E}_{i,j}^{n}).\label{xstepb}
\end{eqnarray}

The routine listed below solves the 2-d wave equation in a resonant
cavity using the Crank-Nicholson scheme
discussed above. The routine first Fourier transforms $H_x$, $H_y$, $E_z$, and
$J_z$ in both the $x$- and $y$-directions, takes
a time-step using Eqs.~(\ref{xstepa})--(\ref{xstepb}), and then reconstructs
$H_x$, $H_y$, and $E_z$  via an double inverse Fourier transform.
{\small\begin{verbatim}
// Wave2D.cpp

// Function to evolve 2-d wave equation:

//  d H_x / dt + c d E_z / dy = 0

//  d H_y / dt + c d E_z / dx = 0

//  d E_z / dt + c d H_y / dx + c d H_x / dy = J_z

// in region 0 < x < L_x and 0 < y < L_y

// Boundary conditions:

//  E_z(0, y) = E_z(L_x, y) = E_z(x, 0) = E_z(x, L_y) = 0

//  H_x(0, y) = H_x(L_x, y) = d H_y(0, y) / dx =  d H_y(L_x, y) / dx = 0

//  H_y(x, 0) = H_y(x, L_y) = d H_x(x, 0) / dy =  d H_x(x, L_y) / dy = 0

// Matrices Hx, Hy, Ez, Jz assumed to be of extent I+1, J+1.
// Now, (i,j)th elements of matrices correspond to

//  x_i = i * dx    i=0,I

//  y_j = j * dy    j=0,J

// Here, dx = L_x / I is grid spacing in x-direction,

// and dy = L_y / J is grid spacing in x-direction.

// Now, Dx = pi c dt / (2 L_x) and Dy = pi c dt / (2 L_y),

// where dt is time-step.

// Uses Crank-Nicholson scheme.

#include <blitz/array.h>

using namespace blitz;

void fft_forward_cos (Array<double,1> f, Array<double,1>& F);
void fft_backward_cos (Array<double,1> F, Array<double,1>& f);
void fft_forward_sin (Array<double,1> f, Array<double,1>& F);
void fft_backward_sin (Array<double,1> F, Array<double,1>& f);

void Wave2D (Array<double,2>& Hx, Array<double,2>& Hy, Array<double,2>& Ez,
             Array<double,2> Jz, double Dx, double Dy, double dt)
{
  // Find I and J. Declare local arrays
  int I = Hx.extent(0) - 1;
  int J = Hx.extent(1) - 1;
  Array<double,2> X(I+1, J+1), XX(I+1, J+1), XXX(I+1, J+1);
  Array<double,2> Y(I+1, J+1), YY(I+1, J+1), YYY(I+1, J+1); 
  Array<double,2> E(I+1, J+1), EE(I+1, J+1), EEE(I+1, J+1);
  Array<double,2> K(I+1, J+1), KK(I+1, J+1);

  // Fourier transform solution in x-direction
  for (int j = 0; j <= J; j++)
    {
      Array<double,1> In(I+1), Out(I+1);

      // Fourier transform Hx
      for (int i = 0; i <= I; i++) In(i) = Hx(i, j);
      fft_forward_sin (In, Out);
      for (int i = 0; i <= I; i++) X(i, j) = Out(i);

      // Fourier transform Hy
      for (int i = 0; i <= I; i++) In(i) = Hy(i, j);
      fft_forward_cos (In, Out);
      for (int i = 0; i <= I; i++) Y(i, j) = Out(i);

      // Fourier transform Ez
      for (int i = 0; i <= I; i++) In(i) = Ez(i, j);
      fft_forward_sin (In, Out);
      for (int i = 0; i <= I; i++) E(i, j) = Out(i);

      // Fourier transform Jz
      for (int i = 0; i <= I; i++) In(i) = dt * Jz(i, j);
      fft_forward_sin (In, Out);
      for (int i = 0; i <= I; i++) K(i, j) = Out(i);
    }

  // Fourier transform solution in y-direction
  for (int i = 0; i <= I; i++)
    {
      Array<double,1> In(J+1), Out(J+1);

      // Fourier transform Hx
      for (int j = 0; j <= J; j++) In(j) = X(i, j);
      fft_forward_cos (In, Out);
      for (int j = 0; j <= J; j++) XX(i, j) = Out(j);

      // Fourier transform Hy
      for (int j = 0; j <= J; j++) In(j) = Y(i, j);
      fft_forward_sin (In, Out);
      for (int j = 0; j <= J; j++) YY(i, j) = Out(j);

      // Fourier transform Ez
      for (int j = 0; j <= J; j++) In(j) = E(i, j);
      fft_forward_sin (In, Out);
      for (int j = 0; j <= J; j++) EE(i, j) = Out(j);

      // Fourier transform Jz
      for (int j = 0; j <= J; j++) In(j) = K(i, j);
      fft_forward_sin (In, Out);
      for (int j = 0; j <= J; j++) KK(i, j) = Out(j);
    }
  
  // Evolve XX, YY, and EE
  for (int i = 0; i <= I; i++)
    for (int j = 0; j <= J; j++)
      {
        double x = double (i) * Dx;
        double y = double (j) * Dy;
        double fp = 1. + x*x + y*y;
        double fm = 1. - x*x - y*y;

        EEE(i, j) = fm * EE(i, j) + 2. * y * XX(i, j) +
          2. * x * YY(i,j) + KK(i, j);
        EEE(i, j) /= fp;
        XXX(i, j) = XX(i, j) - y * (EEE(i, j) + EE(i, j));
        YYY(i, j) = YY(i, j) - x * (EEE(i, j) + EE(i, j));  
      }

  // Reconstruct solution via inverse Fourier transform in y-direction
  for (int i = 0; i <= I; i++)
    {
      Array<double,1> In(J+1), Out(J+1);

      // Reconstruct Hx
      for (int j = 0; j <= J; j++) In(j) = XXX(i, j);
      fft_backward_cos (In, Out);
      for (int j = 0; j <= J; j++) X(i, j) = Out(j);

      // Reconstruct Hy
      for (int j = 0; j <= J; j++) In(j) = YYY(i, j);
      fft_backward_sin (In, Out);
      for (int j = 0; j <= J; j++) Y(i, j) = Out(j);

      // Reconstruct Ez
      for (int j = 0; j <= J; j++) In(j) = EEE(i, j);
      fft_backward_sin (In, Out);
      for (int j = 0; j <= J; j++) E(i, j) = Out(j);
    } 

  // Reconstruct solution via inverse Fourier transform in x-direction
  for (int j = 0; j <= J; j++)
    {
      Array<double,1> In(I+1), Out(I+1);

      // Reconstruct Hx
      for (int i = 0; i <= I; i++) In(i) = X(i, j);
      fft_backward_sin (In, Out);
      for (int i = 0; i <= I; i++) Hx(i, j) = Out(i);

      // Reconstruct Hy
      for (int i = 0; i <= I; i++) In(i) = Y(i, j);
      fft_backward_cos (In, Out);
      for (int i = 0; i <= I; i++) Hy(i, j) = Out(i);

      // Reconstruct Ez
      for (int i = 0; i <= I; i++) In(i) = E(i, j);
      fft_backward_sin (In, Out);
      for (int i = 0; i <= I; i++) Ez(i, j) = Out(i);
    }
}
\end{verbatim}}

The numerical calculations discussed below were performed
using the above routine. The electromagnetic
fields $H_x$, $H_y$, and $E_z$ were all initialized to zero everywhere  at $t=0$.
 Figure~\ref{fcavity1} shows the maximum
amplitude of $E_z$ versus the frequency, $f$, for an $m=1/n=1$ driving
current distribution. It can
be seen that there is a clear resonance at $f\simeq 0.7$. 

\begin{figure}
\epsfysize=3in
\centerline{\epsffile{Chapter07/cavity1.eps}}
\caption{\em Electromagnetic waves in a 2-d resonant cavity.
The maximum amplitude of $E_z(x=0.5,y=0.5)$ between $t=0$ and $t=20$ versus the driving frequency, $f$.
Numerical calculation performed using  $m=1$, $n=1$, $L_x=1$, $L_y=1$, $c=1$,
$I=J=32$, and $\delta t = 10^{-2}$.}\label{fcavity1}
\end{figure}

Figures~\ref{fcavity2} and \ref{fcavity3} illustrate the typical time variation
of $E_z$, $H_x$, and $H_y$ for a non-resonant and a resonant case, respectively.
For the non-resonant case, the traces take the form of interference patterns
between the directly driven response, which oscillates at the driving frequency $f$,
and the transient response, which oscillates at the natural frequency $f_0$ of the cavity.
Note that the transients never decay, since there is no dissipation in the present problem.
Incidentally, it is easily demonstrated that
\begin{equation}
f_0 = \frac{c}{2}\sqrt{\frac{1}{(n\,L_x)^2}+\frac{1}{(m\,L_y)^2}}.
\end{equation}
Hence, it follows that $f_0=1/\sqrt{2} = 0.7071$ for $n=m=L_x=L_y=c=1$, which corresponds very well
to the resonant frequency found in Fig.~\ref{fcavity1}.
For the resonant case, the traces take the form of waves of ever increasing amplitude 
which  oscillate at the
natural frequency $f_0$.

\begin{figure}
\epsfysize=3in
\centerline{\epsffile{Chapter07/cavity2.eps}}
\caption{\em Electromagnetic waves in a 2-d resonant cavity.
Time traces of $E_z(x=0.5,y=0.5)$ (solid curve), $H_x(x=0.5,y=0.0)$ (dashed curve), and
$H_y(x=0.0,y=0.5)$ (dotted curve---obscured by dashed curve).
Numerical calculation performed using $m=1$, $n=1$, $L_x=1$, $L_y=1$, $c=1$,
$I=J=32$, $\delta t = 10^{-2}$, and $f=0.6$.}\label{fcavity2}
\end{figure}

\begin{figure}
\epsfysize=3in
\centerline{\epsffile{Chapter07/cavity3.eps}}
\caption{\em Electromagnetic waves in a 2-d resonant cavity.
Time traces of $E_z(x=0.5,y=0.5)$ (solid curve), $H_x(x=0.5,y=0.0)$ (dashed curve), and
$H_y(x=0.0,y=0.5)$ (dotted curve---obscured by dashed curve).
Numerical calculation performed using $m=1$, $n=1$, $L_x=1$, $L_y=1$, $c=1$,
$I=J=32$, $\delta t = 10^{-2}$, and $f=0.7071$.}\label{fcavity3}
\end{figure}

Finally, Figs.~\ref{fcavity3} and \ref{fcavity4} illustrate the spatial
variation of the electromagnetic fields driven within the cavity when
$m=1$ and $n=1$. 

\begin{figure}
\epsfysize=3in
\centerline{\epsffile{Chapter07/cavity4.eps}}
\caption{\em Electromagnetic waves in a 2-d resonant cavity.
Spatial variation of $E_z$ (solid curve) and
$H_y$ (dashed curve) in the $x$-direction at $t=20$ and $y=0.5$.
Numerical calculation performed using $m=1$, $n=1$, $L_x=1$, $L_y=1$, $c=1$,
$I=J=32$, $\delta t = 10^{-2}$, and $f=0.7071$.}\label{fcavity4}
\end{figure}

\begin{figure}
\epsfysize=3in
\centerline{\epsffile{Chapter07/cavity5.eps}}
\caption{\em Electromagnetic waves in a 2-d resonant cavity.
Spatial variation of $E_z$ (solid curve) and
$H_x$ (dashed curve) in the $y$-direction at $t=20$ and $x=0.5$.
Numerical calculation performed using $m=1$, $n=1$, $L_x=1$, $L_y=1$, $c=1$,
$I=J=32$, $\delta t = 10^{-2}$, and $f=0.7071$.}\label{fcavity5}
\end{figure}
