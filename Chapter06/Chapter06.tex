\chapter{Diffusion Equation}
\section{Introduction}
The diffusion equation
\begin{equation}
\frac{\partial T({\bf r},t)}{\partial t} = D\,\nabla^2 T({\bf r},t),
\end{equation}
where $D>0$ is the (uniform) coefficient of diffusion, describes many interesting physical
phenomena. For instance, in heat conduction we can write
\begin{equation}
{\bf q} = - \kappa\,\nabla T,
\end{equation}
where ${\bf q}$ is the heat flux, $T$ the temperature, and $\kappa$ the coefficient
of thermal conductivity. The above equation merely states that heat flows down a temperature
gradient. In the absence of sinks or sources of heat, energy conservation
requires that
\begin{equation}
-\frac{\partial Q}{\partial t} =  \int {\bf q}\cdot d{\bf S},
\end{equation}
where $Q$ is the thermal energy contained in  some volume $V$ bounded by a closed surface $S$.
The above equation states that the rate of decrease of the thermal energy content of
 volume $V$
 equals  the instantaneous heat flux flowing across its boundary.
However, $Q=\int c\,T\,dV$, where $c$ is the heat capacity per unit volume. Making
use of the previous
equations, as well as the divergence theorem, we obtain the following diffusion
equation for the temperature:
\begin{equation}
\frac{\partial T}{\partial t} = D\,\nabla^2 T,
\end{equation}
where $D=\kappa/c$. In a typical heat conduction problem, we are given the
temperature  $T({\bf r}, t_0)$ at some initial time $t_0$, and then asked
to evaluate $T({\bf r},t)$ at all subsequent times. Such a problem is
known as an {\em initial value problem}. The spatial boundary conditions can be either
of type Dirichlet ({\em i.e.}, $T$ specified on the boundary), type
Neumann ({\em i.e.}, $\nabla T$ specified on the boundary), or some combination.

\section{1-D Problem with Mixed Boundary Conditions}
Consider the solution of the diffusion equation in one dimension.
Suppose that
\begin{equation}\label{diff}
\frac{\partial T(x,t)}{\partial t} = D\,\frac{\partial^2 T(x,t)}{\partial x^2},
\end{equation}
for $x_l\leq x\leq x_h$, subject to the mixed spatial boundary conditions
\begin{equation}
\alpha_l(t)\,T(x,t) + \beta_l(t)\,\frac{\partial T(x,t)}{\partial x} = \gamma_l(t),
\end{equation}
at $x=x_l$, and
\begin{equation}
\alpha_h(t) \,T(x,t)+ \beta_h(t)\,\frac{\partial  T(x,t)}{\partial x} = \gamma_h(t),
\end{equation}
at $x=x_h$. Here, $\alpha_l$, $\beta_l$, {\em etc.}, are known functions of time.
Of course, $T(x,t_0)$ must be specified at some initial time $t_0$.

Equation~(\ref{diff}) needs to be discretized in both time and space.
In time, we  discretize
on the equally spaced grid
\begin{equation}
t_n = t_0 + n\,\delta t,
\end{equation}
where $\delta t$ is the {\em time-step}. Adopting a
simple first-order differencing scheme, Eq.~(\ref{diff}) becomes
\begin{equation}\label{diff1}
\frac{T(x,t_{n+1}) - T(x,t_n)}{\delta t} 
= D \,\frac{\partial^2 T(x,t_n)}{\partial x^2}+ O(\delta t).
\end{equation}
 In space,
we discretize on the usual equally spaced grid-points specified in Eq.~(\ref{e57}),
and approximate $d^2/dx^2$ via the second-order, central difference scheme
introduced in Eq.~(\ref{e58}). The spatial boundary conditions are discretized in a similar
manner to Eqs.~(\ref{e529}) and (\ref{e530}). Thus, Eq.~(\ref{diff1}) yields
\begin{equation}
\frac{T_i^{n+1}-T_i^n}{\delta t} = D\,\frac{T_{i-1}^n-2\,T_i^n+T_{i+1}^n}{(\delta x)^2},
\end{equation}
or 
\begin{equation}\label{diff2}
T_i^{n+1} = T_i^n + C\,\left(T_{i-1}^n-2\,T_i^n+T_{i+1}^n\right)
\end{equation}
for $i=1,N$, where $T_i^n \equiv T(x_i,t_n)$, and $C= D\,\delta t/(\delta x)^2$. 
The discretized boundary conditions take the form
\begin{eqnarray}
T_0^n &=& \frac{\gamma_l^n\,\delta x-\beta_l^n\,T_1^n}{\alpha_l^n\,\delta x -\beta_l^n},\\[0.5ex]
T_{N+1}^n &=& \frac{\gamma_h^n\,\delta x + \beta_h^n\,T_N^n}{\alpha_h^n\,\delta x +\beta_h^n},\label{diff3}
\end{eqnarray}
where $\gamma_l^n\equiv \gamma_l(t_n)$, {\em etc.} The discretization scheme outlined above
is termed first-order in time and second-order in space.

Equations (\ref{diff2})--(\ref{diff3}) constitute a fairly straightforward iterative scheme which
can be used to evolve the $T(x,t)$ in time.

\section{An Example 1-D Diffusion Equation Solver}
Listed below is an example 1-d diffusion equation solving routine which
makes use of the {\tt Blitz++} library.
{\small\begin{verbatim}
// Diffusion1D.cpp

// Function to evolve diffusion equation in 1-d:

//  dT / dt = D d^2 T / dx^2  for  xl <= x <= xh

//  alpha_l T + beta_l dT/dx = gamma_l  at x=xl

//  alpha_h T + beta_h dT/dx = gamma_h  at x=xh

// Array T assumed to be of extent N+2.

// Now, ith element of array corresponds to

//  x_i = xl + i * dx    i=0,N+1

// Here, dx = (xh - xl) / (N+1) is grid spacing.

// Function evolves T by single time-step.

// C = D dt / dx^2, where dt is time-step.

// Uses explicit scheme.

#include <blitz/array.h>

using namespace blitz;

void Diffusion1D (Array<double,1>& T, 
                  double alpha_l, double beta_l, double gamma_l,
                  double alpha_h, double beta_h, double gamma_h,
                  double dx, double C)
{
  // Set N. Declare local array.
  int N = T.extent(0) - 2;
  Array<double,1> T0(N+2);	

  // Evolve T
  T0 = T;	
  for (int i = 1; i <= N; i++)
    T(i) += C * (T0(i-1) - 2. * T0(i) + T0(i+1));

  // Set boundary conditions
  T(0) = (gamma_l * dx - beta_l * T(1)) / (alpha_l * dx - beta_l);
  T(N+1) = (gamma_h * dx + beta_h * T(N)) / (alpha_h * dx + beta_h);
}
\end{verbatim}}

\section{An Example 1-D Solution of the Diffusion Equation}\label{simple}
Let us now solve the diffusion equation in 1-d using the finite difference
technique discussed above. We seek the solution of Eq.~(\ref{diff})
in the region $-x_0 \leq x \leq x_0$, subject to the initial
condition
\begin{equation}
T(x,t_0) = \exp\!\left(\frac{-x^2}{4\,D\,t_0}\right),
\end{equation}
where $t_0>0$. The spatial boundary conditions are
\begin{equation}
 T(\pm x_0,t) = \sqrt{\frac{t_0}{t}}\, \exp\!\left(\frac{-x_0^2}{4\,D\,t}\right).
\end{equation}
 Of course, we can solve this problem
analytically to give
\begin{equation}
T(x,t) =\sqrt{\frac{t_0}{t}}\, \exp\!\left(\frac{-x^2}{4\,D\,t}\right).
\end{equation}
Note that the above equation describes a Gaussian 
 pulse which gradually decreases in height
and broadens in width in such a manner that its area is conserved. The width of the pulse
varies approximately as
\begin{equation}
{\mit\Delta}x \sim \sqrt{D\,t}.
\end{equation}
Moreover, the pulse approaches a $\delta$-function as $t\rightarrow 0$. 

\begin{figure}
\epsfysize=3in
\centerline{\epsffile{Chapter06/diff1.eps}}
\caption{\em Diffusive evolution of a 1-d Gaussian pulse.
Numerical  calculation performed using
$D=1$, $x_0=5$, $\delta t = 4\times 10^{-3}$, and $N=100$. The pulse is evolved from $t=0.1$ to $t=1$. The
solid curve shows the initial condition at $t=0.1$, the dashed curve  the numerical solution
at $t=1$, and the dotted curve (obscured by the dashed curve)  the analytic solution at $t=1$. }\label{fdiff1}
\end{figure}

Figure~\ref{fdiff1} shows a comparison between the analytic and numerical solutions for
a calculation performed using $D=1$, $x_0=5$,  $t_0=0.1$, $\delta t = 4\times 10^{-3}$, and $N=100$.
It can be seen that the  analytic and numerical solutions are in excellent agreement.

\begin{figure}
\epsfysize=4in
\centerline{\epsffile{Chapter06/diff2.eps}}
\caption{\em  Diffusive evolution of a 1-d Gaussian pulse.
Numerical  calculation performed using
$D=1$, $x_0=5$,  $\delta t = 4\times 10^{-3}$, and $N=125$. The simulation is started
at $t=0.1$. The top-left, top-right, bottom-left, and bottom-right panels
show the solution at $t=0.45$, $t=0.46$, $t=0.47$, and $t=0.48$, respectively.}\label{fdiff2}
\end{figure}

It is reasonable to expect that as $N$ increases at fixed $\delta t$
({\em i.e.}, the spatial resolution increases at fixed temporal resolution) the numerical
solution should become more and more accurate. This is indeed the case---at least, until $N$ 
exceeds a critical value. Beyond this value, there is a
catastrophic breakdown in the numerical solution. This breakdown is illustrated in Fig.~\ref{fdiff2}. 
It can be seen that the
solution develops rapidly growing short-wavelength oscillations. Indeed, the solution
eventually becomes effectively infinite. Let us investigate this unusual
and rather disturbing phenomenon.

\section{von Neumann Stability Analysis}
Clearly, our simple finite difference algorithm 
for solving the 1-d diffusion equation is subject to a numerical instability under certain circumstances.
Let us try to establish when this instability occurs. Consider the time evolution
of a single Fourier mode of wave-number $k$: 
\begin{equation}
T(x,t) = \hat{T}(t)\,{\rm e}^{\,{\rm i}\,k\,x}.
\end{equation}
Substitution of the above expression into our finite difference scheme (\ref{diff2}) yields
\begin{equation}
\hat{T}^{n+1}\,{\rm e}^{\,{\rm i}\,k\,x_n} = \hat{T}^n\,{\rm e}^{\,{\rm i}\,k\,x_n}\,
\left[1 + C\,({\rm e}^{-{\rm i}\,k\,\delta x}-2+ {\rm e}^{+{\rm i}\,k\,\delta x})\right],
\end{equation}
or
\begin{equation}
\hat{T}^{n+1} = A\,\hat{T}^n,
\end{equation}
where 
\begin{equation}\label{stable}
A = 1 - 2\,C\,(1-\cos k\,\delta x) = 1 - 4\,C\,\sin^2(k\,\delta x/2).
\end{equation}
Thus, the amplitude of the Fourier mode is amplified by a factor $A$ at each time-step.
In order for the differencing scheme to be stable, the
modulus of this amplification
factor must be {\em less than unity}  for {\em all} possible values of $k$. Now, the largest
possible value of $\sin^2(k\,\delta x/2)$ is unity: hence, the wave-length corresponding to
this value is that of the most unstable Fourier mode. In fact, the most unstable mode
possesses a wave-length which is {\em half} the grid-spacing: {\em i.e.},
$\lambda = \delta x/2$. It follows from Eq.~(\ref{stable}) that this mode
is stable provided 
\begin{equation}
C < \frac{1}{2}.
\end{equation}
Finally, from the definition of $C$, our stability condition can be written
\begin{equation}\label{limit}
\delta t < \frac{(\delta x)^2}{2\,D}.
\end{equation}
Note that $C=0.408$ for the stable calculation shown in Fig.~\ref{fdiff1}, whereas $C=0.635$
for the unstable calculation shown in Fig.~\ref{fdiff2}.
Incidentally, the type of stability analysis outlined above is called {\em von Neumann stability analysis}.
Note that the neglect of the spatial boundary conditions in the above calculation is justified
because the unstable modes vary on very small length-scales which are typically of order the grid spacing.

According to Eq.~(\ref{limit}), our finite difference scheme for solving the 1-d diffusion
equation is only stable provided that the time-step remains below some critical value.
Note that this critical value scales like the {\em square} of the grid-spacing. This is
a very unfavorable scaling, since it implies that a doubling of the spatial resolution
requires a simultaneous reduction
in the time-step  by a factor of four in order to maintain numerical stability. 
Certainly, the above constraint limits us to absurdly small time-steps in high resolution
calculations.
Is there any way of overcoming this constraint?

\section{The Crank-Nicholson Scheme}\label{cn}
Let us revisit our temporal differencing scheme:
\begin{equation}
\frac{T(x,t_{n+1}) - T(x,t_n)}{\delta t} 
= D \,\frac{\partial^2 T(x,t_n)}{\partial x^2}+ O(\delta t).
\end{equation}
Note that the right-hand side is evaluated entirely at the {\em start}
of the time-step: {\em i.e.}, at $t_n$. This type of
temporal differencing scheme is termed an {\em explicit} scheme. Now, explicit schemes are
very straightforward to implement, but are also notoriously prone to numerical instabilities.
Fortunately, we can often overcome these  instabilities by making our differencing
scheme {\em implicit} in nature. An implicit scheme is one in which the right-hand
side is evaluated partly (or wholly) at the {\em end} of the time-step: {\em i.e.}, at $t_{n+1}$.
Unfortunately, implicit schemes are generally a great deal more complicated to implement
than explicit schemes.

The well-known Crank-Nicholson implicit method for solving the diffusion equation involves
taking the 
average of the right-hand side between the beginning and  end of  the time-step. In other words,
\begin{equation}
\frac{T(x,t_{n+1}) - T(x,t_n)}{\delta t} 
= \frac{D}{2} \,\frac{\partial^2 T(x,t_n)}{\partial x^2}+
\frac{D}{2} \,\frac{\partial^2 T(x,t_{n+1})}{\partial x^2}+ O(\delta t)^2.
\end{equation}
As indicated by the error term, this method is actually {\em second-order} in time.

Adopting our usual spatial differencing scheme, the above expression yields
\begin{equation}\label{crank}
T_i^{n+1} - \frac{C}{2}\,\left(T_{i-1}^{n+1}-2\,T_i^{n+1}+T_{i+1}^{n+1}\right)=
 T_i^n + \frac{C}{2}\,\left(T_{i-1}^n-2\,T_i^n+T_{i+1}^n\right).
\end{equation}
When we perform a von Neumann stability analysis of the above scheme, we obtain the
following expression for the amplification factor:
\begin{equation}
A = \frac{1 - 2\,C\,\sin^2(k\,\delta x/2)}{1+2\,C\,\sin^2(k\,\delta x/2)}.
\end{equation}
Note that $|A|<1$ for all values of $k$. It follows that the Crank-Nicholson
scheme is {\em unconditionally stable}. Unfortunately, Eq.~(\ref{crank}) constitutes
a tridiagonal matrix equation linking the $T_i^{n+1}$ and the $T_i^n$. Thus, the
price we pay for the high accuracy and unconditional stability of the Crank-Nicholson
scheme is  having to invert a tridiagonal matrix equation at each time-step. Usually,
this price is well worth paying.

\section{An Improved 1-D Diffusion Equation Solver}
Listed below is an improved 1-d diffusion equation solver which uses the Crank-Nicholson
scheme, as well as the previous listed tridiagonal matrix solver and the {\tt Blitz++}
library. Note the great structural similarity between this solver and the previously listed
1-d Poisson solver (see Sect.~\ref{poisson1d}).
{\small\begin{verbatim}
// CrankNicholson1D.cpp

// Function to evolve diffusion equation in 1-d:

//  dT / dt = D d^2 T / dx^2  for  xl <= x <= xh

//  alpha_l T + beta_l dT/dx = gamma_l  at x=xl

//  alpha_h T + beta_h dT/dx = gamma_h  at x=xh

// Array T assumed to be of extent N+2.

// Now, ith element of array corresponds to

//  x_i = xl + i * dx    i=0,N+1

// Here, dx = (xh - xl) / (N+1) is grid spacing.

// Function evolves T by single time-step.

// C = D dt / dx^2, where dt is time-step.

// Uses Crank-Nicholson implicit scheme.

#include <blitz/array.h>

using namespace blitz;

void Tridiagonal (Array<double,1> a, Array<double,1> b, Array<double,1> c, 
                  Array<double,1> w, Array<double,1>& u);

void CrankNicholson1D (Array<double,1>& T, 
                  double alpha_l, double beta_l, double gamma_l,
                  double alpha_h, double beta_h, double gamma_h,
                  double dx, double C)
{
  // Find N. Declare local arrays.
  int N = T.extent(0) - 2;
  Array<double,1> a(N+2), b(N+2), c(N+2), w(N+2);

  // Initialize tridiagonal matrix
  for (int i = 2; i <= N; i++) a(i) = - 0.5 * C;
  for (int i = 1; i <= N; i++) b(i) = 1. + C;
  b(1) += 0.5 * C * beta_l / (alpha_l * dx - beta_l);
  b(N) -= 0.5 * C * beta_h / (alpha_h * dx + beta_h);
  for (int i = 1; i <= N-1; i++) c(i) = - 0.5 * C;

  // Initialize right-hand side vector
  for (int i = 1; i <= N; i++)
    w(i) = T(i) + 0.5 * C * (T(i-1) - 2. * T(i) + T(i+1));
  w(1) += 0.5 * C * gamma_l * dx / (alpha_l * dx - beta_l);
  w(N) += 0.5 * C * gamma_h * dx / (alpha_h * dx + beta_h);

  // Invert tridiagonal matrix equation
  Tridiagonal (a, b, c, w, T);

  // Calculate i=0 and i=N+1 values
  T(0) = (gamma_l * dx - beta_l * T(1)) /
    (alpha_l * dx - beta_l);
  T(N+1) = (gamma_h * dx + beta_h * T(N)) /
    (alpha_h * dx + beta_h);
}
\end{verbatim}}

\section{An Improved 1-D Solution of the Diffusion Equation}
Let us now solve the simple diffusion problem introduced in Sect.~\ref{simple} with the above listed
Crank-Nicholson routine. Figure~\ref{fdiff3} 
shows a comparison between the analytic and numerical solutions for
a calculation performed using $D=1$, $x_0=5$,  $t_0=0.1$, $\delta t = 0.1$, and $N=100$.
It can be seen that the analytic and numerical solutions are in excellent agreement.
Note, however, that the time-step  used in this calculation ({\em i.e.}, $\delta t = 0.1$) is much larger
than that used in our previous calculation ({\em i.e.}, $\delta t = 4\times
10^{-3}$), which employed an explicit differencing
scheme---see Fig.~\ref{fdiff1}. According to Eq.~(\ref{limit}), an explicit
 scheme is limited to time-steps less than about $5\times 10^{-3}$ for the problem under
investigation.
Thus, we have been able to exceed this limit by a factor of  20 with our
implicit scheme, yet still maintain numerical stability. 
Note that our Crank-Nicholson scheme is able to
obtain accurate results with a time-step as large as $0.1$ because it is
{\em second-order} in time. 

\begin{figure}
\epsfysize=3in
\centerline{\epsffile{Chapter06/diff3.eps}}
\caption{\em Diffusive evolution of a 1-d Gaussian  pulse.
Numerical  calculation performed using
$D=1$, $x_0=5$, $\delta t = 0.1$, and $N=100$. The pulse is evolved from $t=0.1$ to $t=1$. The
solid curve shows the initial condition at $t=0.1$, the dashed curve  the numerical solution
at $t=1$, and the dotted curve (obscured by the dashed curve)  the analytic solution at $t=1$. }\label{fdiff3}
\end{figure}

\section{2-D Problem with Dirichlet Boundary Conditions}
Let us consider the solution of the diffusion equation in two dimensions. Suppose
that
\begin{equation}\label{diff2deq}
\frac{\partial T(x,y,t)}{\partial t} = D\,\frac{\partial^2 T(x,y,t)}{\partial x^2}+D\,\frac{\partial^2 T(x,y,t)}
{\partial y^2},
\end{equation}
for $x_l\leq x\leq x_h$, and $0\leq y\leq L$. Suppose that $T(x,y,t)$ satisfies mixed
boundary conditions in the $x$-direction:
\begin{equation}
\alpha_l(t)\,T(x,y,t) + \beta_l(t)\,\frac{\partial T(x,y,t)}{\partial x} = \gamma_l(y,t),
\end{equation}
at $x=x_l$, and
\begin{equation}
\alpha_h(t) \,T(x,y,t)+ \beta_h(t)\,\frac{\partial T(x,y,t)}{\partial x} = \gamma_h(y,t),
\end{equation}
at $x=x_h$. Here, $\alpha_l$, $\beta_l$, {\em etc.}, are known functions of $t$,
whereas $\gamma_l$, $\gamma_h$ are known functions of $y$ and $t$.
 Furthermore, suppose that $T(x,y,t)$ satisfies the following simple Dirichlet boundary
conditions in the $y$-direction:
\begin{equation}
T(x,0,t) = T(x,L,t) = 0.
\end{equation}

As before, we discretize in time on the uniform grid $t_n=t_0+n\,\delta t$, for $n=0,1,2,\cdots$.
Furthermore, in the $x$-direction, we discretize on the uniform grid $x_i = x_l + i\,\delta x$, for
$i=0,N+1$, where $\delta x = (x_h-x_l)/(N+1)$. Finally, in the $y$-direction, we discretize
on the uniform grid $y_j = j\,\delta y$, for $j=0,J$, where $\delta y = L/J$. 
Adopting the Crank-Nicholson temporal differencing scheme discussed in Sect.~\ref{cn}, and
the second-order spatial differencing scheme outlined in
Sect.~\ref{pois1d}, Eq.~(\ref{diff2deq}) yields
\begin{eqnarray}\label{edisc}
\frac{T_{i,j}^{n+1}-T_{i,j}^n}{\delta t} - \frac{D}{2}\,
\frac{T_{i-1,j}^{n+1}-2\,T_{i,j}^{n+1}+T_{i+1,j}^{n+1}}{(\delta x)^2}
- \frac{D}{2}\left(\frac{\partial^2 T}{\partial y^2}\right)_{i,j}^{n+1}=&&\nonumber\\[0.5ex]
\frac{D}{2}\,
\frac{T_{i-1,j}^{n}-2\,T_{i,j}^{n}+T_{i+1,j}^{n}}{(\delta x)^2} +
\frac{D}{2}\left(\frac{\partial^2 T}{\partial y^2}\right)_{i,j}^{n},&&
\end{eqnarray}
where $T_{i,j}^n\equiv T(x_i,y_j,t_n)$. The discretized boundary conditions
take the form
\begin{eqnarray}\label{bcxdir}
T_{0,j}^n &=& \frac{\gamma_{l\,j}^n\,\delta x-\beta_l^n\,T_{1,j}^n}{\alpha_l^n\,\delta x -\beta_l^n},\\[0.5ex]
T_{N+1,j}^n &=& \frac{\gamma_{h\,j}^n\,\delta x + \beta_h^n\,T_{N,j}^n}{\alpha_h^n\,\delta x +\beta_h^n},
\label{bcxdir1}
\end{eqnarray}
plus
\begin{equation}\label{bcydir}
T_{i,0}^n = T_{i,J}^n = 0.
\end{equation}
Here, $\alpha_l^n\equiv \alpha_l(t_n)$, {\em etc.}, and $\gamma_{l\,j}^n\equiv
\gamma_l(y_j,t_n)$, {\em etc.}

Adopting the Fourier  method introduced in Sect.~\ref{fftsine}, we
write the $T_{i,j}^n$ in terms of their Fourier-sine harmonics:
\begin{equation}\label{recon}
T_{i,j}^n = \sum_{k=0}^{J} \hat{T}_{i,k}^n\,\sin(j\,k\,\pi/J),
\end{equation}
which automatically satisfies the boundary conditions (\ref{bcydir}).
The above expression can be inverted to give (see Sect.~\ref{sfft})
\begin{equation}\label{fftthat}
\hat{T}_{i,j}^n = \frac{2}{J}\sum_{k=0}^{J} T_{i,k}^n\,\sin(j\,k\,\pi/J).
\end{equation}
When Eq.~(\ref{edisc}) is written in terms of the $\hat{T}_{i,j}^n$, it reduces to
\begin{eqnarray}
-\frac{C}{2}\,\hat{T}_{i-1,j}^{n+1} + \left\{1+C\,(1+j^2\,\kappa^2/2)\right\}\,
\hat{T}_{i,j}^{n+1}-\frac{C}{2}\,\hat{T}_{i+1,j}^{n+1}
=&&\nonumber\\[0.5ex] \frac{C}{2}\,\hat{T}_{i-1,j}^{n} + \left\{1-C\,(1+j^2\,\kappa^2/2)\right\}\,
\hat{T}_{i,j}^{n}+\frac{C}{2}\,\hat{T}_{i+1,j}^{n},&&\label{tri1}
\end{eqnarray}
for $i=1,N$, and $j=0,J$. Here, $C= D\,\delta t/(\delta x)^2$, and $\kappa = \pi\,\delta x/L$.
Moreover, the boundary conditions (\ref{bcxdir}) and  (\ref{bcxdir1}) yield
\begin{eqnarray}
\hat{T}_{0,j}^n &=& \frac{\Gamma_{l\,j}^n\,\delta x-\beta_l^n\,\hat{T}_{1,j}^n}{\alpha_l^n\,\delta x -\beta_l^n},\\[0.5ex]
\hat{T}_{N+1,j}^n &=& \frac{\Gamma_{h\,j}^n\,\delta x + \beta_h^n\,\hat{T}_{N,j}^n}{\alpha_h^n\,\delta x +\beta_h^n},\label{tri3}
\end{eqnarray}
where
\begin{equation}\label{fftbc}
\hat{\Gamma}_{l\,j}^n = \frac{2}{J}\sum_{k=0}^{J} \gamma_{l,k}^n\,\sin(j\,k\,\pi/J),
\end{equation}
{\em etc.} Equations~(\ref{tri1})---(\ref{tri3}) constitute a set of $J+1$
uncoupled tridiagonal matrix equations for the $\hat{T}_{i,j}^{n+1}$, with one
equation for each separate value of $j$. 

In order to advance our solution by one time-step, we first Fourier transform the
$T_{i,j}^n$ and the boundary conditions, according to Eqs.~(\ref{fftthat}) and (\ref{fftbc}).
Next, we invert the $J+1$ tridiagonal equations (\ref{tri1})---(\ref{tri3}) to
 obtain the $\hat{T}_{i,j}^{n+1}$. Finally, we reconstruct the $T_{i,j}^n$ via
Eq.~(\ref{recon}).

\section{2-D Problem with Neumann Boundary Conditions}
Let us replace the Dirichlet  boundary conditions (\ref{bcydir}) by the following
simple Neumann boundary conditions:
\begin{equation}
\frac{\partial T(x,0,t)}{\partial y} = \frac{\partial T(x,L,t)}{\partial y} = 0.
\end{equation}
The  method of solution outlined in the previous section is unaffected, 
except that the Fourier-sine transforms
are replaced by Fourier-cosine transforms---see Sects.~\ref{fftcos} and \ref{sfft}.

\section{An Example 2-D Diffusion Equation Solver}
Listed below is an example 2-d diffusion equation solver which uses the Crank-Nicholson
scheme, as well as the previous listed tridiagonal matrix solver and the {\tt Blitz++}
library. Note the great structural similarity between this solver and the previously listed
2-d Poisson solver (see Sect.~\ref{poisson2d}).
{\small\begin{verbatim}
// CrankNicholson2D.cpp

// Function to evolve diffusion equation in 2-d:

//  dT / dt = D d^2 T / dx^2 +  D d^2 T / dy^2  for  xl <= x <= xh  
//                                              and  0 <= y <= L

//  alphaL T + betaL dT/dx = gammaL(y)  at x=xl

//  alphaH T + betaH dT/dx = gammaH(y)  at x=xh

// In y-direction, either simple Dirichlet boundary conditions:

//  T(x,0) = T(x,L) = 0

// or simple Neumann boundary conditions:

//  dT/dy(x,0) = dT/dy(x,L) = 0

// Matrix T assumed to be of extent N+2, J+1.
// Arrays gammaL, gammaH assumed to be of extent J+1.

// Now, (i,j)th elements of matrices correspond to

//  x_i = xl + i * dx    i=0,N+1

//  y_j = j * L / J      j=0,J

// Here, dx = (xh - xl) / (N+1) is grid spacing in x-direction.

// Now, C =  D dt / dx^2, and kappa = pi * dx / L

// Finally, Neumann=0/1 selects Dirichlet/Neumann bcs in y-direction.

// Uses Crank-Nicholson scheme.

#include <blitz/array.h>

using namespace blitz;

void fft_forward_cos (Array<double,1> f, Array<double,1>& F);
void fft_backward_cos (Array<double,1> F, Array<double,1>& f);
void fft_forward_sin (Array<double,1> f, Array<double,1>& F);
void fft_backward_sin (Array<double,1> F, Array<double,1>& f);
void Tridiagonal (Array<double,1> a, Array<double,1> b, Array<double,1> c, 
                  Array<double,1> w, Array<double,1>& u);

void CrankNicholson2D (Array<double,2>& T,
                       double alphaL, double betaL, Array<double,1> gammaL, 
                       double alphaH, double betaH, Array<double,1> gammaH,
                       double dx, double C, double kappa, int Neumann)
{
  // Find N and J. Declare local arrays.
  int N = T.extent(0) - 2;
  int J = T.extent(1) - 1;
  Array<double,2> TT(N+2, J+1), V(N+2, J+1);
  Array<double,1> GammaL(J+1), GammaH(J+1);

  // Fourier transform T  
  for (int i = 0; i <= N+1; i++)
    {
      Array<double,1> In(J+1), Out(J+1);

      for (int j = 0; j <= J; j++) In(j) = T(i, j);

      if (Neumann)
        fft_forward_cos (In, Out);
      else
        fft_forward_sin (In, Out);

      for (int j = 0; j <= J; j++) TT(i, j) = Out(j);
    }

  // Fourier transform boundary conditions
  if (Neumann)
    {
      fft_forward_cos (gammaL, GammaL);
      fft_forward_cos (gammaH, GammaH);
    }
  else
    {
      fft_forward_sin (gammaL, GammaL);
      fft_forward_sin (gammaH, GammaH);
    }

  // Construct source term
  for (int i = 1; i <= N; i++)
    for (int j = 0; j <= J; j++) 
      V(i, j) = 
        0.5 * C * TT(i-1, j) + 
        (1. - C * (1. + 0.5 * double (j * j) * kappa * kappa)) * TT(i, j) +
        0.5 * C * TT(i+1, j);
     
  // Solve tridiagonal matrix equations
  if (Neumann)
    {
      for (int j = 0; j <= J; j++)
       {  
          Array<double,1> a(N+2), b(N+2), c(N+2), w(N+2), u(N+2);

          // Initialize tridiagonal matrix
          for (int i = 2; i <= N; i++) a(i) = - 0.5 * C;
          for (int i = 1; i <= N; i++)
           b(i) = 1. + C * (1. + 0.5 * double (j * j) * kappa * kappa);
          b(1) += 0.5 * C * betaL / (alphaL * dx - betaL);
          b(N) -= 0.5 * C * betaH / (alphaH * dx + betaH);
          for (int i = 1; i <= N-1; i++) c(i) = - 0.5 * C;
	  
          // Initialize right-hand side vector
          for (int i = 1; i <= N; i++)
           w(i) = V(i, j);
          w(1) += 0.5 * C * GammaL(j) * dx / (alphaL * dx - betaL);
          w(N) += 0.5 * C * GammaH(j) * dx / (alphaH * dx + betaH);
	  
          // Invert tridiagonal matrix equation
          Tridiagonal (a, b, c, w, u);
          for (int i = 1; i <= N; i++) TT(i, j) = u(i);
       }
    }
  else
    {
      for (int j = 1; j < J; j++)
       { 
          Array<double,1> a(N+2), b(N+2), c(N+2), w(N+2), u(N+2);

          // Initialize tridiagonal matrix
          for (int i = 2; i <= N; i++) a(i) = - 0.5 * C;
          for (int i = 1; i <= N; i++)
           b(i) = 1. + C * (1. + 0.5 * double (j * j) * kappa * kappa);
          b(1) -= betaL / (alphaL * dx - betaL);
          b(N) += betaH / (alphaH * dx + betaH);
          for (int i = 1; i <= N-1; i++) c(i) = - 0.5 * C;
	  
          // Initialize right-hand side vector
          for (int i = 1; i <= N; i++)
           w(i) = V(i, j);
          w(1) += 0.5 * C * GammaL(j) * dx / (alphaL * dx - betaL);
          w(N) += 0.5 * C * GammaH(j) * dx / (alphaH * dx + betaH);
	  
          // Invert tridiagonal matrix equation
          Tridiagonal (a, b, c, w, u);
          for (int i = 1; i <= N; i++) TT(i, j) = u(i);
       }
      
      for (int i = 1; i <= N  ; i++) 
        {
          TT(i, 0) = 0.; TT(i, J) = 0.;
        }
    }

  // Reconstruct solution
  for (int i = 1; i <= N; i++)
    {
      Array<double,1> In(J+1), Out(J+1);

      for (int j = 0; j <= J; j++) In(j) = TT(i, j);

      if (Neumann)
        fft_backward_cos (In, Out);
      else
        fft_backward_sin (In, Out);

      for (int j = 0; j <= J; j++) T(i, j) = Out(j);
    }

  // Calculate i=0 and i=N+1 values  
  for (int j = 0; j <= J; j++)
    {
      T(0, j) = (gammaL(j) * dx - betaL * T(1, j)) /
        (alphaL * dx - betaL);
      T(N+1, j) = (gammaH(j) * dx + betaH * T(N, j)) /
        (alphaH * dx + betaH);
    }
}
\end{verbatim}}

\section{An Example 2-D Solution of the Diffusion Equation}
Let us now solve the diffusion equation in 2-d using the finite difference technique discussed above.
We seek the solution of Eq.~(\ref{diff2deq}) in the region $0\leq x\leq 1$ and
$0\leq y\leq 1$, subject to the following initial condition at $t=0$:
\begin{eqnarray}
T(x,y,0) &=& 1\mbox{\hspace{1cm}for~~$|x-0.5|<0.1$~~and~~$|y-0.5|<0.1$,}\nonumber\\[0.5ex]
T(x,y,0) &=& 0\mbox{\hspace{1cm}otherwise.}
\end{eqnarray}
The boundary conditions are simply $T(0,y)=T(1,y)=T(x,0)=T(x,1)=0$. 

\begin{figure}
\epsfysize=5in
\centerline{\epsffile{Chapter06/temp.eps}}
\caption{\em Diffusion in two dimensions.
Numerical  calculation performed using
$D=1$, $\delta t = 10^{-4}$, and $N=J=128$. Density plots of $T(x,y,t)$ are shown at
$t=0.000$ (top-left), $t=0.001$ (top-right), $t=0.002$ (middle-left), $t=0.003$
(middle-right), $t=0.004$ (bottom-left), and $t=0.005$ (bottom-right). }
\label{diff2d6}
\end{figure}

Figure~\ref{diff2d6} shows the evolution of $T(x,y,t)$ for a calculation performed
with the previously listed 2-d diffusion equation solver 
using $D=1$, $\delta t = 10^{-4}$, and $N=J=128$.

\section{3-D Problems}
The techniques discussed above for solving
the diffusion equation in two dimensions with a restricted class of boundary conditions can easily
be generalized to three dimensions. In the 3-d case, it is necessary to Fourier
transform in {\em two} directions (the $y$ and $z$ directions, say) in order to reduce the
problem to a system of uncoupled tridiagonal matrix equations. These equations can be inverted
in the usual manner, and the solution can then be reconstructed via a double inverse Fourier
transform.