\chapter{Scientific Programming in C}

\section{Introduction}
As we have already mentioned, C is a  flexible, extremely powerful, high-level programming language which was
initially designed for writing operating systems and  system applications. 
In fact, all UNIX operating systems, as well
as most UNIX applications ({\em e.g.}, text editors, window managers,
{\em etc.}) are written in C. However, C is also
an excellent vehicle for scientific programming, since, almost by definition,
a good scientific programming language must be powerful, flexible, and high-level.
Having said this, many of the features of C which send computer scientists into
raptures are not particularly relevant to the needs of the scientific programmer. 
Hence, in the following, we shall only describe  that subset of the C 
language which is
really necessary to write scientific programs. It may be objected that our cut-down version of C bears a suspicious resemblance
to FORTRAN.   However, this resemblance is hardly surprising. After all, FORTRAN is a high-level programming
language which was specifically designed with scientific computing in mind. 

As discussed previously, C++  is an extension of the C language whose main aim
is to facilitate object-orientated programming.
The object-orientated features of C++ are superfluous to our
needs in this course. However,  C++ incorporates some new, non-object-orientated features
which are extremely useful to the scientific programmer. We shall briefly discuss
these features towards the end of this section. Finally, we shall describe 
 some prewritten C++ classes which allow us to incorporate complex arithmetic
(which is not part of the C language), variable size arrays, and graphics into our programs.

\section{Variables}
Variable names in C can consist of letters and numbers in any order, except that
the first character must be a letter. Names are {\em case sensitive}, so
upper- and lower-case letters are not interchangeable. The underscore
character (\verb{_{) can also be included in variable names, and is treated as a
letter. There is no restriction on the length of  names in C. 
Of course, variable names are not allowed to clash with keywords that play a
special role in 
the C language, such as {\tt int, double, if, return, void}, {\em etc}.
The following are examples of valid variable names in C:
{\small\begin{verbatim}
x  c14  area  electron_mass  TEMPERATURE 
\end{verbatim}}

The C language supports a great variety of different data types. However,
the two  data types which occur most often 
in scientific programs are {\em integer},
denoted {\tt int}, and {\em floating-point}, denoted {\tt double}. (Note that
variables of the most basic floating-point data type {\tt float} are not generally stored to sufficient
precision by the computer to be of much use in scientific programming.)
The data type ({\tt int} or {\tt double}) of every variable in a C program
must be declared {\em before}\/ that variable can appear in an executable statement.

{\em Integer constants}\/ in C are denoted, in the regular fashion, by strings
of arabic numbers: {\em e.g.},
{\small\begin{verbatim}
0  57  4567  128933
\end{verbatim}}\noindent
{\em Floating-point constants}\/ can be written in either regular or scientific
notation: {\em e.g.},
{\small\begin{verbatim}
0.01  70.456  3e+5  .5067e-16
\end{verbatim}}

{\em Strings}\/ are mainly used in scientific programs for data input and output
purposes.
A string consists of any number of consecutive characters (including blanks)
enclosed in double quotation marks: {\em e.g.},
{\small\begin{verbatim}
"red"  "Austin TX, 78723"  "512-926-1477"
\end{verbatim}}\noindent
Line-feeds can be incorporated into strings via the {\em escape sequence}\/
{\verb{\n{: {\em e.g.},
{\small\begin{verbatim}
"Line 1\nLine 2\nLine 3"
\end{verbatim}}\noindent
The above string would be displayed on a computer terminal as{\small\begin{verbatim}
Line 1
Line 2
Line 3
\end{verbatim}}\noindent

A {\em declaration}\/ associates a group of variables with a specific data
type. As mentioned previously, all variables must be declared before they can appear in executable
statements. A declaration consists of a data type followed by one or more
variable names, ending in a semicolon. For instance,
{\small\begin{verbatim}
int    a, b, c;
double    acc, epsilon, t;
\end{verbatim}}\noindent
In the above, {\tt a}, {\tt b}, and {\tt c} are declared to be integer
variables, whereas {\tt acc}, {\tt epsilon}, and {\tt t} are declared to
be floating-point variables. 

A type declaration can also be used to
assign initial values to variables.
Some examples of how to do this are given below:
{\small\begin{verbatim}
int    a = 3, b = 5;
double    factor = 1.2E-5;
\end{verbatim}}\noindent
Here, the integer variables {\tt a} and {\tt b} are assigned the initial
values 3 and 5, respectively, whereas the floating-point variable
{\tt factor} is assigned the initial value $1.2\times 10^{-5}$. 

Note that there is no restriction on the length of a type declaration:
such a  declaration can even be split over many lines,
 so long as its end is signaled by a
semicolon. However, all declaration statements in a program (or program segment) must
occur {\em prior}\/ to the first executable statement.

\section{Expressions and Statements}
An {\em expression}\/ represents a single data item---usually  a number. The
expression may consist of a single entity, such as a constant or variable, 
or it may consist of some combination of such entities, interconnected by
one or more 
{\em operators}. Expressions can also represent logical
conditions which are either true or false. However, in C, the conditions
true and false are represented by the integer values {\tt 1} and {\tt 0},
respectively. Several simple expressions are given below:
{\small\begin{verbatim}
a + b
x = y
t = u + v
x <= y
++j
\end{verbatim}}\noindent
The first expression, which employs the
{\em addition operator}\/ (\verb{+{), represents the sum of the values assigned to
variables {\tt a} and {\tt b}. The second expression involves the {\em assignment
operator}\/ (\verb{={), and causes the value represented by {\tt y} to be
assigned to {\tt x}. In the third expression, the value of the expression
{\tt (u + v)} is assigned to {\tt t}. The fourth expression takes the value
{\tt 1} (true) if the value of {\tt x} is less than or equal to the
value of {\tt y}. Otherwise, the expression takes the value {\tt 0} (false).
Here, \verb{<={ is a {\em relational operator}\/ that compares the values
of {\tt x} and {\tt y}. The final example causes the value of {\tt j}
to be increased by {\tt 1}. Thus, the expression is equivalent to
{\small\begin{verbatim}
j = j + 1
\end{verbatim}}\noindent
The {\em increment}\/ (by unity) operator \verb{++{ is called a {\em unary}\/ operator,
because it only possesses one operand. 

A {\em statement}\/ causes the computer to carry out some definite action. There
are three different classes of statements in C: {\em expression statements},
{\em compound statements}, and {\em control statements}. 

An expression statement consists of an expression followed by a semicolon. The
execution of such a  statement causes the associated expression to be evaluated. 
For example:
{\small\begin{verbatim}
a = 6;
c = a + b;
++j;
\end{verbatim}}\noindent
The first two expression statements both cause the value of the expression
on the right of the equal sign to be assigned to the variable on the left.
The third expression statement causes the value of {\tt j} to be incremented
by {\tt 1}.
Again, there is no restriction on the length of an expression statement:
such a statement can even be split over many lines,
 so long as its end is signaled by a
semicolon.

A compound statement consists of several individual statements enclosed within
a pair of braces \verb?{?~\verb{}{. The individual statements may themselves
be expression statements, compound statements, or control statements. Unlike
expression statements, compound statements do {\em not}\/ end with semicolons. 
A typical compound statement is shown below:
{\small\begin{verbatim}
{
  pi = 3.141593;
  circumference = 2. * pi * radius;
  area = pi * radius * radius;
}
\end{verbatim}}\noindent
This particular compound statement  consists of three expression statements, but
acts like a single entity in the program in which it appears. 
 
A {\em symbolic constant}\/ is a name that substitutes for a sequence of
characters. The characters may represent either a number or a string. 
When a program is compiled, each occurrence of a symbolic constant is
replaced by its corresponding character sequence. Symbolic constants are
usually defined at the beginning of a program, by writing
{\small\begin{verbatim}
#define  NAME  text
\end{verbatim}}\noindent
where \verb{NAME{ represents a symbolic name, typically written in upper-case
letters, and \verb{text{ represents the sequence of characters that is
associated with that  name. Note that \verb{text{ does {\em not}\/
end with a semicolon, since a symbolic constant definition is not a true
C statement. In fact, during compilation,
the resolution of symbolic names is performed
(by the {\em C preprocessor}) before the start of true compilation. 
For instance, suppose that a C program  contains the following  
symbolic constant definition:
{\small\begin{verbatim}
#define  PI  3.141593
\end{verbatim}}\noindent
Suppose, further, that the program contains the statement
{\small\begin{verbatim}
area = PI * radius * radius;
\end{verbatim}}\noindent
During the compilation process, the preprocessor replaces each
occurrence of the  symbolic constant {\tt PI}  by its corresponding text. Hence,
the above statement becomes
{\small\begin{verbatim}
area = 3.141593 * radius * radius;
\end{verbatim}}\noindent
Symbolic constants are particularly useful in scientific programs for
representing constants of nature, such as the mass of an electron, 
the speed of light, {\em etc}. Since these quantities are fixed, there is
little point in assigning  variables in which to store them.

\section{Operators}
As we have seen, general expressions are formed by joining together constants
and variables via various operators. Operators in C fall into five main classes:
{\em arithmetic operators}, {\em unary operators}, {\em relational and logical
operators}, {\em assignment operators}, and the {\em conditional operator}. 
Let us, now, examine each of these classes in detail. 

There are {\em four}\/ main arithmetic operators in C. These are:
{\small\begin{verbatim}
addition          +
subtraction       -
multiplication    *
division          /
\end{verbatim}}\noindent
Unbelievably, there is no built-in  exponentiation operator in C (C was written by
computer scientists)! Instead, there is a {\em library function}\/ ({\tt pow})
which carries out this operation (see later).

It is poor programming practice to mix types in arithmetic expressions. In other
words, the two operands operated on by the addition, subtraction, multiplication,
or division operators should both be either  of type {\tt int} or type {\tt double}.
The value of an expression can be converted to a different data type
by prepending the name of the desired data type, enclosed in parenthesis. This
type of construction is known as a {\em cast}. Thus, to convert an integer
variable {\tt j} into a floating-point variable with the same value, we would
write
{\small\begin{verbatim}
(double) j
\end{verbatim}}\noindent
Finally, to avoid mixing data types when dividing a floating-point variable
{\tt x} by an integer variable {\tt i}, we would write
{\small\begin{verbatim}
x / (double) i
\end{verbatim}}\noindent
Of course, the result of this operation would be of type {\tt double}.

The operators within C are grouped hierarchically according to their
{\em precedence}\/  ({\em i.e.}, their order of evaluation).
Amongst the arithmetic operators, \verb{*{ and \verb{/{ have
precedence over \verb{+{ and \verb{-{. In other words, 
when evaluating expressions, C performs multiplication and division 
operations prior to addition and subtraction operations. Of course,
the rules of precedence can always be bypassed by judicious use of parentheses. 
Thus, the
expression 
{\small\begin{verbatim}
a - b / c + d
\end{verbatim}}\noindent
is equivalent to the unambiguous expression 
{\small\begin{verbatim}
a - (b / c) + d
\end{verbatim}}\noindent
since division takes precedence over addition and subtraction.


The distinguishing feature of unary operators is that 
they only act on  single operands. The most common unary operator
is the {\em unary minus}, which occurs
when a numerical constant, variable, or expression
is preceded by a minus sign. Note that the unary minus is distinctly different
from the arithmetic operator (\verb{-{) which denotes subtraction, since the latter
operator acts on two separate operands. The two other common unary
operators are the {\em increment operator}, \verb{++{, and the
{\em decrement operator}, \verb{--{. The increment operator causes its
operand to be increased by {\tt 1}, whereas the decrement operator causes
its operand to be decreased by {\tt 1}. For example,
{\tt --i} is equivalent to {\tt i = i - 1}. A {\em cast}\/ is also considered to be
a unary operator. Note that unary operators have precedence over arithmetic
operators. Hence, \verb{- x + y{ is equivalent to the unambiguous expression
\verb{(-x) + y{, since the unary minus operator has precedence over the 
addition operator.

Note that there is a subtle distinction between the expressions \verb{a++{ and \verb{++a{.
In the former case, the value of the variable \verb{a{ is returned {\em before}\/
it is incremented. In the latter case, the value of \verb{a{ is returned {\em after}\/
 incrementation. Thus,
{\small\begin{verbatim}
b = a++;
\end{verbatim}}\noindent
is equivalent to
{\small\begin{verbatim}
b = a;
a = a + 1;
\end{verbatim}}\noindent
whereas 
{\small\begin{verbatim}
b = ++a;
\end{verbatim}}\noindent
is equivalent to
{\small\begin{verbatim}
a = a + 1;
b = a;
\end{verbatim}}\noindent
There is a similar distinction between the expressions \verb{a--{ and \verb{--a{.

There are four relational operators in C. These are:
{\small\begin{verbatim}
less than                   <
less than or equal to       <=
greater than                >
greater than or equal to    >=
\end{verbatim}}\noindent
The precedence of these operators is lower than that of arithmetic operators.

Closely associated with the relational operators are the two {\em equality
operators}:
{\small\begin{verbatim}
equal to        ==
not equal to    !=
\end{verbatim}}\noindent
The precedence of the equality operators is below that of the relational
operators. 

The relational and equality operators are used to form logical expressions,
which represent conditions that are either true or false. The resulting expressions
are of type {\tt int}, since true is represented by the integer value {\tt 1}
and false by the integer value {\tt 0}. For example, the
expression {\tt i < j} is true (value {\tt 1}) if the value of {\tt i} 
is less than the value of {\tt j}, and false (value {\tt 0}) otherwise. Likewise,
the expression {\tt j == 3} is true  if the value of {\tt j}
is equal to {\tt 3}, and false  otherwise. 

C also possess two {\em logical operators}. These are:
{\small\begin{verbatim}
&&        and
||        or
\end{verbatim}}\noindent
The logical operators act on operands which are themselves logical expressions.
The net effect is to combine the individual logical expressions into more
complex expressions that are either true or false. The result of
a {\em logical and}\/ operation is only  true if both operands are true,
whereas the result of a {\em logical or}\/ operation is only false
if both operands are false. For instance, 
the expression {\tt (i >= 5) \verb{&&{ (j == 3)} is true 
if the value of {\tt i} is greater than or equal to {\tt 5} {\em and}\/ 
the value of {\tt j} is equal to {\tt 3}, otherwise it
is false. The precedence of the {\em logical
and}\/ operator is higher than that of the {\em logical or}\/ operator, but lower 
than that of the equality operators. 

C also includes the unary operator \verb{!{ that negates the value of a logical
expression: {\em i.e.}, it causes an expression that is originally true to become
false, and {\em vice versa}. This operator is referred to as the
{\em logical negation} or {\em logical not}\/ operator. For instance,
the expression \verb{!(k == 4){ is true if the value of {\tt k} is not
equal to {\tt 4}, and false otherwise.

Note that it is poor programming practice to rely too heavily on operator 
precedence,
since such reliance tends to  makes  C programs very hard for other people
to follow. For instance, instead of writing
{\small\begin{verbatim}
i + j == 3 && i * l >= 5
\end{verbatim}}\noindent
and relying on the fact that arithmetic operators have precedence over
relational and equality operators, which, in turn, have precedence over
logical operators, it is better to write
{\small\begin{verbatim}
((i + j) == 3) && (i * l >= 5)
\end{verbatim}}\noindent
whose meaning is fairly unambiguous, even to people who cannot remember the
order of precedence of the various operators in C. 

The most common assignment operator in C is \verb{={. For instance, the
expression
{\small\begin{verbatim}
f = 3.4
\end{verbatim}}\noindent
causes the floating-point value {\tt 3.4} to be assigned to the
variable {\tt f}. Note that the assignment operator \verb{={ and the
equality operator \verb{=={ perform completely different functions
in C, and should not be confused. Multiple assignments are permissible
in C. For example,
{\small\begin{verbatim}
i = j = k = 4
\end{verbatim}}\noindent
causes the integer value {\tt 4} to be assigned to {\tt i}, {\tt j}, and
{\tt k}, simultaneously. Note, again, that it is poor programming
practice to mix data types in assignment expressions. Thus, the data types
of the constants or variables on either side of the \verb{={ sign
should always match. 

C contains four additional assignment operators: \verb{+={, \verb{-={,
\verb{*={, and \verb{/={. The expression
{\small\begin{verbatim}
i += 6
\end{verbatim}}\noindent
is equivalent to \verb{i = i + 6{. Likewise, the expression
{\small\begin{verbatim}
i -= 6
\end{verbatim}}\noindent
is equivalent to \verb{i = i - 6{. The expression 
{\small\begin{verbatim}
i *= 6
\end{verbatim}}\noindent
is equivalent to  \verb{i = i * 6{. Finally, the expression 
{\small\begin{verbatim}
i /= 6
\end{verbatim}}\noindent
is equivalent to  \verb{i = i / 6{.
Note that the precedence of assignment operators is below that of 
all the operators discussed previously. 

Simple conditional operations can be carried out with the {\em conditional
operator}\/ \verb{(? :){. An expression that makes use
of the conditional operator is called a {\em conditional expression}. 
Such an expression takes the general form
{\small\begin{verbatim}
expression 1  ?  expression 2  :  expression 3
\end{verbatim}}\noindent
If \verb{expression 1{ is true ({\em i.e.}, if its value is nonzero)
then \verb{expression 2{ is evaluated and becomes the value of the conditional
expression. On the other hand, if 
 \verb{expression 1{ is false ({\em i.e.}, if its value is zero)
then \verb{expression 3{ is evaluated and becomes the value of the conditional
expression. For instance, the expression 
{\small\begin{verbatim}
(j < 5) ? 12 : -6
\end{verbatim}}\noindent
takes the value {\tt 12} if the value of {\tt j} is less than {\tt 5}, and
the value {\tt -6} otherwise. The assignment statement
{\small\begin{verbatim}
k = (i < 0) ? n : m
\end{verbatim}}\noindent
causes the value of {\tt n} to be assigned to the variable {\tt k}
if the value of {\tt i} is less than zero, and the value of
{\tt m} to be assigned to {\tt k} otherwise. The precedence of
the conditional operator is just above that of the assignment operators. 

As we have already mentioned, scientific programs tend to be extremely resource intensive.
Scientific programmers should, therefore, always be on the lookout for methods
of speeding up the execution of their codes. It is important to realize that
multiplication (\verb{*{) and division (\verb{/{) operations consume {\em considerably
more}\/ CPU time that addition (\verb{+{), subtraction (\verb{-{), comparison, or assignment
operations. Thus, a simple rule of thumb for writing efficient code is to try to
avoid redundant multiplication and division operations. This is particularly important
for sections of code which are executed repeatedly: {\em e.g.}, code which lies within
control loops. The classic illustration of this point is the evaluation of a
polynomial. The most straightforward method of evaluating  (say) a fourth-order polynomial
would be to write something like:
{\small\begin{verbatim}
p = c_0 + c_1 * x + c_2 * x * x + c_3 * x * x * x + c_4 * x * x * x * x
\end{verbatim}}\noindent 
Note that the above expression employs {\em ten}\/ expensive multiplication operations. 
However, this number can be reduced to {\em four}\/ via a simple algebraic rearrangement:
{\small\begin{verbatim}
p = c_0 + x * (c_1 + x * (c_2 + x * (c_3 + x * c_4)))
\end{verbatim}}\noindent 
Clearly, the latter expression is far more computationally efficient than the former.

\section{Library Functions}
The C language is accompanied by a number of standard {\em library functions}\/ which
carry out various useful tasks. In particular, all input and output operations
({\em e.g.}, writing to the terminal) and all math operations ({\em e.g.}, 
evaluation of sines and cosines) are implemented by library functions. 

In order to use a library function, it is necessary to call the appropriate
{\em header file}\/ at the beginning of the program. The header file
informs the program of the name, type, and number and type of arguments,
of all of  the functions contained in the library in question. 
A header file is called via the preprocessor statement
{\small\begin{verbatim}
#include   <filename>
\end{verbatim}}\noindent
where \verb{filename{ represents the name of the header file. 

A library function is accessed  by simply writing the function  name, followed
by a list of {\em arguments}, which represent the information being passed to
the function. The arguments must be enclosed in parentheses, and separated by commas:
they  can be constants, variables, or more complex expressions. Note
that the parentheses must be present even when there are no arguments.

The C {\em math library}\/ has the header file {\tt math.h}, and contains the
following useful functions:
{\small\begin{verbatim}
Function        Type        Purpose
--------        ----        -------

acos(d)         double      Return arc cosine of d (in range 0 to pi)
asin(d)         double      Return arc sine of d (in range -pi/2 to pi/2)
atan(d)         double      Return arc tangent of d (in range -pi/2 to pi/2)
atan2(d1, d2)   double      Return arc tangent of d1/d2 (in range -pi to pi)
cbrt(d)         double      Return cube root of d
cos(d)          double      Return cosine of d
cosh(d)         double      Return hyperbolic cosine of d
exp(d)          double      Return exponential of d
fabs(d)         double      Return absolute value of d
hypot(d1, d2)   double      Return sqrt(d1 * d1 + d2 * d2)
log(d)          double      Return natural logarithm of d
log10(d)        double      Return logarithm (base 10) of d
pow(d1, d2)     double      Return d1 raised to the power d2
sin(d)          double      Return sine of d
sinh(d)         double      Return hyperbolic sine of d
sqrt(d)         double      Return square root of d
tan(d)          double      Return tangent of d
tanh(d)         double      Return hyperbolic tangent of d
\end{verbatim}}\noindent
Here, {\tt Type} refers to the data type of the quantity that is returned by
the function. Moreover, {\tt d}, {\tt d1}, {\em etc.} indicate arguments of
type {\tt double}. 

A program that makes use of the C math library would contain the statement
{\small\begin{verbatim}
#include   <math.h>
\end{verbatim}}\noindent
close to its start. In the body of the program,
a statement like
{\small\begin{verbatim}
x = cos(y);
\end{verbatim}}\noindent
would cause the variable {\tt x} to be assigned a value which is the cosine of
the value of the variable {\tt y} (both {\tt x} and {\tt y} should be of
type {\tt double}).

Note that math library functions tend to be {\em extremely expensive}\/ in terms of CPU time, and
should, therefore, only be employed when absolutely necessary. The classic illustration
of this point is the use of the {\tt pow()} function. This function assumes that, in general,
it will be called with a {\em fractional}\/ power, and, therefore, implements a full-blown (and
very expensive) 
series expansion. Clearly, it is not computationally efficient to use this function to
square or cube a quantity. In other words, if a quantity needs to be raised to a small,
positive integer power then this should be implemented {\em directly}, instead of
using the {\tt pow()} function: {\em i.e.}, we should write \verb{x * x{ rather than
{\tt pow(x, 2)}, and \verb{x * x * x{ rather than {\tt pow(x, 3)}, {\em etc.}
(Of course, a properly designed exponentiation function would realize that it is more
efficient to evaluate small positive integer powers by the direct method. Unfortunately,
the {\tt pow()} function was written by computer scientists!)

The C math library comes with a useful set of predefined mathematical constants:{\small\begin{verbatim}
Name        Description    
----        -----------    

M_PI        Pi, the ratio of a circle's circumference to its diameter. 
M_PI_2      Pi divided by two.
M_PI_4      Pi divided by four.    
M_1_PI      The reciprocal of pi (1/pi).
M_SQRT2     The square root of two.
M_SQRT1_2   The reciprocal of the square root of two 
             (also the square root of 1/2). 
M_E         The base of natural logarithms.
\end{verbatim}}\noindent

The other library functions commonly used in C programs will be introduced,
as appropriate, during the remainder of this discussion. 
 

\section{Data Input and Output}\label{sio}
Data input and output operations in C are carried out by the standard input/output
library (header file: {\tt stdio.h}) via the functions {\tt scanf}, {\tt printf},
{\tt fscanf}, and {\tt fprintf}, which read and write data from/to the terminal,
and  from/to a data file, respectively. The additional functions
{\tt fopen} and {\tt fclose} open and close, respectively, connections between
a C program and a data file.
In the following, these
functions are described in detail.

The {\tt scanf} function reads data from standard input (usually, the terminal).
A call to this function takes the general form
{\small\begin{verbatim}
scanf(control_string,  arg1,  arg2,  arg3,  ...)
\end{verbatim}}\noindent
where {\tt control\verb{_{string} refers to a character string containing certain required
formatting information, and {\tt arg1}, {\tt arg2}, {\em etc.}, are arguments
that represent the individual input data items. 

The control string consists of individual groups of characters, with one character
group for each data input item. In its simplest form, each character group
 consists of a
percent sign (\verb{%{), followed by  a set of {\em conversion characters}\/ which
indicate the type of the corresponding data item. The two most
useful sets of conversion characters are as follows:
{\small\begin{verbatim}
Character      Type
---------      ----

d              int
lf             double
\end{verbatim}}\noindent
The arguments are a set of variables whose types match the corresponding
character groups in the control string. For reasons which will become apparent
later on, {\em each variable name must be preceded by an ampersand}\/ (\verb{&{). 
Below is a typical application of the {\tt scanf} function:
{\small\begin{verbatim}
#include  <stdio.h>
. . . 
int  k;
double  x, y;
. . .
scanf("%d %lf %lf", &k, &x, &y);
. . .
\end{verbatim}}\noindent
In this example, the {\tt scanf} function reads an integer value and
two floating-point values, from standard input,
 into the integer variable {\tt k} and the
two floating-point variables {\tt x} and {\tt y}, respectively. 

The {\tt scanf} function returns an integer equal to the number of
data values successfully read from standard input, which can be fewer
than expected, or even zero, in the event of a matching failure. The
special value {\tt EOF} (which on most systems corresponds to $-1$)
is returned if an end-of-file is reached before any attempted conversion occurs.
The following code snippet gives an example of how the {\tt scanf} function can
be checked for error-free input:
{\small\begin{verbatim}
#include  <stdio.h>
. . . 
int  check_input;
double  x, y, z;
. . .
check_input = scanf("%lf %lf %lf", &x, &y, &z);
if (check_input < 3)
 {
  printf("Error during data input\n");
  . . .
 }
. . .
\end{verbatim}}\noindent
See later for an explanation of the {\tt if()} construct.

The {\tt printf} function writes data to standard output (usually, the terminal).
A call to this function takes the general form
{\small\begin{verbatim}
printf(control_string,  arg1,  arg2,  arg3,  ...)
\end{verbatim}}\noindent
where {\tt control\verb{_{string} refers to a character string containing 
formatting information, and {\tt arg1}, {\tt arg2}, {\em etc.}, are arguments
that represent the individual output data items. 

The control string consists of individual groups of characters, with
one character group for each output data item. 
In
its simplest form, each character group
 consists of a
percent sign (\verb{%{), followed by a {\em conversion character}\/ which
controls the format of the corresponding data item. The most
useful conversion characters are as follows:
{\small\begin{verbatim}
Character      Meaning
---------      -------

d              Display data item as signed decimal integer
f              Display data item as floating-point number without exponent
e              Display data item as floating-point number with exponent    
\end{verbatim}}\noindent
The arguments are a set of variables whose types match the corresponding
character groups in the control string
({\em i.e.}, type {\tt int} for {\tt d} format, and
type {\tt double} for {\tt f} or {\tt e} format). In contrast to the {\tt scanf} function,
the arguments are {\em not}\/ preceded by ampersands. 
Below is a typical application of the {\tt scanf} function:
{\small\begin{verbatim}
#include  <stdio.h>
. . . 
int  k = 3;
double  x = 5.4, y = -9.81;
. . .
printf("%d %f %f\n", k, x, y);
. . .
\end{verbatim}}\noindent
In this example, the program outputs the values of the integer variable
{\tt k} and the floating-point variables {\tt x} and {\tt y} to
the terminal. Executing the program produces the following output:
{\small\begin{verbatim}
3 5.400000 -9.810000   
%
\end{verbatim}}\noindent
Note that the 	purpose of the escape sequence \verb{\n{ in the control string is
to generate a line-feed after the three data items have been written to the
terminal. 

Of course, the {\tt printf} function can also be used to write a simple text string
to the terminal: {\em e.g.}, 
{\small\begin{verbatim}
printf(text_string)
\end{verbatim}}\noindent
Ordinary text  can also be incorporated into the control string described
above. 


An example illustrating somewhat more advanced use of the
{\tt printf} function is given below:
{\small\begin{verbatim}
#include  <stdio.h>
. . . 
int  k = 3;
double  x = 5.4, y = -9.81;
. . .
printf("k = %3d  x + y = %9.4f  x*y = %11.3e\n", k, x + y, x*y);
. . .
\end{verbatim}}\noindent
Executing the program produces the following output:
{\small\begin{verbatim}
k =   3  x + y =   -4.4100  x*y =  -5.297e+01        
%
\end{verbatim}}\noindent
Note that the final two arguments of the {\tt printf} function are
arithmetic expressions. Note, also, the incorporation of explanatory text
into the control string. 

The character sequence \verb{%3d{ in the control
string indicates that the associated data item should be output as
a signed decimal integer occupying a field whose width is {\em at least}\/
3 characters. More generally, the character sequence \verb{%nd{
indicates that the associated data item should be output as
a signed decimal integer occupying a field whose width is {\em at least}\/
{\tt n}  characters.
If the number of characters in the data item is less than
{\tt n} characters, then the data item
is preceded by enough leading blanks to fill the specified field. On the
other hand, if the data item exceeds the specified field width then additional
space is allocated to the data item, such that the entire data item is displayed. 

The character sequence \verb{%9.4f{ in the control string indicates that
the associated data item should be output as a floating-point number, in
non-scientific format, which occupies a field of at least 9 characters, and
has 4 figures after the decimal point.  More generally, the character
sequence \verb{%n.mf{ indicates that the associated data item should be
output as a floating-point number, in non-scientific format, which occupies a field 
of at least {\tt n} characters, and
has {\tt m} figures after the decimal point.

Finally, the character sequence \verb{%11.3e{ in the control string indicates that
the associated data item should be output as a floating-point number, in
scientific format, which occupies a field of at least 11 characters, and
has 3 figures after the decimal point.  More generally, the character
sequence \verb{%n.me{ indicates that the associated data item should be
output as a floating-point number, in scientific format, which occupies a field 
of at least {\tt n} characters, and
has {\tt m} figures after the decimal point.

The {\tt printf} function returns an integer equal to the number of
printed characters, or a negative value if there was an output error.

When working with a {\em data file}, the first step is to establish a
{\em buffer area}, where information is temporarily stored whilst
being transferred between the program and the  file. This buffer
area allows information to be read or written to the data file in question more
rapidly than would otherwise be possible. A buffer area is established
by writing
{\small\begin{verbatim}
FILE *stream;
\end{verbatim}}\noindent
where {\tt FILE} (upper-case letters required) is a special structure type
that establishes a buffer area, and {\tt stream} is the identifier of the 
created buffer
area.  Note that a buffer area 
is often referred to as an input/output {\em stream}. The
meaning of the asterisk (\verb{*{) that precedes the identifier of the stream,
in the above statement,
will become clear later on. It is, of course, possible to establish
multiple input/output streams (provided that  their identifiers are
distinct).

A data file must be opened and attached to a specific input/output stream before
it can be created or processed. This operation is performed by the
function {\tt fopen}. A typical call to {\tt fopen} takes the
form
{\small\begin{verbatim}
stream = fopen(file_name, file_type);
\end{verbatim}}\noindent
where {\tt stream} is the identifier of
the input/output stream to which the file is to be attached, and 
\verb{file_name{ and \verb{file_type{ are character strings that represent
the name of the data file and the manner in which the data file will be
utilized, respectively. The \verb{file_type{ string must be one of the strings
listed below:
{\small\begin{verbatim}
file_type         Meaning
---------         -------

"r"        Open existing file for reading only
"w"        Open new file for writing only (Any existing file 
            will be overwritten)
"a"        Open existing file in append mode. (Output will be
            appended to the file)
\end{verbatim}}\noindent
The {\tt fopen} function returns the integer value {\tt NULL} (which on most systems
corresponds to zero) in the event
of an error.

A data file must also be closed at the end of the program. This operation
is performed by the function {\tt fclose}. The syntax for a call to
{\tt fclose} is simply
{\small\begin{verbatim}
fclose(stream);
\end{verbatim}}\noindent
where {\tt stream} is the name of the input/output stream which is to
be deattached from a data file. The {\tt fclose} function
returns the integer value {\tt 0} upon successful completion, otherwise it
returns the special value {\tt EOF}.



Data can be read from an open data file using the {\tt fscanf} function, whose
syntax is
{\small\begin{verbatim}
fscanf(stream, control_string,  arg1,  arg2,  arg3,  ...)
\end{verbatim}}\noindent
Here, {\tt stream} is the identifier of the input/output stream to which the
file is attached, and the remaining arguments have exactly the same format
and meaning as the corresponding arguments for the {\tt scanf} function. 
The return values of {\tt fscanf} are similar to those of the {\tt scanf} function.

Likewise, data can be written to  an open data file using the {\tt fprintf} 
function, whose
syntax is
{\small\begin{verbatim}
fprintf(stream, control_string,  arg1,  arg2,  arg3,  ...)
\end{verbatim}}\noindent
Here, {\tt stream} is the identifier of the input/output stream to which the
file is attached, and the remaining arguments have exactly the same format
and meaning as the corresponding arguments for the {\tt printf} function.
The return values of {\tt fprintf} are similar to those of the {\tt printf} function.

An example of a C program which outputs data to the file ``data.out'' is given
below:
{\small\begin{verbatim}
#include  <stdio.h>
. . .
int  k = 3;
double  x = 5.4, y = -9.81;
FILE *output;
. . .
output = fopen("data.out", "w");
if (output == NULL)
 {
  printf("Error opening file data.out\n");
  . . .
 }
. . .
fprintf(output, "k = %3d  x + y = %9.4f  x*y = %11.3e\n", k, x + y, x*y);
. . .
fclose(output);
. . .
\end{verbatim}}\noindent 
On execution, the above program will write the line
{\small\begin{verbatim}
k =   3  x + y =   -4.4100  x*y =  -5.297e+01        
\end{verbatim}}\noindent
to the data file ``data.out''.


\section{Structure of a C Program}
The syntax of a complete C program is given below:
{\small\begin{verbatim}
. . .
int main() 
{
 . . .
 return 0;
}
. . .
\end{verbatim}}\noindent
The meaning of the statements {\tt int main()} and {\tt return} will
become clear later on.  Preprocessor statements ({\em e.g.}, 
\verb{#define{ and \verb{#include{ statements) are conventionally placed {\em before}\/ the
{\tt int main()} statement. All executable statements must be placed
{\em between}\/ the {\tt int main()} and {\tt return} statements. Function
definitions (see later) are conventionally placed {\em after}\/ the
{\tt return} statement.

A simple C program ({\tt quadratic.c}) that calculates the real roots of a quadratic
equation using the well-known quadratic formula is listed below.
{\small\begin{verbatim}
/* quadratic.c */
/*
  Program to evaluate real roots of quadratic equation

     2
  a x  + b x + c = 0

  using quadratic formula

                     2  
  x = ( -b +/- sqrt(b - 4 a c) ) / (2 a)
*/

#include <stdio.h>
#include <math.h>

int main() 
{
  double a, b, c, d, x1, x2;

  /* Read input data */
  printf("\na = ");
  scanf("%lf", &a);
  printf("b = ");
  scanf("%lf", &b);
  printf("c = ");
  scanf("%lf", &c);

  /* Perform calculation */
  d = sqrt(b * b - 4. * a * c);
  x1 = (-b + d) / (2. * a);
  x2 = (-b - d) / (2. * a);

  /* Display output */
  printf("\nx1 = %12.3e   x2 = %12.3e\n", x1, x2);

  return 0;
}
\end{verbatim}}\noindent
Note the use of {\em comments}\/ (which are placed between the
delimiters \verb{/*{ and \verb{*/{) to first explain the function of the program
and  then identify the program's major sections. Note, also, the use
of {\em indentation}\/ to highlight the executable statements. When 
executed, the
above program produces the following output:
{\small\begin{verbatim}
a = 2
b = 4
c = 1

x1 =   -2.929e-01   x2 =   -1.707e+00
%
\end{verbatim}}\noindent 
Of course, the {\tt 2}, {\tt 4}, and {\tt 1} were entered by the user in response
to the programs's prompts. 

It is important to realize that there is more to writing a complete
computer program than simply arranging the individual declarations and statements
in the right order. Attention should also be given to making the program
and its output as {\em readable}\/ as possible, so that the program's function
 is
{\em immediately obvious}\/ to other people. This can be achieved by judicious use
of indentation and whitespace, as well as the inclusion of comments, and
the generation of clearly labeled output.  It is hoped that this
approach will be exemplified by the example programs used in this course. 

\section{Control Statements}
The C language includes a wide variety of powerful and flexible control
statements. The most useful of these are described in the following. 

The {\tt if-else} statement is used to carry out a logical test and
then take one of two possible actions, depending on whether the outcome
of the test is true or false. The {\tt else} portion of the statement
is optional. Thus, the simplest possible {\tt if-else} statement takes the form:
{\small\begin{verbatim}
if (expression) statement
\end{verbatim}}\noindent
The expression must be placed in parenthesis, as shown. In this form,
the  statement will only be executed if the  expression has a nonzero
value ({\em i.e.}, if {\tt expression} if true). If the  expression
has a value of zero ({\em i.e.}, if {\tt expression} is false) then
the  statement will be ignored. The statement can be either
simple or compound. 

The program {\tt quadratic.c}, listed previously, is incapable of
dealing correctly with cases where the roots are complex ({\em i.e.}, $b^2<4\,a\,c$),
or cases where $a=0$. It is good programming practice to test for situations
which fall outside the domain of validity of a program, and 
produce some sort of error message when these occur. An amended version of 
{\tt quadratic.c} which uses {\tt if-else} statements to
reject invalid input data is listed below.
{\small\begin{verbatim}
/* quadratic1.c */
/*
  Program to evaluate real roots of quadratic equation

     2
  a x  + b x + c = 0

  using quadratic formula

                     2  
  x = ( -b +/- sqrt(b - 4 a c) ) / (2 a)

  Program rejects cases where roots are complex
  or where a = 0.
*/

#include <stdio.h>
#include <math.h>
#include <stdlib.h>

int main() 
{
  double a, b, c, d, e, x1, x2;

  /* Read input data */
  printf("\na = ");
  scanf("%lf", &a);
  printf("b = ");
  scanf("%lf", &b);
  printf("c = ");
  scanf("%lf", &c);
 
  /* Test for complex roots */
  e = b * b - 4. * a * c;

  if (e < 0.) 
   {
    printf("\nError: roots are complex\n");
    exit(1);
   }

  /* Test for a = 0. */
  if (a == 0.) 
   {
    printf("\nError: a = 0.\n");
    exit(1);
   }

  /* Perform calculation */
  d = sqrt(e);
  x1 = (-b + d) / (2. * a);
  x2 = (-b - d) / (2. * a);

  /* Display output */
  printf("\nx1 = %12.3e   x2 = %12.3e\n", x1, x2);

  return 0;
}
\end{verbatim}}\noindent
Note the use of indentation to highlight statements which are conditionally
executed ({\em i.e.}, statements within an {\tt if-else} statement). 
The standard library function call {\tt exit(1)} (header file: {\tt stdlib.h}) causes the program
to abort with an error status. Execution of the above program for the
case of complex roots yields the following output:
{\small\begin{verbatim}
a = 4
b = 2
c = 6

Error: roots are complex
%
\end{verbatim}}\noindent

The general form of an {\tt if-else} statement, which includes the
{\tt else} clause, is
{\small\begin{verbatim}
if (expression)  statement 1  else  statement 2
\end{verbatim}}\noindent
If the expression has a non-zero value ({\em i.e.}, if {\tt expression}
is true) then {\tt statement\,1} is executed. Otherwise, {\tt statement\,2}
is executed. The program listed below is an extended version of the
previous program {\tt quadratic.c} which is capable of dealing
with complex roots. 
{\small\begin{verbatim}
/* quadratic2.c */
/*
  Program to evaluate all roots of quadratic equation

     2
  a x  + b x + c = 0

  using quadratic formula

                     2  
  x = ( -b +/- sqrt(b - 4 a c) ) / (2 a)

  Program rejects cases where a = 0.
*/

#include <stdio.h>
#include <math.h>
#include <stdlib.h>

int main() 
{
  double a, b, c, d, e, x1, x2;

  /* Read input data */
  printf("\na = ");
  scanf("%lf", &a);
  printf("b = ");
  scanf("%lf", &b);
  printf("c = ");
  scanf("%lf", &c);
 
  /* Test for a = 0. */
  if (a == 0.) 
   {
    printf("\nError: a = 0.\n");
    exit(1);
   }

  /* Perform calculation */
  e = b * b - 4. * a * c;

  if (e > 0.) // Test for real roots
   {  
    /* Case of real roots */
    d = sqrt(e);
    x1 = (-b + d) / (2. * a);
    x2 = (-b - d) / (2. * a);
    printf("\nx1 = %12.3e   x2 = %12.3e\n", x1, x2);
   } 
  else 
   {
    /* Case of complex roots */
    d = sqrt(-e);
    x1 = -b / (2. * a);
    x2 = d / (2. * a);
    printf("\nx1 = (%12.3e, %12.3e)   x2 = (%12.3e, %12.3e)\n", 
           x1, x2, x1, -x2);
   }
  return 0;
}
\end{verbatim}}\noindent
Note the use of an {\tt if-else} statement to deal with the two alternative cases 
of real and complex roots. Note also that the C compiler ignores
all characters on a line which occur after the {\tt //}  construct.\footnote{
Strictly speaking, this is a C++ extension to the C language. However, most
modern C compilers now accept this comment style.} Hence, this
construct can be used to comment individual lines in a program. The output
from the above program for the case of complex roots looks like:
{\small\begin{verbatim}
a = 9
b = 2
c = 2

x1 = (  -1.111e-01,    4.581e-01)   x2 = (  -1.111e-01,   -4.581e-01)
%
\end{verbatim}}

The {\tt while} statement is used to carry out looping operations,
in which a group of statements is executed repeatedly until some
condition is satisfied. The general form of a {\tt while} statement is
{\small\begin{verbatim}
while  (expression)  statement
\end{verbatim}}\noindent
The statement is executed repeatedly, as long as the expression
is nonzero ({\em i.e.}, as long as {\tt expression} is true). The
statement can be either simple or compound. Of course, the statement
 must include some
feature that eventually alters the value of the expression, thus
providing an escape mechanism from the loop. 

The program listed below  ({\tt iteration.c}) uses a {\tt while} statement to solve
an algebraic equation via iteration, as explained in the initial comments.
{\small\begin{verbatim}
/* iteration.c */
/* 
   Program to solve algebraic equation

    5      2 
   x  + a x  - b = 0

   by iteration. Easily shown that equation must have at least
   one real root. Coefficients a and b are supplied by user.

   Iteration scheme is as follows:

                    2  0.2
   x    =  ( b - a x  )
    n+1             n

   where x_n is nth iteration. User must supply initial guess for x.
   Iteration continues until relative change in x per iteration is 
   less than eps (user supplied) or until number of iterations exceeds 
   NITER. Program aborts if (b - a x*x) becomes negative. 
*/

#include <stdio.h>
#include <math.h>
#include <stdlib.h>

/* Set max. allowable no. of iterations */
#define NITER 30   

int main() 
{
  double a, b, eps, x, x0, dx = 1., d;
  int count = 0;

  /* Read input data */
  printf("\na = ");
  scanf("%lf", &a);
  printf("b = ");
  scanf("%lf", &b);
  printf("eps = ");
  scanf("%lf", &eps);

  /* Read initial guess for x */
  printf("\nInitial guess for x = ");  
  scanf("%lf", &x);
  x0 = x;

  while (dx > eps)  // Start iteration loop: test for convergence
   {
    /* Check for too many iterations */
    ++count;
    if (count > NITER) 
     {
      printf("\nError: no convergence\n");
      exit(1);
     }

    /* Reject complex roots */
    d = b - a * x * x;
    if (d < 0.)    
     {
      printf("Error: complex roots - try another initial guess\n");
      exit(1);
     }

    /* Perform iteration */
    x = pow(d, 0.2);
    dx = fabs( (x - x0) / x );
    x0 = x;

    /* Output data on iteration */
    printf("Iter = %3d   x = %8.4f   dx = %12.3e\n", count, x, dx);
   }
  return 0;
}
\end{verbatim}}\noindent
The typical output from the above program looks like:
{\small\begin{verbatim}
a = 3
b = 10
eps = 1.e-6

Initial guess for x = 1
Iter =   1   x =   1.4758   dx =    3.224e-01
Iter =   2   x =   1.2823   dx =    1.509e-01
Iter =   3   x =   1.3834   dx =    7.314e-02
Iter =   4   x =   1.3361   dx =    3.541e-02
Iter =   5   x =   1.3595   dx =    1.720e-02
Iter =   6   x =   1.3483   dx =    8.350e-03
Iter =   7   x =   1.3537   dx =    4.056e-03
Iter =   8   x =   1.3511   dx =    1.969e-03
Iter =   9   x =   1.3524   dx =    9.564e-04
Iter =  10   x =   1.3518   dx =    4.644e-04
Iter =  11   x =   1.3521   dx =    2.255e-04
Iter =  12   x =   1.3519   dx =    1.095e-04
Iter =  13   x =   1.3520   dx =    5.318e-05
Iter =  14   x =   1.3519   dx =    2.583e-05
Iter =  15   x =   1.3520   dx =    1.254e-05
Iter =  16   x =   1.3520   dx =    6.091e-06
Iter =  17   x =   1.3520   dx =    2.958e-06
Iter =  18   x =   1.3520   dx =    1.436e-06
Iter =  19   x =   1.3520   dx =    6.975e-07      
%    
\end{verbatim}}

When a loop is constructed using a {\tt while} statement, the test for
the continuation of the loop is carried out at the {\em beginning}\/ 
of each pass. Sometimes, however, it is desirable to have a loop where the
test for continuation takes place at the {\em end}\/ of each pass. This
can be accomplished by means of a {\tt do-while} statement. The
general form of a {\tt do-while} statement is
{\small\begin{verbatim}
do  statement  while  (expression);
\end{verbatim}}\noindent
The statement is executed repeatedly, as long as the expression is true. 
Note, however, that the statement is always executed at least once, since 
the test for repetition does not take place until the end of the first
pass through the loop. The statement can be either simple or compound, and
should, of course,  include some feature that eventually alters the value of
the expression.

The program listed below is a marginally improved version of the
previous program ({\tt iteration.c}) which uses a {\tt do-while} loop to test for
convergence at the end (as opposed to the beginning) of each iteration loop.
{\small\begin{verbatim}
/* iteration1.c */
/* 
   Program to solve algebraic equation

    5      2 
   x  + a x  - b = 0

   by iteration. Easily shown that equation must have at least
   one real root. Coefficients a and b are supplied by user.

   Iteration scheme is as follows:

                    2  0.2
   x    =  ( b - a x  )
    n+1             n

   where x_n is nth iteration. User must supply initial guess for x.
   Iteration continues until relative change in x per iteration is 
   less than eps (user supplied) or until number of iterations exceeds 
   NITER. Program aborts if (b - a x*x) becomes negative. 
*/

#include <stdio.h>
#include <math.h>
#include <stdlib.h>

/* Set max. allowable no. of iterations */
#define NITER 30  

int main() 
{
  double a, b, eps, x, x0, dx, d;
  int count = 0;

  /* Read input data */
  printf("\na = ");
  scanf("%lf", &a);
  printf("b = ");
  scanf("%lf", &b);
  printf("eps = ");
  scanf("%lf", &eps);

  /* Read initial guess for x */
  printf("\nInitial guess for x = ");  
  scanf("%lf", &x);
  x0 = x;

  do   // Start iteration loop
   {
    /* Check for too many iterations */
    ++count;
    if (count > NITER) 
     {
      printf("\nError: no convergence\n");
      exit(1);
     }

    /* Reject complex roots */
    d = b - a * x * x;
    if (d < 0.)     
     {
      printf("Error: complex roots - try another initial guess\n");
      exit(1);
     }

    /* Perform iteration */
    x = pow(d, 0.2);
    dx = fabs( (x - x0) / x );
    x0 = x;

    /* Output data on iteration */
    printf("Iter = %3d   x = %8.4f   dx = %12.3e\n", count, x, dx);

   } while (dx > eps);  // Test for convergence

  return 0;
}
\end{verbatim}}\noindent
The output from the above program is essentially identical to that from the
program {\tt iteration.c}. 

The {\tt while} and {\tt do-while} statements are particularly well
suited to looping situations in which the number of passes through the
loop {\em is not}\/ known in advance. Conversely,  situations in which
the number of passes through the loop {\em is}\/ known in advance are often
best dealt with using a {\tt for} statement. The general form of
a {\tt for} statement is
{\small\begin{verbatim}
for  (expression 1;  expression 2;  expression 3)  statement
\end{verbatim}}\noindent
where {\tt expression\,1} is used to initialize some parameter (called an
{\em index}) that controls the looping action, {\tt expression\,2} represents
a condition that must be true for the loop to continue execution, and
{\tt expression\,3} is used to alter the value of the parameter
initially assigned by {\tt expression\,1}. When a {\tt for} statement
is executed, {\tt expression\,2} is evaluated and tested at the
{\em beginning}\/ of each pass through the loop, whereas {\tt expression\,3}
is evaluated at the {\em end}\/ of each pass. 

The program listed below uses a {\tt for} statement to evaluate the
factorial of a non-negative integer.
{\small\begin{verbatim}
/* factorial.c */
/* 
   Program to evaluate factorial of non-negative
   integer n supplied by user.
*/

#include <stdio.h>
#include <stdlib.h>

int main()
{
  int n, count;
  double fact = 1.;
  
  /* Read in value of n */
  printf("\nn = ");
  scanf("%d", &n);

  /* Reject negative value of n */
  if (n < 0) 
   {
    printf("\nError: factorial of negative integer not defined\n");
    exit(1);
   }

  /* Calculate factorial */
  for (count = n; count > 0; --count) fact *= (double) count;

  /* Output result */
  printf("\nn = %5d    Factorial(n) = %12.3e\n", n, fact);

  return 0;
}
\end{verbatim}}\noindent
The typical output from the above program is shown below:
{\small\begin{verbatim}
n = 6
 
n =     6    Factorial(n) =    7.200e+02 
%
\end{verbatim}}

The statements which occur within {\tt if-else}, {\tt while}, {\tt do-while}, or
{\tt for} statements can themselves be control statements, giving rise to
the possibility of nested {\tt if-else} statements, conditionally executed loops, nested loops,
{\em etc.} When dealing with nested control statements, it is vital to adhere religiously to the
syntax rules described above, in order to avoid confusion.

\section{Functions}
We have seen that C supports the use of predefined library functions which are
used to carry out a large number of commonly occurring  tasks. However, C also allows programmers
to define their own functions. The use of programmer-defined functions permits a
large program to be broken down into a number of smaller, self-contained units.
In other words, a C program can be {\em modularized}\/ via the sensible use of programmer-defined functions.
In general, modular programs are far easier to write and debug than monolithic programs.
Furthermore, proper modularization allows other people to grasp the logical structure of a
program with far greater ease than would otherwise be the case.

A {\em function} is a self-contained program segment that carries out some specific,
well-defined task. Every C program consists of one or more functions. One of these
functions must be called {\tt main}. Execution of the program  always begins by
carrying out the instructions contained in {\tt main}. 
Note that if a program contains multiple functions then their definitions may appear in {\em any order}.
The same function can be accessed from several different places within a program. Once the
function has carried out its intended action, control is returned to the point from
which the function was accessed. Generally speaking, a function processes information passed to
it from the calling portion of the program, and returns a {\em single value}. Some functions,
however, accept information but do not return anything.

A function definition has two principal components: the {\em first line} (including the
{\em argument declarations}), and the so-called {\em body}\/ of the function.

The first line of a function takes the general form
{\small\begin{verbatim}
data-type  name(type 1  arg 1,  type 2  arg 2,  ...,  type n  arg n)
\end{verbatim}}\noindent
where {\tt data-type} represents the data type of the item that is returned by the
function, {\tt name} represents the name of the function, and {\tt type 1, type 2, \ldots, type n}
represent the data types of the arguments {\tt arg 1, arg 2, \ldots, arg n}. The allowable data types
for a function are:
{\small\begin{verbatim}
int       for a function which returns an integer value
double    for a function which returns an floating-point value
void      for a function which does not return any value
\end{verbatim}}\noindent
The allowable data types  for a function's arguments are {\tt int} and {\tt double}. Note that
the identifiers used to reference the arguments of a function are {\em local}, in the sense
that they are not recognized outside of the function. Thus, the argument names in a function
definition {\em need not}\/ be the same as those used in the
segments of the program from which the function was called. 
However, the corresponding data types of the arguments must always match.

The body of a function is a compound statement that defines the action to be taken by the
function. Like a regular compound statement, the body can contain expression statements, control
statements, other compound statements, {\em etc.} 
The body can even access other functions. In fact, it can even access
itself---this process is known as {\em recursion}. 
In addition, however, the body must
include one or more {\tt return} statements in order to return a value
to the calling portion of the program. 

A {\tt return} statement causes the program logic to return to the point in the program from which the
function was accessed.
The general form of a {\tt return} statement is:
{\small\begin{verbatim}
return  expression;
\end{verbatim}}\noindent
This statement causes the value of {\tt expression} to be  returned to the calling
part of the program. Of course, the data type of {\tt expression} should match the declared
data type of the function. For a {\tt void} function, which does not return any value, the
appropriate {\tt return} statement is simply:
{\small\begin{verbatim}
return;
\end{verbatim}}\noindent
A maximum of {\em one}\/ expression can be included in a {\tt return} statement. Thus, a function
can  return a maximum of one value to the calling part of the program.
However,  a function definition
can include multiple {\tt return} statements, each containing a different expression, which
are conditionally executed, depending on the program logic.

Note that, by convention, the {\tt main} function is of type {\tt int} and returns the integer
value {\tt 0} to the operating system, indicating the error-free termination of the
program. In its simplest form, the
{\tt main} function possesses {\em no} arguments. The library function call {\tt exit(1)}, employed
in previous example programs, causes the execution of a program to abort, returning the integer value
{\tt 1} to the operating system, which (by convention) indicates that the program terminated with an
error status. 

The program segment listed below shows how the previous program {\tt factorial.c} can be converted
into a function {\tt factorial(n)} which returns the factorial (in the form of a floating-point number) of
the non-negative integer {\tt n}:
{\small\begin{verbatim}
double factorial(int n) 
{
  /* 
     Function to evaluate factorial (in floating-point form)
     of non-negative integer n.
  */

  int count;
  double fact = 1.;

  /* Abort if n is negative integer */
  if (n < 0) 
   {
    printf("\nError: factorial of negative integer not defined\n");
    exit(1);
   }

  /* Calculate factorial */
  for (count = n; count > 0; --count) fact *= (double) count;

  /* Return value of factorial */
  return fact;      
}
\end{verbatim}}\noindent

A function can be accessed, or {\em called},  by specifying its name, followed by a list
of arguments enclosed in parentheses and separated by commas. If the function call
does not require any arguments then an empty pair of parentheses must follow the
name of the function. The function call may be part of a simple expression, such as
an assignment statement, or it may be one of the operands within a more complex
expression. The arguments appearing in a function call may be expressed as constants, single
variables, or more complex expressions. However, both the number and the types of the
arguments must match those in the function definition. 

The program listed below ({\tt printfact.c}) uses the function {\tt factorial()}, described above,
to print out the factorials of all the integers between 0 and 20:
{\small\begin{verbatim}
/* printfact.c */
/*
  Program to print factorials of all integers
  between 0 and 20
*/

#include <stdio.h>
#include <stdlib.h>

//%%%%%%%%%%%%%%%%%%%%%%%%%%%%%%%%%%%%%%%%%%%%%%%%%%%%%%%%%%%

double factorial(int n) 
{
  /* 
     Function to evaluate factorial (in floating-point form)
     of non-negative integer n.
  */

  int count;
  double fact = 1.;

  /* Abort if n is negative integer */
  if (n < 0) 
   {
    printf("\nError: factorial of negative integer not defined\n");
    exit(1);
   }

  /* Calculate factorial */
  for (count = n; count > 0; --count) fact *= (double) count;

  /* Return value of factorial */
  return fact;      
}

//%%%%%%%%%%%%%%%%%%%%%%%%%%%%%%%%%%%%%%%%%%%%%%%%%%%%%%%%%%%%%

int main()
{
  int j;

  /* Print factorials of all integers between 0 and 20 */
  for (j = 0; j <= 20; ++j) 
    printf("j = %3d    factorial(j) = %12.3e\n", j, factorial(j));

  return 0;
}
\end{verbatim}}\noindent
Note that the call to {\tt factorial()} takes place inside a complex expression
({\em i.e.}, a {\tt printf()} function call). Note also that the argument of
{\tt factorial()} has a different name (but the same data type) in the two sections
of the program ({\em i.e.}, in the {\tt main()} function and the {\tt factorial()} function). 
The output from the above program looks like:
{\small\begin{verbatim}
j =   0    factorial(j) =    1.000e+00
j =   1    factorial(j) =    1.000e+00
j =   2    factorial(j) =    2.000e+00
j =   3    factorial(j) =    6.000e+00
j =   4    factorial(j) =    2.400e+01
j =   5    factorial(j) =    1.200e+02
j =   6    factorial(j) =    7.200e+02
j =   7    factorial(j) =    5.040e+03
j =   8    factorial(j) =    4.032e+04
j =   9    factorial(j) =    3.629e+05
j =  10    factorial(j) =    3.629e+06
j =  11    factorial(j) =    3.992e+07
j =  12    factorial(j) =    4.790e+08
j =  13    factorial(j) =    6.227e+09
j =  14    factorial(j) =    8.718e+10
j =  15    factorial(j) =    1.308e+12
j =  16    factorial(j) =    2.092e+13
j =  17    factorial(j) =    3.557e+14
j =  18    factorial(j) =    6.402e+15
j =  19    factorial(j) =    1.216e+17
j =  20    factorial(j) =    2.433e+18      
%
\end{verbatim}}

Ideally, function definitions should always {\em precede}\/ the corresponding
function calls in a C program. This
requirement  can usually
be satisfied by judicious ordering of the various functions which make up a program, but, almost
inevitably, restricts the location of the {\tt main()} function to the {\em end}\/ of the program. Hence,
if the order of the two functions in the above program [{\em i.e.}, {\tt factorial()} and {\tt main()}] were
{\em swapped}\/  then an error message would be generated on compilation, since an attempt would be made
to call {\tt factorial()} prior to its definition.
Unfortunately, for the sake of logical clarity, most C programmers prefer to place the {\tt main()} function
at the {\em beginning} of their programs. After all, {\tt main()} is always the first part of
a program to be executed. In such situations, function calls [within {\tt main()}] are bound to
precede the corresponding function definitions: fortunately, 
however, compilation errors can be avoided by using a
construct known as a {\em function prototype}. 

Function prototypes are conventionally placed at the beginning of a program 
({\em i.e.}, before the {\tt main()}
function) and are used to inform the compiler of the name, data type, and number and data types
of the arguments, of all user-defined functions employed in the program. The general form of
a function prototype is
{\small\begin{verbatim}
data-type  name(type 1,  type 2,  ...,  type n);
\end{verbatim}}\noindent
where {\tt data-type} represents the data type of the item returned by the
referenced function, {\tt name} is the name of the function, and {\tt type 1, type 2,\ldots, type n}
are the data types of the arguments of the function. Note that it is not necessary to specify
the names of the arguments in a function prototype. Incidentally, the function prototypes for
predefined library functions are contained within the associated header files which must be
included at the beginning of every program which uses these functions.


The program listed below is a modified version of {\tt printfact.c} in which the {\tt main()} function
is the first function to be defined:
{\small\begin{verbatim}
/* printfact1.c */
/*
  Program to print factorials of all integers
  between 0 and 20
*/

#include <stdio.h>
#include <stdlib.h>

/* Prototype for function factorial() */
double factorial(int);    

int main()
{
  int j;

  /* Print factorials of all integers between 0 and 20 */
  for (j = 0; j <= 20; ++j) 
    printf("j = %3d    factorial(j) = %12.3e\n", j, factorial(j));

  return 0;
}

//%%%%%%%%%%%%%%%%%%%%%%%%%%%%%%%%%%%%%%%%%%%%%%%%%%%%%%%%%%%

double factorial(int n) 
{
  /* 
     Function to evaluate factorial (in floating-point form)
     of non-negative integer n.
  */

  int count;
  double fact = 1.;

  /* Abort if n is negative integer */
  if (n < 0) 
   {
    printf("\nError: factorial of negative integer not defined\n");
    exit(1);
   }

  /* Calculate factorial */
  for (count = n; count > 0; --count) fact *= (double) count;

  /* Return value of factorial */
  return fact;      
}
\end{verbatim}}\noindent
Note the presence of the function prototype for {\tt factorial()} prior to the definition
of  {\tt main()}. This is needed because the program calls {\tt factorial()} before this function
has been
defined. 
 The output from  the above program is identical to that from {\tt printfact.c}.

It is generally considered to be good programming practice to provide function prototypes for
{\em all}\/ user-defined functions accessed in a program, whether or not they are strictly
required by the compiler.\footnote{Function prototypes are a requirement for all
user-defined functions in C++ programs.} The reason for this is fairly simple. If we
provide a prototype for a given function then the compiler can carefully compare each
use of the function, within the program, with this prototype so as to determine whether
or not we are calling the function properly. In the absence of a prototype, an
incorrect call to a function ({\em e.g.}, using the wrong number of arguments, or
arguments of the wrong data type) can give rise to run-time errors which are difficult
to diagnose.

When a single value is passed to a function as an argument then the value of that argument
is simply {\em copied}\/ to the function. Thus, the  argument's value can subsequently
be altered within the function
but this {\em will  not}\/ affect its value in the calling routine. This
procedure for passing the value of an argument to a function is called {\em passing by value}.

Passing an argument by value has advantages and disadvantages. The advantages are that
it allows a single-valued argument to be written as an expression, rather than being restricted
to a single variable. Furthermore, in cases where  the argument is a variable, the value of
this variable is {\em protected}\/ from alterations which take place within the function.
The main disadvantage is that information cannot be transferred back to the calling
portion of the program via arguments. In other words, passing by value is a strictly
{\em one-way}\/ method of transferring information. The program listed below, which is another
modified version of {\tt printfact.c}, illustrates this point:
{\small\begin{verbatim}
/* printfact2.c */
/*
  Program to print factorials of all integers
  between 0 and 20
*/

#include <stdio.h>
#include <stdlib.h>

/* Prototype for function factorial() */
double factorial(int);    

int main() 
{
  int j;

  /* Print factorials of all integers between 0 and 20 */
  for (j = 0; j <= 20; ++j) 
    printf("j = %3d    factorial(j) = %12.3e\n", j, factorial(j));

  return 0;
}

//%%%%%%%%%%%%%%%%%%%%%%%%%%%%%%%%%%%%%%%%%%%%%%%%%%%%%%%%%%%

double factorial(int n) 
{
  /* 
     Function to evaluate factorial (in floating-point form)
     of non-negative integer n.
  */

  double fact = 1.;

  /* Abort if n is negative integer */
  if (n < 0) 
   {
    printf("\nError: factorial of negative integer not defined\n");
    exit(1);
   }

  /* Calculate factorial */
  for (; n > 0; --n) fact *= (double) n;

  /* Return value of factorial */
  return fact;      
}
\end{verbatim}}\noindent
Note that the function {\tt factorial()} has been modified such that its integer
argument {\tt n} is also used as the index  in a {\tt for} statement. Thus, the
value of this argument is {\em modified}\/ within the function. Nevertheless, the output of the
above program is identical to that from {\tt printfact.c}, since the modifications
to {\tt n} are {\em not}\/ passed back to the main part of the program. Note, incidentally, the use
of a null initialization expression in the {\tt for} statement appearing in 
{\tt factorial()}. 

\section{Pointers}
One of the main characteristics of a  scientific program is that {\em large amounts}\/ of
numerical information are exchanged between the various functions which make up
the program. It is generally most convenient to pass this information via the
{\em argument lists}, rather than the {\em names}, of the these functions. After all, only
one number can be passed via a function name, whereas scientific programs generally require far more
than one number to be passed during a function call. Hence, the functions employed in
scientific programs generally return no values via their names ({\em i.e.}, they tend to be of
data type {\tt void}) but possess large strings of arguments. There is one obvious
problem with this approach. Namely, a {\tt void} function which passes all of its arguments by
value is incapable of returning any information to the program segment from which it was called. 
Fortunately, there is a way of getting around this difficulty: we can pass the arguments
of a function by {\em reference}, rather than by {\em value}, using constructs known as {\em pointers}. 
This allows the  {\em two-way}\/ communication of information via arguments during function calls. 
Pointers are discussed in the following. 

Suppose that {\tt v} is a variable in a C program which represents some particular data item.
Of course, the program stores this data item at some particular location in the computer's memory. 
The data item can thus be accessed if we know its location, or {\em address}, in computer
memory. The address of {\tt v}'s memory location is determined by the expression \verb{&v{,
where \verb{&{ is a unary operator known as the {\em address operator}. 

Suppose that we assign the address of {\tt v} to another variable {\tt pv}. In other
words,
{\small\begin{verbatim}
pv = &v
\end{verbatim}}\noindent
This new variable is called a {\em pointer}\/ to {\tt v}, since it points to the location where
{\tt v} is stored in memory. Remember, however, that {\tt pv} represents {\tt v}'s address, and {\em not}\/
its value. 

The data item represented by {\tt v} ({\em i.e.}, the data item stored at {\tt v}'s
memory location) can be accessed via the expression \verb{*pv{, where \verb{*{ is a unary
operator, called the {\em indirection operator}, which only operates on pointer variables. 
Thus, \verb{*pv{ and {\tt v} both represent the same data item. Furthermore,
if we write \verb{pv = &v{ and \verb{u = *pv{ then both {\tt u} and {\tt v} represent
the same value. 

The simple program listed below illustrates some of the points made above:
{\small\begin{verbatim}
/* pointer.c */
/* 
   Simple illustration of the action of pointers
*/

#include <stdio.h>

main() 
{
  int u = 5;     
  int v;
  int *pu;      // Declare pointer to an integer variable
  int *pv;      // Declare pointer to an integer variable

  pu = &u;      // Assign address of u to pu
  v = *pu;      // Assign value of u to v
  pv = &v;      // Assign address of v to pv

  printf("\nu = %d  &u = %X  pu = %X  *pu = %d", u, &u, pu, *pu);
  printf("\nv = %d  &v = %X  pv = %X  *pv = %d\n", v, &v, pv, *pv);

  return 0;
}
\end{verbatim}}\noindent
Note that {\tt pu} is a pointer to {\tt u}, whereas {\tt pv} is a pointer to {\tt v}. Incidentally,
the conversion character {\tt X}, which appears in the control strings of the 
above {\tt printf()} 
function calls, indicates that the associated data item should be output as a
{\em hexadecimal number}---this is the conventional method of representing an address in computer memory.
Execution of the above program yields the following output:
{\small\begin{verbatim}
u = 5  &u = BFFFFA24  pu = BFFFFA24  *pu = 5
v = 5  &v = BFFFFA20  pv = BFFFFA20  *pv = 5         
%
\end{verbatim}}\noindent
In the first line, we see that {\tt u} represents the value {\tt 5}, as specified in its declaration
statement. The address of {\tt u} is determined automatically by the compiler to be
{\tt BFFFFA24} (hexadecimal). The pointer {\tt pu} is assigned this value. Finally, the
value to which {\tt pu} points is {\tt 5}, as expected. Similarly, the second line
shows that {\tt v} also represents the value {\tt 5}. This is as expected, since we assigned
the value \verb{*pu{ to {\tt v}. The address of {\tt v} is {\tt BFFFFA20}. Of course,
{\tt u} and {\tt v} have different addresses. 

The unary operators \verb{&{ and \verb{*{ are members of the same precedence group
as the other unary operators ({\em e.g.}, \verb{++{ and \verb{--{). The
address operator (\verb{&{) can only act upon operands which possess a unique
address, such as ordinary variables. Thus, the address operator {\em cannot}\/ act upon
arithmetic expression, such as \verb{2 * (u + v){. The indirection operator (\verb{*{)
can only act upon operands which are pointers.

Pointer variables, like all other variables, must be declared before they can appear in
executable statements. A pointer declaration takes the general form
{\small\begin{verbatim}
data-type  *ptvar;
\end{verbatim}}\noindent
where {\tt ptvar} is the name of the pointer variable, and {\tt data-type} is the
data type of the data item towards which the pointer points. Note that an asterisk must always
 precede the name of a pointer variable in a pointer declaration. 

Referring to Sect.~\ref{sio}, we can now appreciate that the mysterious asterisk
which appears in the declaration of an input/output stream, {\em e.g.},
{\small\begin{verbatim}
FILE  *stream;
\end{verbatim}}\noindent
is there because {\tt stream} is a pointer variable (pointing towards an
object of the special data type {\tt FILE}). In fact, {\tt stream} points
towards the beginning of the associated input/output stream in memory.

Pointers are often passed to a function as arguments. This allows data items
within the calling part of the program to be accessed by the function, altered within the
function, and then passed back to the calling portion of the program in altered form. This
use of pointers is referred to as passing arguments by {\em reference}, rather than by value.

When an argument is passed by value, the associated data item is simply copied to the function.
Thus, any alteration to the data item within the function is not passed back to the calling routine.
When an argument is passed by reference, however, the {\em address}\/ of the associated data item
is passed to the function. The contents of this address can be freely accessed by both
the function and the calling routine. Furthermore, any changes made to the data item stored at
this address are recognized by both the function and the calling routine. Thus, the use of a
pointer as an argument allows the {\em two-way}\/ communication of information between a function and
its calling routine. 

The program listed below, which is yet another modified version of {\tt printfact.c}, uses
a pointer to pass back information from a function to its calling routine:
{\small\begin{verbatim}
/* printfact3.c */
/*
  Program to print factorials of all integers
  between 0 and 20
*/

#include <stdio.h>
#include <stdlib.h>

/* Prototype for function factorial() */
void factorial(int, double *);    

int main() 
{
  int j;
  double fact;

  /* Print factorials of all integers between 0 and 20 */
  for (j = 0; j <= 20; ++j) 
   {
    factorial(j, &fact);
    printf("j = %3d    factorial(j) = %12.3e\n", j, fact);
   }
  return 0;
}

//%%%%%%%%%%%%%%%%%%%%%%%%%%%%%%%%%%%%%%%%%%%%%%%%%%%%%%%%%%%

void factorial(int n, double *fact) 
{
  /* 
     Function to evaluate factorial *fact (in floating-point form)
     of non-negative integer n.
  */

  *fact = 1.;

  /* Abort if n is negative integer */
  if (n < 0) 
   {
    printf("\nError: factorial of negative integer not defined\n");
    exit(1);
   }

  /* Calculate factorial */
  for (; n > 0; --n) *fact *= (double) n;

  return;      
}
\end{verbatim}}\noindent
The output from this program is again identical to that from {\tt printfact.c}. Note that
the function {\tt factorial()} has been modified such that there is no data item associated with its
 name ({\em i.e.}, the function is of data type {\tt void}). However, the argument list
of this function has been extended such that there are now {\em two}\/ arguments. As before, the
first argument is the value of the positive integer {\tt n} whose factorial is to be
evaluated by the function. The second argument, {\tt fact},  is a {\em pointer}\/ which passes back
the factorial of {\tt n} (in the form of a floating-point number) to the main
part of the program. Incidentally, the compiler knows that {\tt fact} is a pointer
because its name is proceeded by an {\em asterisk}\/ in the argument declaration for {\tt factorial()}. 
Of course, in the body of the function, reference is made to
\verb{*fact{ ({\em i.e.}, the value of the data item stored in the memory location
towards which {\tt fact} points) rather than \verb{fact{ ({\em i.e.}, the address of the
memory location towards which \verb{fact{ points). Note that a {\tt void} function, which
returns no value, can only
be called via
a statement consisting of the function name followed by a list of its arguments (in parentheses and
separated by commas). Thus, the function {\tt factorial()} is called in the main part of the
program via the statement
{\small\begin{verbatim}
factorial(j, &fact);
\end{verbatim}}\noindent
This statement passes the integer value {\tt j} to {\tt factorial()}, which, in turn, passes
back the value of the factorial of {\tt j} via its second argument. Note that since the second
argument is passed by reference, rather than by value, it is written \verb{&fact{ ({\em i.e.}, the address
of the memory location where the floating-point value {\tt fact} is stored)
rather than {\tt fact} ({\em i.e.}, the value of the floating-point variable {\tt fact}).
Note,  finally,
that the function prototype for {\tt factorial()} takes the form
{\small\begin{verbatim}
void factorial(int, double *);
\end{verbatim}}\noindent
Here, the {\em asterisk}\/ after {\tt double} indicates that the second argument is a {\em pointer} to
a floating-point data item.

We can now appreciate that the mysterious ampersands which must precede variable names in
{\tt scanf()} calls: {\em e.g.}, 
{\small\begin{verbatim}
scanf("%d %lf %lf", &k, &x, &y);
\end{verbatim}}\noindent
are not so mysterious, after all. {\tt scanf()} is a  function which returns data
to its calling routine via its arguments (excluding its first argument, which is a control string).
Hence, these arguments must be passed to {\tt scanf()} by {\em reference}, 
rather than by {\em value}, otherwise they would be unable
to pass information back to the calling routine. It follows that we must pass the {\em addresses}\/ of variables
({\em e.g.}, \verb{&k{) to {\tt scanf()}, rather than the {\em values}\/ of these variables ({\em e.g.}, \verb{k{).
Note that since the {\tt printf()} function does not return any information
to its calling routine via its arguments, there is no need to pass these
 arguments by reference---passing by value
is fine. This explains why there are no ampersands in the argument list of a {\tt printf()} function.

A pointer to a function can be passed to another function as an argument. This allows
one function to be transferred to another, as though the first function were a
variable. This is very useful in scientific programming. Imagine that we have a routine
which numerically integrates a general one-dimensional function. Ideally, we would like to use
this routine to integrate more than one specific function. We can achieve this by passing (to the routine) the
name of the function to be integrated as an argument. Thus, for example, we can use the {\em same}\/ routine
to integrate a polynomial,  a trigonometric function, or a logarithmic function. 

	Let us refer to the function whose name is passed as an argument as the
{\em guest function}. Likewise, the function to which this name is passed is called the
{\em host function}. A pointer to a guest function is identified in the host function definition by
an entry of the form
{\small\begin{verbatim}
data-type  (*function-name)(type 1,  type 2, ...)
\end{verbatim}}\noindent
in the host function's argument declaration.\footnote{The parenthesis around {\tt *function-name}
are very important:\\
{\tt data-type  *function-name(type 1,  type 2, ...)}
is interpreted by the compiler as a reference to a function which returns a pointer
to type {\tt data-type}, rather than a pointer to a function which returns type {\tt data-type}.}
 Here, {\tt data-type} is the data type of the
guest function, {\tt function-name} is the local name of the guest function in the host function
definition, and {\tt type 1, type 2, \ldots} are the data types of the guest function's arguments.
The pointer to the guest function also requires an entry of the form
{\small\begin{verbatim}
data-type  (*)(type 1,  type 2,  ...)
\end{verbatim}}\noindent
in the argument declaration of the  host function's prototype.
 The guest function
can be accessed within the host function definition by means of the indirection operator. To
achieve this, the indirection operator must precede the guest function name, and both the
indirection operator and the guest function name must be enclosed in parenthesis: {\em i.e.}, 
{\small\begin{verbatim}
(*function-name)(arg 1,  arg 2,  ...)
\end{verbatim}}\noindent
Here, {\tt arg 1, arg 2,\ldots} are the arguments passed to the guest function.
Finally, the name of a
guest function is passed to the host function, during a call to the latter function, via an
entry like
{\small\begin{verbatim}
function-name
\end{verbatim}}\noindent
in the host function's argument list. 

The program listed below is a rather silly example which illustrates the passing
of function names as arguments to another function:
{\small\begin{verbatim}
/* passfunction.c */
/*
  Program to illustrate the passing of function names as
  arguments to other functions via pointers
*/

#include <stdio.h>

/* Function prototype for host fun. */
void cube(double (*)(double), double, double *); 
                                                
double fun1(double);    // Function prototype for first guest function
double fun2(double);    // Function prototype for second guest function

int main() 
{
  double x, res1, res2;

  /* Input value of x */
  printf("\nx = ");
  scanf("%lf", &x);

  /* Evaluate cube of value of first guest function at x */
  cube(fun1, x, &res1);

  /* Evaluate cube of value of second guest function at x */
  cube(fun2, x, &res2);

  /* Output results */
  printf("\nx = %8.4f   res1 = %8.4f   res2 = %8.4f\n", x, res1, res2);
  
  return 0;
}

//%%%%%%%%%%%%%%%%%%%%%%%%%%%%%%%%%%%%%%%%%%%%%%%%%%%%%%%%%%%

void cube(double (*fun)(double), double x, double *result) 
{
  /*
    Host function: accepts name of floating-point guest function 
    with single floating-point argument as its first argument, 
    evaluates this function at x (the value of its second argument), 
    cubes the result, and returns final result via its third argument.
  */

  double y;

  y = (*fun)(x);        // Evaluate guest function at x
  *result = y * y * y;  // Cube value of guest function at x

  return;
}

//%%%%%%%%%%%%%%%%%%%%%%%%%%%%%%%%%%%%%%%%%%%%%%%%%%%%%%%%%%%

double fun1(double z) 
{
  /*
    First guest function
  */

  return 3.0 * z * z - z;
}

//%%%%%%%%%%%%%%%%%%%%%%%%%%%%%%%%%%%%%%%%%%%%%%%%%%%%%%%%%%%

double fun2(double z) 
{
  /*
    Second guest function
  */
  
  return 4.0 * z  - 5.0 * z * z * z;
}
\end{verbatim}}\noindent
In the above program, the function {\tt cube()} accepts the name of a guest
function (with one argument) as its first argument, evaluates this function
at {\tt x} (the value of its second argument, which is ultimately specified by the user), 
cubes the result, and then passes
the final result back to the main part of the program via its third argument (which,
of course, is a pointer). The two guest functions, {\tt fun1()} and {\tt fun2()}, whose names are
passed to {\tt cube()}, are both simple polynomials. The output from the
above program looks like:
{\small\begin{verbatim}
x = 2  

x =   2.0000   res1 = 1000.0000   res2 = -32768.0000
%
\end{verbatim}}

\section{Global Variables} 
We have seen that a general C program is basically a collection of functions. Furthermore, the
variables used by these functions are {\em local}\/ in scope: {\em i.e.,} a variable defined in
one function is not recognized in another. The main method of transferring data from one
function to another is via the argument lists in function calls. Arguments can be passed
in one of two different manners. When an argument is passed by {\em value}\/ then the value of
a local variable (or expression) in the calling routine is copied to a local variable in
the function which is called. When an argument is passed by {\em reference}\/ then a local variable
in the calling routine shares the same memory location as a local variable in the function
which is called: hence, a change in one variable is automatically mirrored in the other.
However, there is a {\em third}\/ method of transferring information from one function to another.
It is possible to define variables which are {\em global}\/ in extent: such variables are recognized by
{\em all}\/ the functions making up the program, and have the {\em same}\/ values in all of these functions.

The C compiler recognizes a variable as global, as opposed to local,  because its declaration
is located {\em outside} the scope of any of the functions making up the program. Of course,
a global variable can only be used in an executable statement {\em after} it has been declared. Hence,
the natural place to put global variable declaration statements is {\em before} any function definitions:
{\em i.e.}, right at the beginning of the program}. Global variables declarations can be used to
initialize such variables, in the regular manner. However, the initial values must be expressed as
constants, rather than expressions. Furthermore, the initial values are only assigned once, at the
beginning of the program.

The program listed below is yet another version of {\tt printfact.c}, in which
communication between the two sections of the program takes place entirely via
global variables:
{\small\begin{verbatim}
/* printfact4.c */
/*
  Program to print factorials of all integers
  between 0 and 20
*/

#include <stdio.h>
#include <stdlib.h>

/* Prototype for function factorial() */
void factorial();
   
/* Global variable declarations */
int j;             
double fact;

int main() 
{
  /* Print factorials of all integers between 0 and 20 */
  for (j = 0; j <= 20; ++j) 
   {
    factorial();
    printf("j = %3d    factorial(j) = %12.3e\n", j, fact);
   }
  return 0;
}

//%%%%%%%%%%%%%%%%%%%%%%%%%%%%%%%%%%%%%%%%%%%%%%%%%%%%%%%%%%%

void factorial() 
{
  /* 
     Function to evaluate factorial (in floating-point form)
     of non-negative integer j. Result stored in variable fact.
  */

  int count;

  /* Abort if j is negative integer */
  if (j < 0) 
   {
    printf("\nError: factorial of negative integer not defined\n");
    exit(1);
   }

  /* Calculate factorial */
  for (count = j, fact = 1.; count > 0; --count) fact *= (double) count;

  return;      
}
\end{verbatim}}\noindent
The output from the above program is identical to that from {\tt printfact.c}. Observe that
the global variable {\tt j} is used to pass the integer value whose factorial is to
be calculated from the main part of the program to the function {\tt factorial()}, whilst the 
global variable {\tt fact} is used to pass the calculated factorial back to the main part
of the program. Incidentally, note the use of multiple initialization statements (separated by commas)
in the {\tt for} statement appearing in {\tt factorial()}.  

Global variables should be used {\em sparingly} in scientific programs (or any other type of
program), since  there are inherent dangers in their employment. An alteration in the
value of a global variable within a given function is carried over into all  other
parts of the program. Unfortunately, such an alteration
 can sometimes happen inadvertently, as the side-effect
of some other action. Thus, there is the possibility of the value of a global variable
changing unexpectedly, resulting in a subtle programming error which can be {\em extremely difficult}
to track down, since the
offending line could be located {\em anywhere} in the program.
 Similar errors involving local variables are much easier to debug, since the scope of
local variables is far more limited than that of global variables. 

\section{Arrays}
Scientific programs very often deal with multiple data items possessing 
common characteristics. In such cases, it is often convenient to place the
data items in question into an {\em array},  so that they all share a common name ({\em e.g.}, {\tt x}). The
individual data items can be either integers or floating-point numbers. However, they
all must be of the {\em same}\/ data type. 

In C, an element  of an array ({\em i.e.}, an individual data item)
is referred to by specifying the array name followed by one or more {\em subscripts}, with each
subscript enclosed in square brackets. All subscripts must be nonnegative integers. 
Thus, in
an {\tt n}-element array called {\tt x}, the array elements are {\tt x[0], x[1], \ldots, x[n-1]}. 
Note that the first element of the array is {\tt x[0]} and {\em not}\/ {\tt x[1]}, as in other
programming languages. 

The number of subscripts determines the {\em dimensionality}\/ of an array. For example,
{\tt x[i]} refers to an element of a one-dimensional array, {\tt x}. Similarly,
{\tt y[i][j]} refers to an element of a two-dimensional array, {\tt y}, {\em etc.} 

Arrays are declared in much the same manner as ordinary variables, except that each
array name must be accompanied by a size specification (which specifies the number of elements).
For a one-dimensional array, the size is specified by a positive integer constant, enclosed in
square brackets. The generalization for multi-dimensional arrays is fairly obvious. Several
valid array declarations are shown below:
{\small\begin{verbatim}
int j[100];
double x[20];
double y[10][20];
\end{verbatim}}\noindent
Thus, {\tt j} is a {\tt 100}-element  integer array, {\tt x} is a {\tt 20}-element floating
point array, and {\tt y} is a {\tt 10x20} floating-point array. Note that
variable size array declarations, {\em e.g.},
{\small\begin{verbatim}
double a[n];
\end{verbatim}}\noindent
where {\tt n} is an integer variable, are {\em illegal}\/ in C.

It is sometimes convenient to define an array size in terms of a symbolic constant, rather
than a fixed integer quantity. This makes it easier to modify a program that
utilizes an array, since all references to the maximum array size can be altered by simply
changing the value of the symbolic constant. This approach is used in many of the example
programs employed  in this course.

Like an ordinary variable, an array can be either {\em local}\/ or {\em global}\/ in extent, depending on whether
the associated array declaration lies inside or outside, respectively, the scope of any of
the functions which constitute the program. 
Both local and global arrays can be initialized
via their declaration statements.\footnote{Prior to the ANSI standard, to which the GNU C
compiler adheres, local arrays could not be initialized via
their declaration statements.} For instance,
{\small\begin{verbatim}
int j[5] = {1, 3, 5, 7, 9};
\end{verbatim}}\noindent
declares {\tt j} to be a {\tt 5}-element integer array whose elements have the
initial values {\tt j[0]=1}, {\tt j[1]=3}, {\em etc.}

Single operations which involve entire arrays are {\em not}\/ permitted in C. Thus, if
{\tt x} and {\tt y} are similar arrays ({\em i.e.}, the same data type, dimensionality,
and size) then assignment operations, comparison operations, {\em etc.}\ involving
these two arrays must be carried out on an element by element basis. This is
usually accomplished within a loop (or within nested loops, for multi-dimensional arrays). 

The program listed below is a simple illustration of the use of arrays in C. The
program reads  a list of numbers, entered by the user, into a one-dimensional array, {\tt list},
and then calculates the average of these numbers. The program also calculates and outputs
the deviation of each number from the average. 
{\small\begin{verbatim}
/* average.c */
/*
  Program to calculate the average of n numbers and then
  compute the deviation of each number from the average

  Code adapted from "Programming with C", 2nd Edition, Byron Gottfreid,
  Schaum's Outline Series, (McGraw-Hill, New York NY, 1996) 
*/

#include <stdio.h>
#include <stdlib.h>

#define NMAX 100

int main() 
{
  int n, count;
  double avg, d, sum = 0.;
  double list[NMAX];

  /* Read in value for n */
  printf("\nHow many numbers do you want to average? ");
  scanf("%d", &n);
  
  /* Check that n is not too large or too small */
  if ((n > NMAX) || (n <= 0)) 
   {
    printf("\nError: invalid value for n\n");
    exit(1);
   }

  /* Read in the numbers and calculate their sum */
  for (count = 0; count < n; ++count) 
   {
    printf("i = %d  x = ", count + 1);
    scanf("%lf", &list[count]);
    sum += list[count];
   }

  /* Calculate and display the average */
  avg = sum / (double) n;
  printf("\nThe average is %5.2f\n\n", avg);

  /* Calculate and display the deviations about the average */
  for (count = 0; count < n; ++count) 
   {
    d = list[count] - avg;
    printf("i = %d  x = %5.2f  d = %5.2f\n", count + 1, list[count], d);
   }
  return 0;
}
\end{verbatim}}\noindent
Note the use of the symbolic constant {\tt NMAX} to specify the size of the array {\tt list}, and,
hence, the maximum number of values which can be averaged. The typical output
from the above program looks like:
{\small\begin{verbatim}
How many numbers do you want to average? 5
i = 1  x = 4.6
i = 2  x = -2.3
i = 3  x = 8.7
i = 4  x = 0.12
i = 5  x = -2.7
 
The average is  1.68
 
i = 1  x =  4.60  d =  2.92
i = 2  x = -2.30  d = -3.98
i = 3  x =  8.70  d =  7.02
i = 4  x =  0.12  d = -1.56
i = 5  x = -2.70  d = -4.38
%
\end{verbatim}}

It is important to realize that an
 array name in C is essentially a {\em pointer}\/ to the first element in that array.\footnote{An
array name is not exactly equivalent to a pointer, because a pointer can point to any
address in the computer memory---this address can even be changed---whereas an array name
is constrained to always point towards the address of its associated first data item.}
 Thus, if {\tt x}
is a one-dimensional array then the address of the first array element can be expressed as
either \verb{&x[0]{ or simply {\tt x}. Moreover, the address of the second array element can be written
as either \verb{&x[1]{ or {\tt (x+1)}. In general, the address of the {\tt (i+1)}th array element
can be expressed as either  \verb{&x[i]{ or {\tt (x+i)}. Incidentally, it should be
understood that {\tt (x+i)} is a rather special type of expression, since {\tt x} represents an
address, whereas {\tt i} represents an integer quantity. The expression {\tt (x+i)} actually specifies
the address of the array element which is {\tt i} memory locations offset from the address
of the first array element (C, of course, stores all elements of an array both contiguously and
in order in the computer memory). Hence, {\tt (x+i)} is a {\em symbolic representation}\/ of an
address, rather than an arithmetic expression. 

Since  \verb{&x[i]{ and {\tt (x+i)} both represent the address of the {\tt (i+1)}th element of
the array {\tt x}, it follows that {\tt x[i]} and \verb{*(x+i){ must both
represent the contents of that address ({\em i.e.}, the value of the {\tt (i+1)}th element). In
fact, the latter two terms are {\em completely interchangeable}\/ in C programs.

For the moment, let us concentrate on one-dimensional arrays.
An entire array can be passed to a function as an argument. To achieve this, the array name
must appear by itself, without brackets or subscripts, as an argument within the function call.
The corresponding argument in the function definition must be declared as an array. In order to
do this, the array name is written followed by an empty pair of square brackets. The size of the
array is not specified. In a function prototype, an array argument is specified by following the
data type of the argument by an empty pair of square brackets. 

Since, as we have seen, an array name is essentially a pointer, it is clear that when an array is passed
to a function  it is passed by {\em reference}, and not by {\em value}. Hence, if any of the
array elements are altered within the function then these alterations {\em are}\/ recognized
in the calling portion of the program. Likewise, if an array (rather than an individual array
element) appears in the argument list of a {\tt scanf()} function then it should {\em not}\/
be preceded by the address operator (\verb{&{), since an array name already is an address. 
The reason why arrays in C are always passed by reference is fairly obvious. In order
to pass an array by value, it is necessary to copy the value of {\em every}\/ element. On the
other hand, to pass an array by reference it is only necessary to pass the address of the first
element. Clearly, for large arrays, passing by reference is far more efficient than passing by value.

The program listed below is yet another version of {\tt printfact.c}, albeit a far more
efficient one than any of those listed previously. In this version, the
factorials of all the non-zero integers up to 20 are calculated in one fell swoop, by the
function {\tt factorial()}, 
using the recursion relation
\begin{equation}
(n+1)! = (n+1)\,n!
\end{equation}
The factorials are stored as elements of the array {\tt fact[]}, which is passed as an
argument from
{\tt factorial()} to the main part of the program.
{\small\begin{verbatim}
/* printfact5.c */
/*
  Program to print factorials of all integers
  between 0 and 20
*/

#include <stdio.h>

/* Function prototype for factorial() */
void factorial(double []);  

int main() 
{
  int j;
  double fact[21];           // Declaration of array fact[]

  /* Calculate factorials */
  factorial(fact);

  /* Output results */
  for (j = 0; j <= 20; ++j) 
    printf("j = %3d    factorial(j) = %12.3e\n", j, fact[j]);

  return 0;
}

//%%%%%%%%%%%%%%%%%%%%%%%%%%%%%%%%%%%%%%%%%%%%%%%%%%%%%%%%%%%

void factorial(double fact[]) 
{
  /*
    Function to calculate factorials of all integers
    between 0 and 20 (in form of floating-point
    numbers) via recursion formula

    (n+1)! = (n+1) n!

    Factorials returned in array fact[0..20]
  */

  int count;

  fact[0] = 1.;               // Set 0! = 1

  /* Calculate 1! through 20! via recursion */
  for (count = 0; count < 20; ++count) 
    fact[count+1] = (double)(count+1) * fact[count];

  return;
}
\end{verbatim}}\noindent
The output from  the above program is identical to that from {\tt printfact.c}. 

It is important to realize that there is {\em no array bound checking}\/ in C. If
an array {\tt x} is declared to have {\tt 100} elements then the compiler will reserve {\tt 100}
contiguous, appropriately sized, slots in computer memory on its behalf. The contents of these slots can be
accessed via expressions of the form {\tt x[i]}, where the integer {\tt i} should lie in the range
{\tt 0} to {\tt 99}. As we have seen, the compiler interprets {\tt x[i]} to mean the
contents of the memory slot which is {\tt i} slots along from the beginning of the array. 
Unfortunately, expressions such as {\tt x[100]} or {\tt x[1000]} are interpreted in a like manner,
leading the compiler to instruct the executable to
access memory slots which lie off the end of the memory block 
reserved for {\tt x}.   Obviously,  accessing elements of an array which do not
exist is going to produce some sort of error. Exactly what sort of error is very difficult to say---the 
program may crash, it may produce absurdly incorrect output, it may produce 
plausible but incorrect output, it may even produce correct output---it all
depends on exactly what information is being stored in the memory locations surrounding the block
of memory reserved for {\tt x}. This type of error can be {\em extremely difficult}\/ to debug,
since it may not be immediately apparent that something has gone wrong when the program is
executed. It is, therefore, the programmer's responsibility to ensure that {\em all}\/ references to array elements
lie within the declared bounds of the associated arrays.

 Let us now discuss multi-dimensional arrays in more detail. The elements of a multi-dimensional array
are stored contiguously in a block of computer memory. In scanning across this block, from
its start to its end, the order of storage is such than the last subscript of the array
varies {\em most rapidly}\/ whilst
the first  varies {\em least rapidly}. For instance, the elements of the
two-dimensional array {\tt x[2][2]} are stored in the order:
{\tt x[0][0], x[0][1], x[1][0], x[1][1]}. 
 The elements of a multi-dimensional array can only
be addressed if the program is
 explicitly told the size of the array in its second, third, {\em etc.}\ dimensions.
It is, therefore, not surprising to learn that when a multi-dimensional array is passed to a
function, as an argument, then the associated argument declaration within the function definition
{\em must}\/ include {\em explicit size declarations}\/ in all of the subscript positions {\em except
the first}. The same is true for a multi-dimensional array argument appearing in a function prototype.

\section{Character Strings}
The basic C character
data type is called {\tt char}.
A character string in C can be represented symbolically as an array of data type {\tt char}. 
For instance, the declaration
{\small\begin{verbatim}
char word[20] = "four";
\end{verbatim}}\noindent
initializes the 20-element character array {\tt word}, and then stores the
character string ``four'' in it. The resulting elements of {\tt word} are
as follows: 
{\small\begin{verbatim}
word[0] = 'f'  word[1] = 'o'  word[2] = 'u'  word[3] =  'r'  word[4] = '\0'
\end{verbatim}}\noindent
with the remaining elements undefined. Here, {\tt 'f'} represents the character ``f'', {\em etc.},
and \verb{'\0'{ represents the so-called {\em null character}\/ (ASCII code 0), which is used
in C to signal the termination of a character string. The null character is automatically
added to the end of any character string enclosed in double quotes.
Note that, since all character strings in C
must be terminated by the (invisible) null character, it takes a character array of
size at least {\tt n+1} to store an {\tt n}-letter string.

As with arrays of numbers, the name of a character array is essentially equivalent to
a pointer to the first element of that array.
 Hence, \verb{word[i]{ and \verb{*(word + i){ both refer to the
same character in the character array {\tt word}. Note, however, that the name of
a character array is not a true pointer, since the address to which it points
cannot be changed. Of course, we can always represent a character array using
a true pointer. Consider the declaration
{\small\begin{verbatim}
char *word = "four";
\end{verbatim}}\noindent
Here, {\tt word} is declared to be a pointer to a {\tt char} which points
towards the first element of the character string \verb{'f' 'o' 'u' 'r' '\0'{.
Unlike the name of a character array, a true pointer to a {\tt char} can be
redirected. Thus,
{\small\begin{verbatim}
char *word =  "four";
. . .
word = "five";
\end{verbatim}}\noindent
is legal,
whereas
{\small\begin{verbatim}
char word[20] =  "four";
. . .
word = "five";
\end{verbatim}}\noindent
is illegal. Note that, in the former example, the addresses of the first elements
of the strings ``four'' and ``five'' are probably different.
Of course, the contents of a character array can always be changed,
element by element---it is just the address of the first element which must
remain constant. Thus,
{\small\begin{verbatim}
char word[20] =  "four";
. . .
word[0] = 'f';
word[1] = 'i';
word[2] = 'v';
word[3] = 'e';
word[4] = '\0';
\end{verbatim}}\noindent
is perfectly legal.

Note, finally, that a character string can be printed via the {\tt printf()}
function by making use of a \verb{%s{ entry in its control string: {\em e.g.},
{\small\begin{verbatim}
printf("word = %s\n", word);
\end{verbatim}}\noindent
Here, the second argument, {\tt word}, can either be the name of a character array or a true pointer
to the first element of a character string.

\section{Multi-File Programs}
In a program consisting of many different functions, it is often convenient to place
each function in an individual file, and then use the {\tt  make} utility to compile each file separately and
link them together to produce an executable. 

There are a few common-sense rules
associated with multi-file programs. Since a given file is initially compiled {\em separately}\/ from
the rest of the program, all symbolic constants which appear in that file must be defined at its
start. Likewise, all referenced library functions 
must be accompanied by the appropriate references to header files. 
Also, any referenced user-defined functions   must have their prototypes at the start of the file. 
Finally, all global variables used in the file must be declared at its start.
This
usually means that definitions for common symbolic constants, header files for common library functions,
prototypes for common user-defined functions, and
declarations for common global variables will appear in {\em multiple}\/ files. Note that
a given global variable can only be initialized in {\em one}\/ of its  declaration statements,
which is regarded as the {\em true declaration}\/ of that variable
 [conventionally, the
true declaration appears in the file containing the function {\tt main()}]. Indeed, the 
other declarations, which we shall term {\em definitions}, 
 must be preceded by the keyword {\tt extern} to distinguish them from the
true declaration. 

As an example, let us take the program {\tt printfact4.c}, listed previously, and break it up into
multiple files, each containing a single function. The files in question are
called {\tt main.c} and {\tt factorial.c}.  The listings of
the two files which make up the program are as follows:
{\small\begin{verbatim}
/* main.c */
/*
  Program to print factorials of all integers
  between 0 and 20
*/

#include <stdio.h>

/* Prototype for fucntion factorial() */
void factorial();    

/* Global variable declarations */
int j;               
double fact;

int main() 
{
  /* Print factorials of all integers between 0 and 20 */
  for (j = 0; j <= 20; ++j)
    {
      factorial();
      printf("j = %3d    factorial(j) = %12.3e\n", j, fact);
    }
  return 0;
}
\end{verbatim}}\noindent
and
{\small\begin{verbatim}
/* factorial.c */
/* 
   Function to evaluate factorial (in floating point form)
   of non-negative integer j. Result stored in variable fact.
*/

#include <stdio.h>
#include <stdlib.h>

/* Global variable definitions */
extern int j;               
extern double fact;

void factorial() 
{
  int count;

  /* Abort if j is negative integer */
  if (j < 0) 
    {
      printf("\nError: factorial of negative integer not defined\n");
      exit(1);
    }

  /* Calculate factorial */
  for (count = j, fact = 1.; count > 0; --count) fact *= (double) count;

  return;      
}
\end{verbatim}}\noindent
Note that all library functions and user-defined functions referenced in each file are
declared (either via a header file or a function prototype) at the start of that file.
Note, also, the distinction between the global variable {\em declarations}\/ in the file
{\tt main.c} and the global variable {\em definitions}\/ in the file {\tt factorial.c}.


\section{Command Line Parameters}
The {\tt main()} function may optionally possess special arguments which allow parameters
to be passed to this function from the operating system. There are two such arguments, which
are conventionally called {\tt argc} and {\tt argv}. The former argument, {\tt argc}, is
an integer which is set to the number of parameters passed to {\tt main()}, whereas
the latter argument, {\tt argv}, is an array of pointers to character strings which
contain these parameters. In order to pass one or more parameters to a C program
when it is executed from the operating system, the parameters must follow
the program name on the command line: {\em e.g.},
{\small\begin{verbatim}
% program-name  parameter_1  parameter_2  parameter_3 ... parameter_n
\end{verbatim}}\noindent
The program name will be stored in the first item in {\tt argv}, followed by
each of the parameters. Hence, if the program name is followed by {\tt n} parameters
there will be {\tt n + 1} entries in {\tt argv}, ranging from {\tt argv[0]} to
{\tt argv[n]}. Furthermore, {\tt argc} will be automatically set equal to  {\tt n + 1}. 

The program listed below is a simple illustration of the use of command line
parameters: it simply echoes all of the parameters passed to it.
{\small\begin{verbatim}
/* repeat.c */
/*
  Program to read and echo data from command line
*/

int main(int argc, char *argv[])
{
  int i;

  for (i = 1; i < argc; i++) printf("%s ", argv[i]);
  printf("\n");

  return 0;
}
\end{verbatim}}\noindent
Assuming that the executable is called {\tt repeat}, the typical output from this
program is as follows:
{\small\begin{verbatim}
% repeat The quick brown fox jumped over the lazy hounds
The quick brown fox jumped over the lazy hounds
%
\end{verbatim}}\noindent

Suppose that one or more of the parameters passed to a given program are {\em numbers}.
As we have seen, these numbers are actually passed as character strings.
Hence, before they can be employed in calculations, they must be
converted into either type {\tt int} or type {\tt double}. This can be
achieved via the use of the functions {\tt atoi()} and {\tt atof()} (the
appropriate header file for these functions is {\tt stdlib.h}). Thus,
{\tt int atoi(char *ptr)} converts a string pointed to by {\tt ptr} into an
{\tt int}, whereas {\tt double atof(char *ptr)} converts a string pointed to by {\tt ptr} into an
{\tt double}. The program listed below illustrates the use of the {\tt atof()} function:
it reads in a number passed as a command line parameter, interprets it as a temperature in
degrees Fahrenheit, converts it to degrees Celsius, and then prints out the result.
{\small\begin{verbatim}
/* ftoc.c */
/*
  Program to convert temperature in Fahrenheit input
  on command line to temperature in Celsius
*/

#include <stdlib.h>
#include <stdio.h>

int main(int argc, char *argv[])
{
  double deg_f, deg_c;

  /* If no parameter passed to program print error
     message and exit */
  if (argc < 2)
    {
      printf("Usage: ftoc temperature\n");
      exit(1); 
    }

  /* Convert first command line parameter to double */
  deg_f = atof(argv[1]); 
  /* Convert from Fahrenheit to Celsius */
  deg_c = (5. / 9.) * (deg_f - 32.);

  printf("%f degrees Fahrenheit equals %f degrees Celsius\n", 
         deg_f, deg_c);

  return 0;
}
\end{verbatim}}\noindent
Assuming that the executable is called {\tt ftoc}, the typical output from this
program is as follows:
{\small\begin{verbatim}
% ftoc 98
98.000000 degrees Fahrenheit equals 36.666667 degrees Celsius
%
\end{verbatim}}\noindent

\section{Timing}
The header file {\tt time.h} defines a number of library functions which can be
used to assess how much CPU time a C program consumes during execution. The simplest
such function is called {\tt clock()}. A call to this function, with no argument,
will return the amount of CPU time used so far by the calling program. The time
is returned in a special data type, \verb{clock_t{, defined in {\tt time.h}. 
This
time must be divided by \verb{CLOCKS_PER_SEC{, also defined in {\tt time.h}, in order
to covert it into seconds. The ability to measure how much CPU time a given
code consumes is useful in scientific programming: {\em e.g.}, because it allows the effectiveness
of the various available compiler optimization flags to be determined. Optimization
usually (but not always!) speeds up the execution of a program.
However, over aggressive
optimization  can
often slow a program down again.

The program listed below illustrates the
simple use of the {\tt clock()} function. The program compares the CPU
time required to raise a {\tt double} to the fourth power via a direct calculation
and via a call to the {\tt pow()} function. Actually, both operations are performed
a million times and the elapsed CPU time is then divided by a million. 
{\small\begin{verbatim}
/* timing.c */
/* 
  Program to test operation of clock() function
*/

#include <time.h>
#include <math.h>
#define N_LOOP 1000000

int main()
{
  int i;
  double a = 11234567890123456.0, b;
  clock_t time_1, time_2;

  time_1 = clock();
  for (i = 0; i < N_LOOP; i++) b = a * a * a * a;
  time_2 = clock();
  printf ("CPU time needed to evaluate a*a*a*a:    %f microsecs\n",
          (double) (time_2 - time_1) / (double) CLOCKS_PER_SEC);

  time_1 = clock();
  for (i = 0; i < N_LOOP; i++) b = pow(a, 4.);
  time_2 = clock();
  printf ("CPU time needed to evaluate pow(a, 4.): %f microsecs\n",
          (double) (time_2 - time_1) / (double) CLOCKS_PER_SEC);

  return 0;
}
\end{verbatim}}\noindent
The typical output from this program is as follows:
{\small\begin{verbatim} 
CPU time needed to evaluate a*a*a*a:    0.190000 microsecs
CPU time needed to evaluate pow(a, 4.): 1.150000 microsecs
%
\end{verbatim}}\noindent
Clearly, evaluating a fourth power using the {\tt pow()} function is a
lot more expensive than the direct calculation. Hence, as has already
been mentioned, the {\tt pow()} function should {\em  not} be used to raise
floating point quantities to small integer powers.

\section{Random Numbers}
A large class of scientific calculations ({\em e.g.}, so-called Monte Carlo
calculations) require the use of random variables.   A call to the
{\tt rand()} function (header file {\tt stdlib.h}), with no arguments, returns a fairly good approximation to
a random integer in the range {\tt 0} to \verb{RAND_MAX{ (defined in {\tt stdlib.h}). The
{\tt srand()} function sets its argument, which is of type {\tt int}, as the
seed for a new sequence of numbers to be returned by {\tt rand()}. These
sequences are {\em repeatable}\/ by calling {\tt srand()} with the same seed value.
If no seed value is provided, the {\tt rand()} function is automatically 
seeded with the value  {\tt 1}.     It is common practice in C programming to
seed the random number generator with the number of seconds elapsed since
00:00:00 UTC, January 1st, 1970. This number is returned, as an integer, via
a call to the {\tt time (NULL)} function (header file: {\tt <time.h>}). Seeding
the generator in this manner ensures that a different set of random numbers is
generated automatically each time the program is run.

The program listed below illustrates the use of the {\tt rand()} function to construct a
pseudo-random variable, {\tt x}, which is uniformly distributed in the range {\tt 0} to {\tt 1}.
The program calculates $10^7$ values of {\tt x}, and then evaluates the mean and
variance of these values.
{\small\begin{verbatim} 
/* random.c */
/*
  Program to test operation of rand() function
*/

#include <stdlib.h>
#include <stdio.h>
#include <time.h>
#define N_MAX 10000000

int main()
{
  int i, seed;
  double sum_0, sum_1, mean, var, x;

  /* Seed random number generator */
  seed = time(NULL);
  srand(seed);

  /* Calculate mean and variance of x: random number uniformly
     distributed in range 0 to 1 */
  for (i = 1, sum_0 = 0., sum_1 = 0.; i <= N_MAX; i++)
    {
      x = (double) rand() / (double) RAND_MAX;

      sum_0 += x;
      sum_1 += (x - 0.5) * (x - 0.5);
    }
  mean = sum_0 / (double) N_MAX;
  var = sum_1 / (double) N_MAX;

  printf("mean(x) = %12.10f  var(x) = %12.10f\n", mean, var);

  return 0;
}
\end{verbatim}}\noindent
The typical output from this program is as follows:
{\small\begin{verbatim} 
mean(x) = 0.5000335261  var(x) = 0.0833193874    
%
\end{verbatim}}\noindent
As is easily demonstrated, the theoretical mean and variance of {\tt x}
are $1/2$ and $1/12$, respectively. It can be seen that the values returned
by the program agree with these theoretical values to five decimal
places, which is all that can be expected with only $10^7$ calls.


\section{C++ Extensions to C}
In this subsection, we shall briefly discuss some of the useful, non-object-orientated extensions
to C introduced in the C++ language. Files containing source code which incorporates C++ elements should
be distinguished from files containing plain C code via the extension {\tt .cpp}. 

In C, all local variables within a function must be declared {\em  before}\/ the first executable statement. In
C++, this restriction is relaxed: a local variable can be declared (almost) {\em anywhere}\/ in a function, provided
that this declaration occurs prior to the variable's first use. The following code
snippet illustrates the use of this new feature:
{\small\begin{verbatim} 
. . .
for (int i = 0; i < MAX; i++)
 {
  . . .
 }
. . .
\end{verbatim}}\noindent
Observe that the index {\tt i} of the {\tt for} loop is now declared at the start of the loop, instead
of at the start of the function containing the loop. This is far more convenient, and also makes the
code easier to read (since we no longer have to skip back to the declarations at the start
of the function to check that {\tt i} is an {\tt int}). 
Note, however, that the variable {\tt i} is only defined over the extent of the loop ({\em i.e.},
between the curly braces). Any attempt to reference {\tt i} outside the loop will result in a
compilation error. In general, when a variable is declared in C++ its {\em scope}\/  
({\em i.e.}, range of definition) extends from its declaration to the closing curly brace which
terminates the current program block. Program blocks are functions, loops,  conditionally
executed compound statements, {\em etc.}, and are  delineated by curly braces.
There are a number of restrictions to this new method of variable declaration.
 Variables {\em cannot}\/ be declared
within conditional statements, in the second or third expressions of {\tt for} loops, or
in function calls.

In C, we have seen that  in order to pass an argument to a function in such a
manner that changes made to this argument within the function are passed back to the
calling routine, we must actually pass a {\em pointer} to the argument. This procedure, which is
called passing by {\em reference}, 
is illustrated in the code snippet listed below:
{\small\begin{verbatim}
. . .
void square(double, double *);
. . . 
int main()
{
  . . .
  double arg, res;
  square(arg, &res);
  . . .
  return 0;
}
. . .
void square(double x, double *y)
{
  *y = x * x;
  return;
}
\end{verbatim}}\noindent
Here, the second argument to {\tt square()} is returned to {\tt main()}  in modified form.
This argument must, therefore, be passed as a pointer. Consequently, we must write \verb}&res}, rather that
{\tt res}, when calling {\tt square()}, and we must refer to the argument as
\verb}*y}, rather than {\tt y}, in the function itself. 
After a while, all these ampersands and asterisks
can become a little tedious! C++ introduces a new method of passing by reference which somewhat
less involved. This new method is illustrated in the following:
{\small\begin{verbatim}
. . .
void square(double, double &);
. . . 
int main()
{
  . . .
  double arg, res;
  square(arg, res);
  . . .
  return 0;
}
. . .
void square(double x, double &y)
{
  y = x * x;
  return;
}
\end{verbatim}}\noindent
Here, the second argument to  {\tt square()} is again passed by reference. However, this is now indicated
by prepending an ampersand to the variable name in the function declaration. A corresponding
ampersand appears in the associated function prototype. Note that we do not need to
explicitly pass a pointer to the second argument when calling {\tt square()} (this is done behind the
scenes): {\em i.e.}, we write {\tt res}, rather than \verb{&res{, when calling {\tt square()}.
 Likewise, we do not have to explicitly deference the argument in the
function itself (this is also done behind the scenes): {\em i.e.}, we refer to the
argument by its regular local name, {\tt y},  as opposed to \verb{*y{, within the function.


Functions are used in C programs to avoid having to repeat the same block of code
 in many different places in the source. The use of functions also renders a code easier to
read and maintain. However, there is a price to pay for the convenience of functions.
When a regular function is called in an executable, the program jumps to the block of memory in which the
compiled function code is stored, and then jumps back to its original
position in memory space when the function returns. Unfortunately, the large jumps in memory
space associated with a function call take up a non-negligible amount of CPU time. 
Indeed, the
overhead associated with making function calls often discourages scientific programmers
from writing small functions, even when it may be desirable to do so. C++ provides a
way out of this dilemma, via the use of the new keyword {\tt inline}. An
{\em inline function}\/ looks like a normal function when it is used, but is compiled
in a different manner. Calling an inline function from several different locations in a
code does not result in multiple calls to a single function. Instead, the code for the
inline function is inserted into the program code by the compiler wherever the function is used.

Inline functions are only useful for {\em small}\/ functions. The disadvantage of 
inserting the code for a large function multiple times into the code for a
typical program  easily outweighs
the small gain in performance obtained by the elimination of standard function calls.
The break-even point for inline functions is usually about {\em three}\/ executable lines.

To inline a function, a programmer adds the keyword {\tt inline} at the
start of the function's definition. For example:
{\small\begin{verbatim}
inline double square(double x)
{
  return x * x;
}
\end{verbatim}}\noindent

Because the body of an inline function must be known before the
compiler can insert it
into the program code, wherever the function is used, we must define such a function prior
to its first use---a prototype declaration is not enough. It is common practice to define inline
functions at the same location in source code files that the prototypes for regular ({\em i.e.}, outline)
functions are  placed.


Variable size array declarations of the form
{\small\begin{verbatim}
void function(n) 
{
  int n;
  double x[n];
  . . .
}
\end{verbatim}}\noindent
are {\em illegal}\/ in C, which is extremely inconvenient. In C++, such declarations
are enabled via the use of the new  keywords {\tt new} and {\tt delete}.
Thus, the C++ implementation of the above code snippet takes the form:
{\small\begin{verbatim}
void function(n) 
{
  . . .
  int n;
  double *x = new double[n];
  . . .
  x[i] = . . .
  . . .
  . . .
  delete x[];
  . . .
}
\end{verbatim}}\noindent
Note that {\tt x} is actually declared as a pointer, rather than a standard array. The
declaration {\tt new double[n]} reserves a memory block which is just large enough to store {\tt n} doubles,
and then returns the address of the start of this block. The line
{\tt delete x[]} frees up the block of memory associated with the array {\tt x} when
it is no longer needed (note that this is not done automatically).
Unfortunately, the {\tt new} and {\tt delete} keywords cannot
be used to make variable size multi-dimensional arrays.

\section{Complex Numbers}
As we have already mentioned, the C language definition does not include
complex arithmetic---presumably because the square root of minus one is not
a concept which crops up very often in systems programming! Fortunately,
this rather serious deficiency---at least, as far as the scientific programmer
is concerned---is remedied in C++. The program listed below illustrates the
use of the C++ complex class (header file {\tt complex.h}) to perform complex
arithmetic using doubles:
{\small\begin{verbatim}
/* complex.cpp */
/*
  Program to test out C++ complex class 
*/

#include <complex.h>
#include <stdio.h>

/* Define complex double data type */
typedef complex<double> dcomp; 

int main()
{
  dcomp i, a, b, c, d, e, p, q, r; // Declare complex double variables
  double x, y;

  /* Set complex double variable equal to complex double constant */
  i = dcomp (0., 1.); 
  printf("\ni = (%6.4f, %6.4f)\n", i);

  /* Test arithmetic operations with complex double variables */
  a = i * i;
  b = 1. / i;  
  printf("\ni*i = (%6.4f, %6.4f)\n", a);
  printf("1/i = (%6.4f, %6.4f)\n", b);

  /* Test mathematical functions using complex double variables */
  c = sqrt(i);
  d = sin(i);
  e = pow(i, 0.25); 
  printf("\nsqrt(i) = (%6.4f, %6.4f)\n", c);
  printf("sin(i) = (%6.4f, %6.4f)\n", d);
  printf("i^0.25 = (%6.4f, %6.4f)\n", e);

  /* Test complex operations */
  p = conj(i);
  q = real(i);
  r = imag(i);
  printf("\nconj(i) = (%6.4f, %6.4f)\n", p);
  printf("real(i) = %6.4f\n", q);
  printf("imag(i) = %6.4f\n", r);

  return 0;
}
\end{verbatim}}\noindent
The typical output from this program is as follows:
{\small\begin{verbatim} 
i = (0.0000, 1.0000)
 
i*i = (-1.0000, 0.0000)
1/i = (0.0000, -1.0000)
 
sqrt(i) = (0.7071, 0.7071)
sin(i) = (0.0000, 1.1752)
i^0.25 = (0.9239, 0.3827)
 
conj(i) = (0.0000, -1.0000)
real(i) = 0.0000
imag(i) = 1.0000
%                              
\end{verbatim}}\noindent
The program first of all defines the complex double type {\tt dcomp}. Variables
of this type are then declared, set equal to complex constants, employed in arithmetic
expressions, used as the arguments of mathematical functions, {\em etc.}, in much
the same manner that we would perform similar operations in C with variables
of type {\tt double}. Note the special functions {\tt conj()}, {\tt real()},
and {\tt imag()}, which take the complex conjugate of, find the real part of,
and find the imaginary part of a complex variable, respectively.

\section{Variable Size Multi-Dimensional Arrays}\label{blitz}
Multi-dimensional arrays crop up in a wide variety of different applications
in scientific programming.
Moreover, it is very common for the sizes of the arrays employed in a 
scientific code to vary from run to run,
depending on the particular values of the input parameters. 
For instance, in a standard fluid code the array sizes depend on the number of grid points, which, in turn,
depends on the requested accuracy. Thus, a crucial test of the suitability of a programming language
for scientific purposes is its ability to deal with variable size, multi-dimensional arrays in a
convenient manner. Unfortunately, C fails this test rather badly, since variable
size matrix declarations of the form
{\small\begin{verbatim}
void function(a, m, n) 
{
  int m, n;
  double a[m][n];
  . . .
\end{verbatim}}\noindent
are {\em illegal}\/ (unlike in FORTRAN 77, where variable size matrix declarations are
perfectly acceptable). Indeed, this unfortunate (and quite unnecessary!) omission in the C
language definition is often put forward as a reason why C should {\em not}\/ be used for scientific
purposes. Fortunately, we can get around the absence of variable size multi-dimensional arrays
in C by making use of a freely available C++ package called the Blitz++ library---see {\tt http://www.oonumerics.org/blitz/}.

The program listed below illustrates the use of the Blitz++ library. The program adds together two matrices
whose dimensions and elements are input by the user, and then prints out the result. 
{\small\begin{verbatim}
/* addmatrix.c */
/*
  Program to add two variable dimension matrices input by user
*/

#include <stdio.h>
#include <stdlib.h>
#include <blitz/array.h>

using namespace blitz;

/* Function prototypes */
void readin(Array<double,2>);
void writeout(Array<double,2>);
void addmatrices(Array<double,2>, Array<double,2>, Array<double,2>);

int main() 
{

  int n, m;
  
  /* Input number of rows and columns */
  printf("\nPlease input number of rows, n, and number of columns, m: ");
  scanf("%d %d", &n, &m);

  /* Check that n, m are positive integers */
  if (n <= 0 || m <= 0) 
    {
      printf("\nError: invalid values for n and/or m\n");
      exit(1);
    }

  /* Array declarations */
  Array<double,2> A(n, m), B(n, m), C(n, m);
     
  /* Read in elements of A, row by row */
  printf("\nReading in elements of A:\n");
  readin(A);

  /* Read in elements of B, row by row */
  printf("\nReading in elements of B:\n");
  readin(B);

  /* Write out elements of A, row by row */
  printf("\nWriting out elements of A:\n");
  writeout(A);

  /* Write out elements of B, row by row */
  printf("\n\nWriting out elements of B:\n");
  writeout(B);

  /* Add matrices A and B */
  addmatrices(A, B, C);

  /* Write out matrix C = A + B, row by row */
  printf("\n\nWriting out elements of C = A + B:\n");
  writeout(C);
  printf("\n");

  return 0;
}

//%%%%%%%%%%%%%%%%%%%%%%%%%%%%%%%%%%%%%%%%%%%%%%%%%%%%%%%%%%%

/*
  Read in elements of matrix M, row by row
*/  

void readin(Array<double,2> M) 
{
  int n = M.extent(0);
  int m = M.extent(1);

  for (int i = 0; i < n; i++) 
    {
      printf("\nRow %d: ", i + 1);
      for (int j = 0; j < m; j++) scanf("%lf", &M(i, j)); 
    }
  return;
}

//%%%%%%%%%%%%%%%%%%%%%%%%%%%%%%%%%%%%%%%%%%%%%%%%%%%%%%%%%%%
  
/*
  Write out elements of matrix M, row by row
*/  

void writeout(Array<double,2> M) 
{ 
  int n = M.extent(0);
  int m = M.extent(1);

  for (int i = 0; i < n; i++)
    {
      printf("\nRow %d: ", i + 1);
      for (int j = 0; j < m; j++) printf("%7.2f ", M(i, j)); 
    }
  return;
}

//%%%%%%%%%%%%%%%%%%%%%%%%%%%%%%%%%%%%%%%%%%%%%%%%%%%%%%%%%%%
 
/*
    Add matrices M and N and store result in matrix P
*/

void addmatrices(Array<double,2> M, Array<double,2> N, Array<double,2> P) 
{ 
  int n = M.extent(0);
  int m = M.extent(1);
  
  for (int i = 0; i < n; i++) 
    for (int j = 0; j < m; j++) 
      P(i, j) = M(i, j) + N(i, j); 

   return;
}
\end{verbatim}}\noindent
Typical output from the program looks like
{\small\begin{verbatim}

Please input number of rows, n, and number of columns, m: 3 3

Reading in elements of A:

Row 1: 1 4 5

Row 2: 6 7 8

Row 3: 3 6 0

Reading in elements of B:

Row 1: 1 6 0

Row 2: 4 7 8

Row 3: 4 8 8

Writing out elements of A:

Row 1:    1.00    4.00    5.00 
Row 2:    6.00    7.00    8.00 
Row 3:    3.00    6.00    0.00 

Writing out elements of B:

Row 1:    1.00    6.00    0.00 
Row 2:    4.00    7.00    8.00 
Row 3:    4.00    8.00    8.00 

Writing out elements of C = A + B:

Row 1:    2.00   10.00    5.00 
Row 2:   10.00   14.00   16.00 
Row 3:    7.00   14.00    8.00 

\end{verbatim}}\noindent
The header file for the Blitz++ library is called {\tt blitz/array.h}. Moreover, any program file which makes use of
Blitz++ must include the cryptic line {\tt using namespace blitz;} before the first call, otherwise compilation
errors will ensue. The declaration {\tt Array<double,2> A(n, m)} declares a two-dimensional {\tt n}
by {\tt m} array of doubles.
The generalization to an array of integers, or a higher dimension array, is fairly obvious.
Note that the {\tt i}, {\tt j} element of matrix {\tt A} is refered to simply as {\tt A(i,j)}.
 The
function call {\tt M.extent(0)} returns the size of array {\tt M} in its first dimension. Likewise,
{\tt M.extent(1)} returns the size of array {\tt M} in its second dimension, {\em etc.} More details of the operation
of Blitz++ can be found by reading the extensive documentation which accompanies this library.
Unfortunately, the Blitz++ library slows down compilation considerably since it makes use of
some very advanced templating features of the C++ language. 

\section{The CAM Graphics Class}
There are a myriad of useful, prewritten C++ classes which are freely available on the web---for
more details see {\tt http://www.trumphurst.com/cpplibsx.html}. In this Subsection,
we shall discuss just one of these---namely, the {\em CAM graphics class},\footnote{
The CAM graphics class is copyrighted to its author,  Prof.\ Chris Anderson, Department
of Mathematics, UCLA, 1998.}  whose
purpose is to enable a C++ program to generate
simple line plots. This class is freely available from the following URL:
{\tt http://www.math.ucla.edu/\-\verb{~{anderson/CAMclass/CAMClass.html}

	The CAM graphics class actually generates a PostScript\footnote{PostScript
is a registered trademark of Adobe Systems Incorporated.}  file. 
 Postscript is a programming language that describes the appearance of a printed page. 
It was developed by Adobe in 1985, and has become an industry standard for printing and
 imaging. All major printer manufacturers make printers that can interpret PostScript.
 A PostScript file is conventionally
  identified via a {\tt .ps} suffix. The schematic code listed below illustrates the
basic use of the CAM graphics class:
{\small\begin{verbatim}
. . .
#include <gprocess.h> // Header file for CAM graphic class
. . .
CAMgraphicsProcess Gprocess;                 // declare a graphics process
CAMpostScriptDriver Pdriver("filename.ps");  // declare a PostScript driver
Gprocess.attachDriver(Pdriver);              // attach driver to process
. . .
Gprocess.frame();                            // "frame" the first plot
. . .
Gprocess.frame();                            // "frame" the second plot
. . .
. . .  
Gprocess.frame();                            // "frame" the last plot
Gprocess.detachDriver();                     // detach the driver
. . .
\end{verbatim}}\noindent
The header file for the class is called {\tt gprocess.h}. The procedure for
generating a plot is to first declare a graphics process, then declare a
PostScript driver and attach this to a PostScript file---{\tt filename.ps}, in the
above example---and, finally, attach this driver to the process. A PostScript
file can contain multiple pictures, or {\em frames}. Each frame is terminated
by a call to {\tt Gprocess.frame()}. Finally, the driver is detached, which has the
effect of closing the PostScript file. 

The program listed below uses the CAM graphics class to plot the curve $y=\sin^2 x$
for $x$ in the range $-2\pi$ to $2\pi$. 
{\small\begin{verbatim}
/* camgraph1.cpp */
/*
  Illustration of use of CAM graphics class to create simple line plot

  Program plots y = sin^2 x versus x in range -2 PI to +2 PI

  Program adapted from gpsmp1.cpp by Chris Anderson, UCLA 1996
*/

#include <gprocess.h>
#include <math.h>
#include <stdlib.h>

double func(double);

int main()
{
  int N_points = 400;
  double x_start = -2. * M_PI;                
  double x_end = 2. * M_PI; 
  double delta_x = (x_end - x_start) / ((double) N_points - 1.);

  double *x = new double[N_points];  
  double *y = new double[N_points];

  for (int i = 0; i < N_points; i++)
    {
      x[i] = x_start + (double) i * delta_x;
      y[i] = func(x[i]);
      x[i] /= M_PI;
    }

  { // This brace used to limit scope of Gprocess
    CAMgraphicsProcess Gprocess;                // declare a graphics process
    CAMpostScriptDriver Pdriver("graph1.ps");   // declare a PostScript driver
    Gprocess.attachDriver(Pdriver);             // attach driver to process

    Gprocess.setAxisRange(-2., 2., -2., 2.);    // set plotting ranges
    Gprocess.title("y = sin(x*x)");             // label the plot
    Gprocess.labelX("x / PI");
    Gprocess.labelY("y");

    Gprocess.plot(x, y, N_points);              // do the plotting

    Gprocess.frame();                           // "frame" the plot

    Gprocess.detachDriver();                    // detach the driver
  } // This brace calls the destructor for Gprocess:
    // without it the system() call would hang up

  delete[] x;	           
  delete[] y;

  system("gv graph1.ps");  // display plot on screen         

  return 0;
}

double func(double x)
{ 
  return sin(x*x);
}
\end{verbatim}}\noindent
The command {\tt Gprocess.plot(x, y, n)} plots the \verb{n{ values of
vector {\tt y} against the \verb{n{ values of vector {\tt x} as a solid curve.
The command {\tt Gprocess.set\-Axis\-Range\-(x\verb{_{low, x\verb{_{high, y\verb{_{low, y\verb{_{high)} sets the
range of plotting. Finally, the commands {\tt Gprocess.\-title("title")},  
{\tt Gprocess.\-labelX("x\verb{_{label")},
and {\tt Gprocess.\-labelY\-("y\verb{_{label")} label the plot, the $x$-axis, and the $y$-axis, respectively.
Incidentally, the UNIX function call {\tt system("gv graph1.ps")} is used to pass the command
{\tt gv graph1.ps} to the operating system. On execution, this command displays
the contents of {\tt graph1.ps} on the screen.
The graph written in the file {\tt graph1.ps} is shown in Fig~\ref{xxx}

\begin{figure}[h]
\epsfysize=2.5in
\centerline{\epsffile{Chapter02/graph1.ps}}
\caption{\em An example plot generated by the CAM graphics class.}\label{xxx}
\end{figure}

The program shown below illustrates some of the more advanced features of the CAM graphics
class:
{\small\begin{verbatim}
/* camgraph2.cpp */
/*
  Illustration of use of CAMgraphics class to create more advanced line plots

  Program plots three trigonometric functions versus x in range 
   -2 PI to +2 PI using different plot styles and different
   line styles

  Program adapted from gpsmp2.cpp by Chris Anderson, UCLA 1996
*/

#include <gprocess.h>
#include <math.h>
#include <stdlib.h>

double fun1(double);
double fun2(double);
double fun3(double);

int main()
{
  int N_points = 100;
  double x_start = -2. * M_PI;                
  double x_end = 2. * M_PI; 
  double delta_x = (x_end - x_start) / ((double) N_points - 1.);
  
  double *x  = new double[N_points];
  double *y1 = new double[N_points];
  double *y2 = new double[N_points];
  double *y3 = new double[N_points];

  for (int i = 0; i < N_points; i++)
    {
      x[i] = x_start + (double) i * delta_x;
      y1[i] = fun1(x[i]);
      y2[i] = fun2(x[i]);
      y3[i] = fun3(x[i]);
      x[i] /= M_PI;
    }

  {
    CAMgraphicsProcess Gprocess;              // declare a graphics process
    CAMpostScriptDriver Pdriver("graph2.ps"); // declare a PostScript driver
    Gprocess.attachDriver(Pdriver);           // attach driver to process
    
    /* First frame;  using different plot "styles" */
    Gprocess.setAxisRange(-2., 2., -2., 2.);            // set plotting ranges
    Gprocess.title("Plots Using Different Plot Styles");// label the plot
    Gprocess.labelX("x / PI");
    Gprocess.labelY("y");

    Gprocess.plot(x, y1, N_points);              // solid line (default)
    Gprocess.plot(x, y2, N_points, '+');         // + markers
    Gprocess.plot(x, y3, N_points, '+', 2);      // + markers and solid line

    Gprocess.frame();                            // "frame" the plot

    /* Second frame; using different plot line "styles" */
    Gprocess.setAxisRange(-2., 2., -2., 2.);            // set plotting ranges
    Gprocess.title("Plots Using Different Line Styles");// label the plot
    Gprocess.labelX("x / PI");
    Gprocess.labelY("y");

    Gprocess.plot(x, y1, N_points);         // solid line (default)
    Gprocess.setPlotDashPattern(1);
    Gprocess.plot(x, y2, N_points);         // dashed line
    Gprocess.setPlotDashPattern(4);
    Gprocess.plot(x, y3, N_points);         // dashed-dot line

    Gprocess.frame();                       // "frame" the plot

    Gprocess.detachDriver();                // detach the driver
  }

  delete[] x;
  delete[] y1;
  delete[] y2;
  delete[] y3;

  system("gv graph2.ps"); // display plots on screen     
  
  return 0;
}

double fun1(double x)
{ 
  return sin(x);
}

double fun2(double x)
{ 
  return cos(x);
}

double fun3(double x)
{ 
  return cos(2.*x);
}
\end{verbatim}}\noindent
The command {\tt Gprocess.plot(x, y, n, '+')} plots the \verb{n{ values of
vector {\tt y} against the \verb{n{ values of vector {\tt x} as a set of points,
each indicated by a {\tt '+'} character. 
The command {\tt Gprocess.plot(x, y, n, '+', 2)} does the same, but also
connects the points with a solid line. The fourth argument of this command is
an integer code which determines the plot style. The various
options are as follows: {\tt 0} - curve; {\tt 1} - points; {\tt 2} - curve and
points.  The command {\tt Gprocess.setPlot\-Dash\-Pattern(n)} sets the line style.
The argument is again an integer code. The various options are:
{\tt 0} - solid; {\tt 1} - dash; {\tt 2} - double-dash; {\tt 4} - dash-dot;
{\tt 5} - dash-double-dot; {\tt 6} - dots.

 The graphs written in the first and second frames of  {\tt graph2.ps} are shown in 
Figs.~\ref{fg2a} and \ref{fg2b}, respectively.

\begin{figure}
\epsfysize=2.5in
\centerline{\epsffile{Chapter02/graph2a.eps}}
\caption{\em An example plot generated by the CAM graphics class.}\label{fg2a}
\end{figure}


\begin{figure}
\epsfysize=2.5in
\centerline{\epsffile{Chapter02/graph2b.eps}}
\caption{\em An example plot generated by the CAM graphics class.}\label{fg2b}
\end{figure}
