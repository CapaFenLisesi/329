\chapter{Particle-in-Cell Codes}
\section{Introduction}
Consider an unmagnetized, uniform, 1-dimensional plasma consisting of $N$ electrons and $N$ unit-charged 
ions.
Now,  ions are much more massive than  electrons.
Hence, on short time-scales, we can treat the ions as a static 
neutralizing background, and only consider the motion of the electrons. Let $r_i$ be the
$x$-coordinate of the $i$th electron. The equations of motion of the
$i$th electron are written:
\begin{eqnarray}
\frac{dr_i}{dt} &=& v_i,\\[0.5ex]
\frac{dv_i}{dt} &=& - \frac{e\,E(r_i)}{m_e},
\end{eqnarray}
where $e>0$ is the magnitude of the electron charge, $m_e$  the electron mass, and $E(x)$  the $x$-component
of the electric field-strength at position $x$. Now, the electric field-strength can be
expressed in terms of an electric
potential:
\begin{equation}
E(x) = - \frac{d\phi(x)}{dx}.
\end{equation}
Furthermore, from the Poisson-Maxwell equation, we have
\begin{equation}
\frac{d^2\phi(x)}{dx^2} = -\frac{e}{\epsilon_0}\,\{n_0- n(x)\},
\end{equation}
where $\epsilon_0$ is the permittivity of free-space, $n(x)$  the electron number
density ({\em i.e.}, $n(x)\,dx$ is the number of electrons in the interval $x$ to $x+dx$),
and $n_0$  the uniform ion number density. Of course, the average value of $n(x)$ is
equal to $n_0$, since there are equal numbers of ions and electrons. 

Let us consider an initial electron distribution function consisting of two counter-propagating
Maxwellian beams of mean speed $v_b$ and thermal spread $v_{th}$: {\em i.e.},
\begin{equation}
f(x,v) = \frac{n_0}{2}\left\{\frac{1}{\sqrt{2\,\pi}\,v_{th}}
\,{\rm e}^{-(v-v_b)^2/2\,v_{th}^{\,2}}+ \frac{1}{\sqrt{2\,\pi}\,v_{th}}\,{\rm e}^{-(v+v_b)^2/2\,v_{th}^{\,2}}
\right\}.
\end{equation}
Here, $f(x,v)\,dx\,dv$ is the number of electrons between $x$ and $x+dx$ with velocities
in the range $v$ to $v+dv$. Of course, $n(x) = \int_{-\infty}^{\infty} f(x,v)\,dv$.
The beam temperature $T$ is related to the thermal velocity via $v_{th} = \sqrt{k_B\,T/m_e}$,
where $k_B$ is the Boltzmann constant. It is well-known that if $v_b$ is significantly
larger than $v_{th}$ then the above distribution is unstable to a plasma instability
called the {\em two-stream instability}.\footnote{T.H.~Stix, {\em The theory of plasma waves}, 1st Ed.
(McGraw-Hill, New York NY, 1962).} Let us investigate this instability numerically.

\section{Normalization Scheme}
It is convenient to normalize time with respect to $\omega_p^{-1}$, where
\begin{equation}
\omega_p^2 = \frac{n_0\,e^2}{\epsilon_0\,m_e}
\end{equation}
is the  so-called {\em plasma frequency}: {\em i.e.}, the typical frequency of electrostatic electron oscillations. Likewise it is convenient to normalize length with respect to the so-called {\em Debye length}:
$$
\lambda_D = \frac{v_{th}}{\omega_p},
$$
which is the length-scale above which the electrons exhibit collective ({\em i.e.},
plasma-like) effects, instead of acting like individual particles.

Our normalized equations take the form:
\begin{eqnarray}
\frac{dx_i}{dt} &=&v_i,\label{eqm1}\\[0.5ex]
\frac{d v_i}{dt} &=& - E(x_i),\label{eqm2}\\[0.5ex]
E(x) &=& - \frac{d\phi(x)}{dx},\label{eqm4}\\[0.5ex]
\frac{d^2\phi(x)}{dx^2} &=& \frac{n(x)}{n_0}-1.\label{eqm3}
\end{eqnarray}
whereas our initial distribution function becomes
\begin{equation}
f(x,v) = \frac{n_0}{2}\left\{\frac{1}{\sqrt{2\,\pi}}
\,{\rm e}^{-(v-v_b)^2/2}+ \frac{1}{\sqrt{2\,\pi}}\,{\rm e}^{-(v+v_b)^2/2}
\right\}.
\end{equation}
Note that $v_{th}=1$ in normalized units.

Let us solve the above system of equations in the domain $0\leq x\leq L$. 
Furthermore, for the sake of simplicity, let us adopt {\em periodic} boundary conditions:
{\em i.e.}, let us identify the left and right boundaries of our solution domain.
It follows that $n(0)=n(L)$,  $\phi(0)=\phi(L)$, and $E(0) = E(L)$. Moreover, any electron
which crosses the right boundary of the solution domain must reappear at the left boundary with
the same velocity, and
{\em vice versa}.

\section{Solution of Electron Equations of Motion}
We can solve the electron equations of motion, (\ref{eqm1}) and (\ref{eqm2}), as a set
of $2N$ coupled first-order ODEs using the RK4 methods discussed in Sect.~\ref{ode}.
However, in order to evaluate the right-hand sides of these equations we need to know
the electric field $E(x)$  at each time-step.
We can achieve this by solving Poisson's equation, (\ref{eqm3}), every time-step.
However, in order to determine the source term for this equation we need to calculate the
electron number density $n(x)$, which is, of course, a function of the instantaneous
electron locations.

\section{Evaluation of Electron Number Density}
In order to obtain the electron number density $n(x)$
from the electron coordinates $r_i$ we adopt a so-called {\em particle-in-cell} (PIC) approach.
Let us define a set of $J$ equally
spaced spatial grid-points located at coordinates
\begin{equation}
x_j = j\,\delta x,
\end{equation}
for $j=0,J-1$, where $\delta x = L / J$.  Let $n_j\equiv n(x_j)$.
Suppose that the $i$th electron lies between the $j$th and $(j+1)$th grid-points:
{\em i.e.}, $x_j < r_i < x_{j+1}$. We let
\begin{equation}
n_j  \rightarrow n_j+\left.\left(\frac{x_{j+1}-r_i}{x_{j+1}-x_j}\right)\right/\delta x,
\end{equation}
and
\begin{equation}
n_{j+1} \rightarrow n_{j+1}+ \left.\left(\frac{r_i-x_j}{x_{j+1}-x_j}\right)\right/\delta x.
\end{equation}
Thus, $n_j\,\delta x$ increases by 1 if the electron is at the $j$th grid-point, $n_{j+1}\,\delta x$ 
increases by 1 if the electron is at the $(j+1)$th grid-point, and $n_j\,\delta x$ and $n_{j+1}\,\delta x$
both increase by $1/2$ if the electron is halfway between the two grid-points, {\em etc.}
Performing a similar assignment for each electron in turn allows us to
build up the $n_j$ from the electron coordinates (assuming that all the $n_j$ are initialized
to zero at the start of this process).


\section{Solution of Poisson's Equation}
Consider the solution of Poisson's equation:
\begin{equation}
\frac{d^2\phi(x)}{dx^2} = \rho(x),
\end{equation}
where $\rho(x) = n(x)/n_0-1$. Note that $n_0 = N/L$ in normalized
units. Let $\phi_j\equiv\phi(x_j)$ and $\rho_j\equiv
\rho(x_j)$. 
We can write
\begin{eqnarray}
\phi_j &=& \sum_{j'=0,J-1}\hat{\phi}_{j'} \,{\rm e}^{\,{\rm i}\,j\,j'\,2\,\pi/J},\\[0.5ex]
\rho_j &=& \sum_{j'=0,J-1}\hat{\rho}_{j'} \,{\rm e}^{\,{\rm i}\,j\,j'\,2\,\pi/J},
\end{eqnarray}
which automatically satisfies the periodic boundary conditions $\phi_J=\phi_0$ and $\rho_J=\rho_0$. 
Note that $\hat{\rho}_0=0$, since $\int_0^L n(x)\,dx = n_0$. The other $\hat{\rho}_j$ are obtained
from
\begin{equation}
\hat{\rho}_j = \frac{1}{J}\sum_{j'=0,J-1}\rho_{j'} \,{\rm e}^{-{\rm i}\,j\,j'\,2\,\pi/J},
\end{equation}
for $j=1,J-1$.
The Fourier transformed version of Poisson's equation yields
\begin{equation}
\hat{\phi}_0 = 0
\end{equation}
and
\begin{equation}
\hat{\phi}_j = - \frac{\hat{\rho}_j}{j^2\,\kappa^2}
\end{equation}
for $j=1,J/2$, where $\kappa = 2\pi / L$. Finally,
\begin{equation}
\hat{\phi}_j = \hat{\phi}^\ast_{J-j}
\end{equation}
for $j=J/2+1$ to $J-1$, which ensures that the $\phi_j$ remain real.
The discretized version of Eq.~(\ref{eqm4}) is
\begin{equation}
E_j = \frac{\phi_{j-1}-\phi_{j+1}}{2\,\delta x}.
\end{equation}
Of course, $j=0$ and $j=J-1$ are special cases which can be resolved using the periodic
boundary conditions.
Finally, suppose that the coordinate of the $i$th electron lies between the $j$th and $(j+1)$th
grid-points: {\em i.e.}, $x_j < r_i < x_{j+1}$.  We can then use linear interpolation to
evaluate the electric field seen by the $i$th electron:
\begin{equation}
E(r_i) = \left(\frac{x_{j+1}-r_i}{x_{j+1}-x_j}\right)E_j+\left(\frac{r_i - x_j}{x_{j+1}-x_j}\right)
E_{j+1}.
\end{equation}

\section{An example 1-D PIC Code}
The following code  is an implementation of the ideas developed above.

The main function  reads in the calculation parameters, checks that they
are sensible, initializes the electron coordinates, and then evolves the
electron equations of motion from $t=0$ to some specified $t_{\rm max}$,
using a fixed step RK4 routine with some specified time-step $\delta t$.
Information on the electron phase-space coordinates and the electric field is periodically written
to various data-files.
\begin{verbatim}
// 1-d PIC code to solve plasma two-stream instability problem.

#include <stdlib.h>
#include <stdio.h>
#include <math.h>
#include <time.h>
#include <blitz/array.h>
#include <fftw.h>

using namespace blitz;

void Output (char* fn1, char* fn2, double t, 
	     Array<double,1> r, Array<double,1> v);
void Density (Array<double,1> r, Array<double,1>& n);
void Electric (Array<double,1> phi, Array<double,1>& E);
void Poisson1D (Array<double,1>& u, Array<double,1> v, double kappa);
void rk4_fixed (double& x, Array<double,1>& y, 
              void (*rhs_eval)(double, Array<double,1>, Array<double,1>&), 
              double h);
void rhs_eval (double t, Array<double,1> y, Array<double,1>& dydt);
void Load (Array<double,1> r, Array<double,1> v, Array<double,1>& y);
void UnLoad (Array<double,1> y, Array<double,1>& r, Array<double,1>& v);
double distribution (double vb);

double L; int N, J;

int main()
{
  // Parameters
  L;            // Domain of solution 0 <= x <= L (in Debye lengths)
  N;            // Number of electrons
  J;            // Number of grid points
  double vb;    // Beam velocity
  double dt;    // Time-step (in inverse plasma frequencies)
  double tmax;  // Simulation run from t = 0. to t = tmax

  // Get parameters
  printf ("Please input N:  "); scanf ("%d", &N);
  printf ("Please input vb:  "); scanf ("%lf", &vb); 
  printf ("Please input L:  "); scanf ("%lf", &L);
  printf ("Please input J:  "); scanf ("%d", &J);
  printf ("Please input dt:  "); scanf ("%lf", &dt); 
  printf ("Please input tmax:  "); scanf ("%lf", &tmax);
  int skip = int (tmax / dt) / 10;
  if ((N < 1) || (J < 2) || (L <= 0.) || (vb <= 0.) 
      || (dt <= 0.) || (tmax <= 0.) || (skip < 1))
    {
      printf ("Error - invalid input parameters\n");
      exit (1);
    }

  // Set names of output files
  char* phase[11]; char* data[11];
  phase[0] = "phase0.out";phase[1] = "phase1.out";phase[2] = "phase2.out"; 
  phase[3] = "phase3.out";phase[4] = "phase4.out";phase[5] = "phase5.out"; 
  phase[6] = "phase6.out";phase[7] = "phase7.out";phase[8] = "phase8.out";
  phase[9] = "phase9.out";phase[10] = "phase10.out";data[0] = "data0.out";  
  data[1] = "data1.out"; data[2] = "data2.out"; data[3] = "data3.out"; 
  data[4] = "data4.out"; data[5] = "data5.out"; data[6] = "data6.out"; 
  data[7] = "data7.out"; data[8] = "data8.out"; data[9] = "data9.out"; 
  data[10] = "data10.out";

  // Initialize solution
  double t = 0.;
  int seed = time (NULL); srand (seed);
  Array<double,1> r(N), v(N);
  for (int i = 0; i < N; i++)
    {
      r(i) = L * double (rand ()) / double (RAND_MAX);
      v(i) = distribution (vb);
    }
  Output (phase[0], data[0], t, r, v);

  // Evolve solution
  Array<double,1> y(2*N);
  Load (r, v, y);
  for (int k = 1; k <= 10; k++)
    {
      for (int kk = 0; kk < skip; kk++)
        {
           // Take time-step
           rk4_fixed (t, y, rhs_eval, dt);

           // Make sure all coordinates in range 0 to L.
           for (int i = 0; i < N; i++)
             {
               if (y(i) < 0.) y(i) += L;
               if (y(i) > L) y(i) -= L;
             }
	  
           printf ("t = %11.4e\n", t);
        }
      printf ("Plot %3d\n", k);

      // Output data
      UnLoad (y, r, v);
      Output(phase[k], data[k], t, r, v);
    }

  return 0;
}
\end{verbatim}

The following routine outputs the simulation data to various data-files.
\begin{verbatim}
// Write data to output files

void Output (char* fn1, char* fn2, double t,
	     Array<double,1> r, Array<double,1> v)
{  
  // Write phase-space data
  FILE* file = fopen (fn1, "w");
  for (int i = 0; i < N; i++)
    fprintf (file, "%e %e\n", r(i), v(i));
  fclose (file);

  // Write electric field data
  Array<double,1> ne(J), n(J), phi(J), E(J);  
  Density (r, ne);
  for (int j = 0; j < J; j++)
    n(j) = double (J) * ne(j) / double (N) - 1.;
  double kappa = 2. * M_PI / L; 
  Poisson1D (phi, n, kappa);
  Electric (phi, E);

  file = fopen (fn2, "w");
  for (int j = 0; j < J; j++)
    {
      double x = double (j) * L / double (J);
      fprintf (file, "%e %e %e %e\n", x, ne(j), n(j), E(j));
    }
  double x = L;
  fprintf (file, "%e %e %e %e\n", x, ne(0), n(0), E(0));
  fclose (file);
}
\end{verbatim}

The following routine returns a random velocity distributed on a double Maxwellian distribution function corresponding
to two counter-streaming beams. The algorithm used to achieve this is
called the {\em rejection method}, and will be discussed later in
this course.
\begin{verbatim}
// Function to distribute electron velocities randomly so as 
// to generate two counter propagating warm beams of thermal
// velocities unity and mean velocities +/- vb.
// Uses rejection method.

double distribution (double vb)
{ 
  // Initialize random number generator
  static int flag = 0;
  if (flag == 0)
    {
      int seed = time (NULL);
      srand (seed);
      flag = 1;
    }

  // Generate random v value
  double fmax = 0.5 * (1. + exp (-2. * vb * vb));
  double vmin = - 5. * vb;
  double vmax = + 5. * vb;
  double v = vmin + (vmax - vmin) * double (rand ()) / double (RAND_MAX);

  // Accept/reject value
  double f = 0.5 * (exp (-(v - vb) * (v - vb) / 2.) +
		    exp (-(v + vb) * (v + vb) / 2.));
  double x = fmax * double (rand ()) / double (RAND_MAX);
  if (x > f) return distribution (vb);
  else return v;
}
\end{verbatim}

The routine below evaluates the electron number density on an evenly
spaced mesh given the instantaneous electron coordinates.
\begin{verbatim}
// Evaluates electron number density n(0:J-1) from 
// array r(0:N-1) of electron coordinates.

void Density (Array<double,1> r, Array<double,1>& n)
{
  // Initialize 
  double dx = L / double (J);
  n = 0.;

  // Evaluate number density.
  for (int i = 0; i < N; i++)
    {
      int j = int (r(i) / dx);
      double y = r(i) / dx - double (j);
      n(j) += (1. - y) / dx;
      if (j+1 == J) n(0) += y / dx;
      else n(j+1) += y / dx;
    }
}
\end{verbatim}

The following functions are wrapper routines for using the fftw library
with periodic functions.
\begin{verbatim}
// Functions to calculate Fourier transforms of real data 
// using fftw Fast-Fourier-Transform routine.
// Input/ouput arrays are assumed to be of extent J.

// Calculates Fourier transform of array f in arrays Fr and Fi
void fft_forward (Array<double,1>f, Array<double,1>&Fr, 
      Array<double,1>& Fi)
{
  fftw_complex ff[J], FF[J];

  // Load data
  for (int j = 0; j < J; j++)
    {
      c_re (ff[j]) = f(j); c_im (ff[j]) = 0.;
    }

  // Call fftw routine
  fftw_plan p = fftw_create_plan (J, FFTW_FORWARD, FFTW_ESTIMATE);
  fftw_one (p, ff, FF);
  fftw_destroy_plan (p); 

  // Unload data
  for (int j = 0; j < J; j++)
    {
      Fr(j) = c_re (FF[j]); Fi(j) = c_im (FF[j]);
    }

  // Normalize data
  Fr /= double (J);
  Fi /= double (J);
}

// Calculates inverse Fourier transform of arrays Fr and Fi in array f
void fft_backward (Array<double,1> Fr, Array<double,1> Fi, 
      Array<double,1>& f)
{
  fftw_complex ff[J], FF[J];

  // Load data
  for (int j = 0; j < J; j++)
    {
      c_re (FF[j]) = Fr(j); c_im (FF[j]) = Fi(j);
    }

  // Call fftw routine
  fftw_plan p = fftw_create_plan (J, FFTW_BACKWARD, FFTW_ESTIMATE);
  fftw_one (p, FF, ff);
  fftw_destroy_plan (p); 

  // Unload data
  for (int j = 0; j < J; j++)
      f(j) = c_re (ff[j]); 
}
\end{verbatim}

The following routine solves Poisson's equation in 1-D to find the
instantaneous electric potential on a uniform grid.
\begin{verbatim}
// Solves 1-d Poisson equation:
//    d^u / dx^2 = v   for  0 <= x <= L
// Periodic boundary conditions:
//    u(x + L) = u(x),  v(x + L) = v(x)
// Arrays u and v assumed to be of length J.
// Now, jth grid point corresponds to
//    x_j = j dx  for j = 0,J-1
// where dx = L / J.
// Also,
//    kappa = 2 pi / L

void Poisson1D (Array<double,1>& u, Array<double,1> v, double kappa)
{
  // Declare local arrays.
  Array<double,1> Vr(J), Vi(J), Ur(J), Ui(J);

  // Fourier transform source term
  fft_forward (v, Vr, Vi);

  // Calculate Fourier transform of u
  Ur(0) = Ui(0) = 0.;
  for (int j = 1; j <= J/2; j++)
    {
      Ur(j) = - Vr(j) / double (j * j) / kappa / kappa;
      Ui(j) = - Vi(j) / double (j * j) / kappa / kappa;
    } 
  for (int j = J/2; j < J; j++)
    {
      Ur(j) = Ur(J-j);
      Ui(j) = - Ui(J-j);
    }

  // Inverse Fourier transform to obtain u
  fft_backward (Ur, Ui, u);
}
\end{verbatim}

The following function evaluates the electric field on a uniform grid
from the electric potential.
\begin{verbatim}
// Calculate electric field from potential

void Electric (Array<double,1> phi, Array<double,1>& E)
{
  double dx = L / double (J);

  for (int j = 1; j < J-1; j++)
    E(j) = (phi(j-1) - phi(j+1)) / 2. / dx;
  E(0) = (phi(J-1) - phi(1)) / 2. / dx;
  E(J-1) = (phi(J-2) - phi(0)) / 2. / dx;
}
\end{verbatim}

The following routine is the right-hand side routine for the electron
equations of motion. Is is designed to be used with the fixed-step
RK4 solver described earlier in this course.
\begin{verbatim}
// Electron equations of motion:
//    y(0:N-1)  = r_i
//    y(N:2N-1) = dr_i/dt

void rhs_eval (double t, Array<double,1> y, Array<double,1>& dydt)
{
  // Declare local arrays
  Array<double,1> r(N), v(N), rdot(N), vdot(N), r0(N);
  Array<double,1> ne(J), rho(J), phi(J), E(J);

  // Unload data from y
  UnLoad (y, r, v);

  // Make sure all coordinates in range 0 to L
  r0 = r;
  for (int i = 0; i < N; i++)
    {
      if (r0(i) < 0.) r0(i) += L;
      if (r0(i) > L) r0(i) -= L;
    }

  // Calculate electron number density
  Density (r0, ne);

  // Solve Poisson's equation
  double n0 = double (N) / L;
  for (int j = 0; j < J; j++)
    rho(j) = ne(j) / n0 - 1.;
  double kappa = 2. * M_PI / L; 
  Poisson1D (phi, rho, kappa);

  // Calculate electric field
  Electric (phi, E);

  // Equations of motion
  for (int i = 0; i < N; i++)
    {
      double dx = L / double (J);
      int j = int (r0(i) / dx);
      double y = r0(i) / dx - double (j);
      
      double Efield;
      if (j+1 == J)
         Efield = E(j) * (1. - y) + E(0) * y;
      else
         Efield = E(j) * (1. - y) + E(j+1) * y;

      rdot(i) = v(i);
      vdot(i) = - Efield;
    }

  // Load data into dydt
  Load (rdot, vdot, dydt);
}
\end{verbatim}


The following functions load and unload the electron phase-space
coordinates into the solution vector \verb!y! used by the RK4 routine.
\begin{verbatim}
// Load particle coordinates into solution vector

void Load (Array<double,1> r, Array<double,1> v, Array<double,1>& y)
{
  for (int i = 0; i < N; i++)
    {
      y(i) = r(i);
      y(N+i) = v(i);
    }
}

// Unload particle coordinates from solution vector

void UnLoad (Array<double,1> y, Array<double,1>& r, Array<double,1>& v)
{
  for (int i = 0; i < N; i++)
    {
      r(i) = y(i);
      v(i) = y(N+i);
    }
}
\end{verbatim}

\begin{figure}
\epsfysize=3.5in
\centerline{\epsffile{Chapter08/pic1.eps}}
\caption{\em The electron phase-space distribution evaluated
at various times for a 1-dimensional two-stream instability calculation performed with $N=20000$, $J=1000$, $L=100$, $v_b = 3$, and $\delta t =0.1$.}\label{pic1}
\end{figure}

\section{Results}
Figures~\ref{pic1} and \ref{pic2} show the electron phase-space distributions
evaluated at equally spaced times for a two-stream instability calculation
performed with $2\times 10^4$ electrons. It can be seen that the distribution
initially takes the form of two uniform bands, corresponding to two
counter-streaming electron beams. However, as time progresses, the
bands spontaneously develop structure which grows in magnitude and
eventually converts the phase-space distribution into a set of connected
vortices. In this final state, the electrons are basically bouncing backwards
and forwards in  a quasi-periodic electric potential generated by non-uniformities
in the electron density. In other words,  the instability effectively destroys the two
beams. For this reason, the two-stream instability is of major
concern in particle accelerators, which often consist of counter-propagating
charged particle beams.

\begin{figure}
\epsfysize=3.5in
\centerline{\epsffile{Chapter08/pic2.eps}}
\caption{\em The electron phase-space distribution evaluated
at various times for a 1-dimensional two-stream instability calculation performed with $N=20000$, $J=1000$, $L=100$, $v_b = 3$, and $\delta t =0.1$.}\label{pic2}
\end{figure}

\section{Discussion}
Obviously, the ideas discussed above could be generalized in a fairly
straight-forward manner to deal with the evolution of two and three-dimensional charged particle distributions. PIC codes have the advantage that
they are reasonably straight-forward to write. Unfortunately, PIC codes
also have a number of disadvantages. The first is that PIC codes suffer from
high levels of statistical noise,  since they generally only deal with a
relatively small number of particles (typically, $\leq 10^6$). Real 
physical systems
do not exhibit anything like the same level of statistical noise, since they generally
contain of order Avogadro's number ($\sim 10^{24}$) of interacting particles.  Another problem
with PIC codes is that they do not handle charged particle collisions very
well. The reason for this is that there are generally a large number
of particles in each cell (for practical reasons),  and the short range Coulomb fields of these
particles tend to cancel one another out (recall that the electric field is only
calculated at the cell vertices).
